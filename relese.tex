	 %%%%%%%%%%%%%%%%%%%%%
%\documentclass[draft]{amsart}
\documentclass{article}
\usepackage{blindtext}
\usepackage{bm}
\usepackage{psfrag}
\usepackage[usenames,dvipsnames]{color}
\usepackage{xcolor}
\usepackage[all,cmtip,line]{xy}
\usepackage[normalem]{ulem}
\usepackage{tikz} 
\usepackage[utf8]{inputenc}
\usepackage{pgfplots}
%\usepackage{subfig}
\usepackage{scalerel}
\usepackage{amssymb}
\usepackage{caption}
\usepackage{amsmath}
\usepackage{mathabx}
\usepackage{subcaption}
\usepackage[shortlabels]{enumitem}
\usepackage[]{algorithm2e}


% Inverted breve
\usepackage[T3,T1]{fontenc}
\DeclareSymbolFont{tipa}{T3}{cmr}{m}{n}
\DeclareMathAccent{\invbreve}{\mathalpha}{tipa}{16}
\DeclareMathOperator*{\argmin}{arg\,min}
\newcommand{\msout}[1]{\text{\sout{\ensuremath{#1}}}}
%CS Macros 
\newtheorem{theorem}{Theorem}[section]
\newtheorem{lemma}[theorem]{Lemma}
\newtheorem{corollary}[theorem]{Corollary}
\newtheorem{proposition}[theorem]{Proposition} 

\newtheorem{problem}[theorem]{Problem}
\newtheorem{definition}[theorem]{Definition}
\newtheorem{example}[theorem]{Example}
\newtheorem{xca}[theorem]{Exercise}

\newtheorem{remark}[theorem]{Remark}
\newtheorem{assumption}[theorem]{Assumption}
\newcommand{\dc}{downward closed }
%CJ Macros

\newenvironment{proof}{\paragraph{Proof:}}{\hfill$\square$}

\newtheorem{obs}{Observation}[section]
\newtheorem{prop}{Property}[section]
%\newcommand{\norm}[2]{\left\lVert #1\right\rVert_{#2}}
\newcommand{\seminorm}[2]{| #1 |_{#2}}
\newcommand{\RR}{\mathbb{R}}
\newcommand{\vx}{\bm{x}}
\newcommand{\curl}{\operatorname{\bm{curl}}}
\renewcommand{\div}{\operatorname{div}}
%\newcommand{\bL}{\boldsymbol{L}}
\newcommand{\p}[1]{\langle #1\rangle}
\newcommand{\pr}[1]{\left( #1\right)}
\newcommand{\bil}[1]{\kappa\left( #1 \right)}
\newcommand{\modulo}[1]{\left\vert #1\right\vert}
\newcommand{\dv}[1]{{\rm div}\left( #1\right)}
\newcommand{\epep}[1]{\epsilon \left( #1 \right):\epsilon\left( #1\right)}
\newcommand{\hcurl}[1]{\bm{H}\left( \curl; #1 \right)}
\newcommand{\hocurl}[1]{\bm{H}_0(\curl; #1 )}
\newcommand{\hdiv}[1]{\bm{H}(\div;#1)}
\newcommand{\hodiv}[1]{\bm{H}_0(\div;#1)}
\newcommand{\Hsob}[2]{\bm{H}^{#1}( #2 )}
\newcommand{\hsob}[2]{{H}^{#1}( #2 )}
\newcommand{\Hosob}[2]{\bm{H}_0^{#1}( #2 )}
\newcommand{\hosob}[2]{{H}_0^{#1}( #2 )}
\newcommand{\Lp}[2]{\bm{L}^{#1}( #2 )}
\newcommand{\lp}[2]{{L}^{#1}( #2 )}
\newcommand{\hil}[1]{\mathcal{#1}}

\hyphenation{pa-ra-me-tri-zed}

%cs color for changes by CJH
\newcommand{\cj}[1]{{\color{magenta}{#1}}}
\definecolor{forestgreen}{rgb}{0.13, 0.55, 0.13}
\newcommand{\ra}[1]{{\color{forestgreen}#1}}
\newcommand{\ras}[1]{{\color{forestgreen}{\sout{#1}}}}

\newcommand{\jp}[1]{{\color{blue}#1}}

\newcommand{\todo}[1]{{\color{red}[#1]}}


\newcommand{\KK}{{\rm K}}
\newcommand{\LL}{{\rm L}}
\newcommand{\HH}{{\rm H}}
%\newcommand{\RR}{\mathbb{R}}
\newcommand{\CC}{\rm C}
\newcommand{\KL}{\mbox{Karh\'{u}nen-Lo\`{e}ve }}
\newcommand{\Hol}{\mbox{Hol}}
%%%%%%%%%%%%%%%%%%%%%%%%%%%%%%%%%%%%%%%%%%%%%
\newcommand{\be}{\begin{equation}}
\newcommand{\ee}{\end{equation}}
%color for changes by CJ
%\definecolor{forest}{rgb}{0.3,0.4,0.1}
%\definecolor{Ora}{cmyk}{0.2, 0.6, 0.8, 0}
% definitions
\newcommand{\eps}{{\varepsilon}}
\renewcommand{\c}{{\boldsymbol c}}
\newcommand{\bmb}{{\boldsymbol b}}
\newcommand{\set}[2]{\{#1\,:\,#2\}}
% cal letters
\newcommand{\cA}{\mathcal A}
\newcommand{\cB}{\mathcal B}
\newcommand{\C}{\mathcal C}
\newcommand{\cD}{\mathcal D}
\newcommand{\E}{\mathcal E}
\newcommand{\cE}{\mathcal E}
\newcommand{\cF}{\mathcal F}
\newcommand{\cJ}{\mathcal J}
\newcommand{\e}{{\bm e}}
\newcommand{\f}{{\bm f}}
\newcommand{\cK}{\mathcal K}
\newcommand{\cL}{\mathcal L}
\newcommand{\cN}{\mathcal N}
\newcommand{\cO}{\mathcal O}
\renewcommand{\S}{\mathsf S}
\newcommand{\cT}{\mathcal T}
\newcommand{\cU}{\mathcal U}
\newcommand{\cV}{\mathcal V}
\newcommand{\cl}{\mathcal l}
\newcommand{\cG}{\mathcal G}
% frak letters
\newcommand{\fa}{{\mathsf{a}}}
\newcommand{\fp}{{\mathfrak p}\,} %frak p as index for patches
\newcommand{\frakT} {{\mathfrak T}} % set of admissible domain transformations
%boldface symbols
\newcommand{\dist}{\mathrm{dist}}
\newcommand{\bmf} {\bm f}
\newcommand{\ba} {\bm a}
\newcommand{\bi} {\bm i}
\newcommand{\blm} {\bm m}
\newcommand{\bmj} {\bm j}
\newcommand{\bme} {\bm e}
\newcommand{\bnul}{{\boldsymbol 0}}
\newcommand{\bsnu}{{\boldsymbol \nu}}
\newcommand{\bsmu}{{\boldsymbol \mu}}
\newcommand{\bsrho}{{\boldsymbol \rho}}
\newcommand{\bseta}{{\boldsymbol \eta}}
\newcommand{\bszeta}{{\boldsymbol \eta}}
%boldsymbols
\newcommand{\bsb}{{\boldsymbol b}}
\newcommand{\bsx}{{\boldsymbol x}}
\newcommand{\bsy}{{\boldsymbol y}}
\newcommand{\bsw}{{\boldsymbol w}}
\newcommand{\bsz}{{\boldsymbol z}}
\newcommand{\bmu} {\bm{\mu}}
%\newcommand{\br} {\bm r}
\newcommand{\bM} {\bm{M}}
\newcommand{\bR} {\mathsf{R}}
\newcommand{\bT} {\bm{T}}
%
\newcommand{\ce}{{\bm c\bm e}}
\newcommand{\D}{\mathrm{D}}
\newcommand{\N}{\mathrm{N}}
\renewcommand{\L}{\mathsf{L}}
\newcommand{\A}{{\mathcal A}}
\newcommand{\V}{{\mathsf V}}
\newcommand{\W}{{\mathsf W}}
\newcommand{\Kk}{{\mathsf K}}
\newcommand{\B}{{\mathcal A}_{S}}
\newcommand{\dd}{\,{\rm d}}
\newcommand{\ddx}{\dd\bm x}
\newcommand{\ds}{\dd s}
\newcommand{\bg}{\bm{g}}
\newcommand{\bff}{\bm{f}}
\newcommand{\ddd}{\mbox{\,\em{d}}}
\newcommand{\Rd}{{\mathbb R}^d}
\newcommand{\nno}{\nonumber}
\newcommand{\jl}{[\![}
\newcommand{\jr}{]\!]}
\newcommand{\jmp}[1]{\jl#1\jr}
% % dual products
\newcommand{\dual}[2]{\left\langle#1,#2\right\rangle}
\newcommand{\dup}[2]{\langle #1, #2\rangle}

\newcommand{\ujmp}[1]{\bm\jl#1\bm\jr}
\newcommand{\al}{\langle\!\langle}
%\newcommand{\ar}{\rangle\!\rangle}
\newcommand{\avg}[1]{\al#1\ar}
\newcommand{\uavg}[1]{\bm\al#1\bm\ar}
\newcommand{\wt}[1]{\widetilde{#1}}
\newcommand{\wT}{{\widetilde{\mathcal T}_h}}
\newcommand{\wh}[1]{\widehat{#1}}
\newcommand{\wS}{S^{\widetilde{\bf p}}(\Omega,\wT,\widetilde{\bf F})}
\newcommand{\St}{\wt S^{{\bf p}}(\Omega,{{\mathcal T}_h},{\bf F})}
\renewcommand{\bar}[1]{\overline#1}
\renewcommand{\div}{{\rm div}}
\newcommand{\dx}{\dd\uu x}
\newcommand{\I}{\mathbb{I}}
\newcommand{\NNN}[1]{|\!|\!|#1|\!|\!|_{H^1(\O)}}
%\newcommand{\s}{\alpha}
\newcommand{\calP}{{\mathcal P}}
\newcommand{\QT}{{\mathcal Q}_T}
\newcommand{\R}{{\mathcal R}}
\newcommand{\wR}{\widetilde\R}
\newcommand{\M}{{\mathcal M}}
%macros for mathbb symbols
\newcommand{\IC}{{\mathbb C}}
\newcommand{\IE}{{\mathbb E}}
\newcommand{\IL}{{\mathbb L}}
\newcommand{\IN}{{\mathbb N}}
\newcommand{\IM}{{\mathbb M}}
\newcommand{\IK}{{\mathbb K}}
\newcommand{\IR}{{\mathbb R}}
\newcommand{\IU}{{\mathbb U}}
\newcommand{\IT}{{\mathbb T}}
\newcommand{\IW}{{\mathbb W}}
\newcommand{\IP}{{\mathbb P}}
\newcommand{\IQ}{{\mathbb Q}}
\newcommand{\IZ}{{\mathbb Z}}
% dislaystyle 
\def\dis{\displaystyle}
\newcommand{\dsl}{\displaystyle\sum\limits}
\newcommand{\dil}{\displaystyle\int\limits}
%
%\newcommand{\matr}[1]{\bm#1}
\def\n{\noindent}
\newcommand{\hf}{h_{K,f}^\perp}
%\newcommand{\p}{\check{\bm p}}
\newcommand{\bp}{{\bm p}}
\newcommand{\bq}{{\bm q}}
\newcommand{\Vl}{V(\Ml,\bm \psi(\Ml), \uu p)}
\renewcommand{\ss}{\bm\iota}
\newcommand{\K}{\mathfrak{K}}
%norms
\newcommand{\norm}[2]{\left\lVert #1\right\rVert_{#2}}
\newcommand{\normz}[2][]{\| #2 \|_{#1}}
%
\newcommand{\cald}{\Lambda}
% domain
\newcommand{\dom}{D}
%\newcommand{\dom}{\mathrm{D}}
\newcommand{\Dy}{\dom_\bsy} 
\newcommand{\Dscat}{\dom_{\mathrm{scat}}}
%\newcommand{\Dnul}{{\dom_\bnul}} 
\newcommand{\Dnul}{{\hat{\dom}}} 
\newcommand{\Onul}{\Omega_\bnul}

% Operators
\newcommand{\OB}{\operatorname{\mathsf{B}}}
\newcommand{\OA}{\operatorname{\mathsf{A}}}
\newcommand{\OM}{\operatorname{\mathsf{M}}}
\newcommand{\OC}{\operatorname{\mathsf{C}}}
\newcommand{\OT}{\operatorname{\mathsf{T}}}
\newcommand{\OP}{\operatorname{\mathsf{P}}}
\newcommand{\OQ}{\operatorname{\mathsf{Q}}}
\newcommand{\Id}{\operatorname{\mathsf{I}}}
\providecommand{\Cm}{{\mathcal M}}
\renewcommand{\d}{\!\!\operatorname{d}}
% imaginary unit
\newcommand{\ii}{\mathrm i}
% Bilinear forms 
\newcommand{\bs}[4]{\operatorname{\mathsf{s}}_{#1}^{#2}\left(#3,#4\right)}
\newcommand{\bt}[2]{\operatorname{\mathsf{r}}_{\bsz}^{#1}\left(#2\right)}
\newcommand{\bacav}[2]{\operatorname{\mathsf{a}\cav}\left(#1,#2\right)}
\newcommand{\bbacav}[2]{\operatorname{\hat{\mathsf{a}}_T\cav}\left(#1,#2\right)}
\newcommand{\bbacavn}[2]{\operatorname{\hat{\mathsf{a}}\cav_{0}}\left(#1,#2\right)}
\newcommand{\bapc}[2]{\operatorname{\mathsf{a}\pc}\left(#1,#2\right)}
\newcommand{\bbpc}[2]{\operatorname{\mathsf{b}\pc}\left(#1,#2\right)}
\newcommand{\fpc}[1]{\operatorname{\mathsf{f}\pc}\left(#1\right)}
\newcommand{\fpcs}[1]{\operatorname{\mathsf{f}\pc}}
% \newcommand{\fde}[1]{\operatorname{\mathsf{f}\de_T}\left(#1\right)}
% \newcommand{\fdes}{\operatorname{\mathsf{f}\de_T}}
\newcommand{\fde}[1]{{\mathsf{f}\de}\left(#1\right)}
\newcommand{\fdes}{{\mathsf{f}\de_T}}
\newcommand{\fcav}[1]{\operatorname{\mathsf{f}\cav}\left(#1\right)}
\newcommand{\fcavs}{\operatorname{\mathsf{f}\cav}}
\newcommand{\bbapc}[2]{\operatorname{\breve{\mathsf{a}}\pc}\left(#1,#2\right)}
\newcommand{\bbapcn}[2]{\operatorname{\breve{\mathsf{a}}\pc_{0}}\left(#1,#2\right)}
\newcommand{\bbbpc}[2]{\operatorname{\breve{\mathsf{b}}\pc}\left(#1,#2\right)}
\newcommand{\bbbpcn}[2]{\operatorname{\breve{\mathsf{b}}\pc_{0}}\left(#1,#2\right)}
% \newcommand{\badiel}[2]{\operatorname{\mathsf{a}\de_T}\left(#1,#2\right)}
% \newcommand{\fdiel}[1]{\operatorname{\mathsf{f}\de_T}\left(#1\right)}
\newcommand{\badiel}[2]{{\mathsf{a}\de}\left(#1,#2\right)}
\newcommand{\fdiel}[1]{{\mathsf{f}\de}\left(#1\right)}
\newcommand{\apch}{\hat{\mathsf{a}}}
\newcommand{\apc}{{\mathsf{a}}}
\newcommand{\adeh}{\hat{\mathsf{a}}^{\mathrm{de}}}
\newcommand{\ade}{{\mathsf{a}}^{\mathrm{de}}}
\newcommand{\acavh}{\hat{\mathsf{a}}^{\mathrm{cav}}}
\newcommand{\acav}{{\mathsf{a}}^{\mathrm{cav}}}
\newcommand{\fpch}{\hat{\mathsf{f}}}
\newcommand{\fdeh}{\hat{\mathsf{f}}^{\mathrm{de}}}
\newcommand{\fcavh}{\hat{\mathsf{f}}^{\mathrm{cav}}}
% Right-hand sides for Fr\'echet derivatives
\newcommand{\dfpch}{\hat{\mathsf{k}}}
\newcommand{\dfdeh}{\hat{\mathsf{k}}^{\mathrm{de}}}
\newcommand{\dfcavh}{\hat{\mathsf{k}}^{\mathrm{cav}}}

% spaces
% \newcommand{\hcurlbf}[2][]{{\bf H}_{#1}(\mathrm{curl}; #2)}

\newcommand{\hcurlbf}[2][]{{{\bH}_{#1}}(\curl, #2)}
\newcommand{\hncurlbf}[2][]{{{\bH}_0^{#1}}(\curl, #2)}
\newcommand{\hncurlbfs}[1]{\bH_ S(\curl,#1)}

\newcommand{\cmspace}[3]{\mathcal{C}^{#1} \left( #2, #3 \right)}
\newcommand{\cmspaceh}[4]{\mathcal{C}^{#1,#2} \left( #3, #4 \right)}

\newcommand{\rgeo}[1]{\mathcal{C}_b^{#1}\left( (-1,1), \IR^2 \right)}

\newcommand{\rgeoh}[2]{\mathcal{C}_b^{#1,#2}\left( (-1,1), \IR^2 \right)}

\newcommand{\cgeo}[1]{\mathcal{C}^{#1}\left( (-1,1), \IC^2 \right)}

\newcommand{\cgeoh}[1]{\mathcal{C}^{#1,h}\left( (-1,1), \IC^2 \right)}

% BEM Macros
%
\newcommand{\supp}{\operatorname{supp}}
%\newcommand{\curl}{\operatorname{\bm{curl}}}

\renewcommand{\div}{\operatorname{div}}  
\renewcommand{\Re}{\operatorname{Re}}

\newcommand{\cP}{\mathcal{P}}
\newcommand{\cQ}{\mathcal{Q}}

\newcommand{\bN}[1]{\bigl\|#1\bigr\|}  
\newcommand{\ang}[1]{\left<#1\right>}  % angular brackets for duality 
\newcommand{\rst}[1]{\left.#1\right|}  % restriction
\newcommand{\abs}[1]{\left|#1\right|}

\newcommand{\bigO}{\mathcal{O}}
\newcommand{\smo}{\mathcal{o}}
\newcommand{\pc}{}%}}
\newcommand{\de}{^{\mathrm{de}}}
\newcommand{\cav}{^{\mathrm{cav}}}
\newcommand{\die}{^{\mathrm{de}}}
\newcommand{\Dir}{_{\mathrm{Dir}}}
\newcommand{\loc}{{\mathrm{loc}}}
\newcommand{\inc}{^{\mathrm{inc}}}
\newcommand{\matr}[1]{\begin{pmatrix} #1 \end{pmatrix}}  % array with parentheses
\renewcommand{\j}{{\boldsymbol \jmath}}  % dotless j in math mode
\newcommand{\ovl}{\overline}

\newcommand{\vphi}{\varphi}
\newcommand{\veps}{\varepsilon}
\newcommand{\la}{\lambda}
\newcommand{\bvphi}{\boldsymbol \varphi}
\newcommand{\bla}{\boldsymbol \lambda}
\newcommand{\bpsi}{\boldsymbol \psi}
\newcommand{\btau}{{\boldsymbol \tau}}
\newcommand{\bxi}{{\boldsymbol \xi}}
\newcommand{\bPsi}{{\boldsymbol \Psi}}
\newcommand{\bPhi}{{\boldsymbol \psi}}   
\newcommand{\bchi}{{\boldsymbol \chi}}    

\newcommand{\divG}{\operatorname{div_S}}
\newcommand{\scurl}{\operatorname{curl}}

\newcommand{\bA}{\bm{A}}
\newcommand{\bB}{\bm{B}}
\newcommand{\bC}{\bm{C}}
\newcommand{\bE}{\bm{E}}
\newcommand{\bH}{\boldsymbol{H}}
\newcommand{\VH}{\bm{H}}
\newcommand{\bI}{\bm{I}}
\newcommand{\bn}{\bm{n}}
\newcommand{\bu}{\bm{u}}
\newcommand{\bk}{\bm{k}}
\newcommand{\bc}{\bm{c}}
\newcommand{\bw}{\bm{w}}
\newcommand{\bz}{\bm{z}}
\newcommand{\bh}{\bm{h}}
\newcommand{\bv}{\bm{v}}
\newcommand{\br}{\bm{r}}
\newcommand{\bj}{\bm{j}}
\newcommand{\bx}{\bm{x}}
\newcommand{\by}{\bm{y}}
\newcommand{\bb}{\bm{b}}
\newcommand{\bX}{\boldsymbol{X}}
\newcommand{\bV}{\bm{V}}
\newcommand{\bW}{\bm{W}}
\newcommand{\bU}{\bm{U}}
\newcommand{\bL}{\boldsymbol{L}}
\newcommand{\sbV}{\boldsymbol{V}}
\newcommand{\calH}{\mathcal{H}}
\newcommand{\J}{\mathcal{J}}

\newcommand{\iinterv}{(-1,1)\times(-1,1)}

%fields
\newcommand{\einc}{\bE^{\mathrm{inc}}}
\newcommand{\escat}{\bE^{\mathrm{scat}}}
\newcommand{\eref}{\bE^{\mathrm{ref}}}
% \newcommand{\etot}{\bU^{\mathrm{tot}}}
\newcommand{\etot}{\bE}

\newcommand{\hinc}{\VH^{\mathrm{inc}}}
\newcommand{\hscat}{\VH^{\mathrm{scat}}}
\newcommand{\href}{\VH^{\mathrm{ref}}}
%\newcommand{\htot}{\VH^{\mathrm{tot}}}
\newcommand{\htot}{\VH}

\newcommand{\half}{\frac{1}{2}}

%traces
\newcommand{\tD}{\gamma_{\mathrm{D}}}
\newcommand{\tN}{\gamma_{\mathrm{N}}}
\newcommand{\jD}{[\tD]}
\newcommand{\jN}{[\tN]}
\newcommand{\mD}{\{ \tD \}}
\newcommand{\mN}{\{ \tN \}}
%\newcounter{cont}
\title{Shape Holomorphy of Integral Operators on two-dimensional Open Arcs}

\DeclareMathOperator{\spn}{span}

%----------Author 1
\author{Jos\'e Pinto}



\usepackage{cleveref}

\begin{document}
\maketitle

\begin{abstract}
todo
\end{abstract}

\section{Introduction}
This works is concerned on establish the smoothness of the solution map from the geometry configuration composed of multiple disjoint open arcs, to the solutions of the Dirichlet and Neumann problems for an underlaying partial differential operators with boundary conditions on the mentioned open arcs. 

We do this by following closely \cite{Henriquez2021}. In the cited work, the authors extend the solution map to complex geometries (through the identification of the geometries with parametrizations), and show that the corresponding map is holomorphic in the context of the complex-variable\footnote{We refer to holomorphic in the complex variable context to the existence of the Frechet derivative on a open subset}. Hence, by means of classical complex variable theory we conclude that at least locally the solution map is analytic, i.e. derivatives of arbitrary order exist and the Taylor series converge uniformly. 

Our focus is in PDEs defined on domains of the form $\IR^2 \setminus \Gamma$, with boundary conditions on $\Gamma = \left\lbrace \Gamma_1, \hdots, \Gamma_M \right\rbrace$, as set of $M$ disjoint open arcs. In this context boundary integral formulations gives a more  natural framework to define the problem than classical weak formulations, since Green formulas do not follows directly. We refer to (\cite{stephane,stephan1984augmented,STE86,Stephan1987,sloan1991,JEREZHANCKES2011547,JHP20,Averseng2019,kress1996,kress2000}) for some references on problems posed on open arcs and the equivalent three-dimensional counterpart on surfaces with boundaries.  

\subsection{Boundary Value Problems on Open Arcs}
\subsubsection{Boundary Integral Formulation}
\label{sec:bif}
As is standard we assume that there is a fundamental solution associated with the differential operator $\cP$, denoted by $G(\vx,\by)$. We refer to \todo{ref mclean chapter 6} and references therein, for a rigorous definition of the fundamental solution, as well as existence results. We will further assume that the fundamental solution has the following structure:
\begin{align}
\label{eq:funsolgen}
G(\bx,\by )  =
\log \|\bx -\by \| F_1(\|\bx -\by \| ^2)+F_2(\|\bx -\by \| ^2),
\end{align}
where $F_1$, and $F_2$ are assumed to be entire complex functions, that are either scalars or $2 \times 2$ matrices. Furthermore we will also assume that in the scalar case $F_1(0) \neq 0$, and for the vectorial case $F_1(0)$ is invertible. The associated layered potentials, for a Jordan $\mathcal{C}^1$ arc $\Gamma$, are defined as 
\begin{align*}
SL_\Gamma \lambda(\bx) = \int_\Gamma G(\bx,\by) \lambda(\by) d\by, \quad DL_\Gamma \mu(\bx) = \int_\Gamma \left(\mathcal{B}_{\bn,\by} G(\bx,\by) \right)^T \mu(\by) d\by,
\end{align*} 
where $\mathcal{B}_{\bn,\by}$\footnote{the definition of the Double layer is more complicated if more general operators $\cP$ are allowed. We refer to \todo{ref mclean} for the details.} denotes the co-normal trace in the $\by$ variable, according to a predefined orientation of $\Gamma$, and $\lambda, \mu$ are densities defined in $\Gamma$ that could be either scalars or vectors\footnote{We forgo the boldface convection for vectors for the densities of the integral operators as we wish to treat the scalar and vector case together.}. By the definition of the fundamental solution is immediate that both potentials solves the homogeneous PDE (associated with $\cP$) in $\IR^2 \setminus \Gamma$. Furthermore we will assume that both potentials also exhibits the behavior at infinity described by the radiation condition \todo{ref}. 

We will consider the pulled back versions of the potentials, let us consider the transformed densities 
\begin{align*}
\widehat{\lambda} (s) = \lambda \circ \br(s)  \| \br'(s) \|, \quad 
\widehat{\mu} (s) = \mu \circ \br(s),
\end{align*}
and the pulled back potentials 
\begin{align*}
\widehat{SL}_{\Gamma} u := \int_{-1}^1 
 G(\bx,\br(s))  u(s) ds, \quad 
 \widehat{DL}_{\Gamma} u := \int_{-1}^1 
 \left(\mathcal{B}_{\bn,\by}G(\bx,\br(s))\right)^T  u(s) \|\br'(s)\|ds,
\end{align*}
defined for an arbitrary function $u$, thus we have $SL_\Gamma \lambda  = \widehat{SL}_\Gamma \widehat{\lambda}$, and $DL_\Gamma \mu  = \widehat{DL}_\Gamma \widehat{\mu}$. In what follows we will omit the hat notations, as we will only use the pulled back potentials. 

Now we reformulate Problem \todo{ref}, as a set of boundary integral equations. We do so, by imposing the boundary conditions on the indirect representation obtained throug layered potentials. In particular working on the geometry setting of Problem \todo{ref} we seek densities $\bla = (\lambda_1,\hdots,\lambda_M)$, with each $\lambda_i$ defined over $(-1,1)$, and $\bmu = (\mu_1, \hdots, \mu_M)$, with $\mu_i$ defined over $(-1,1)$, such that
\begin{align*}
\sum_{j=1}^M (SL_{\Gamma_j} \lambda_j )\circ \br_i = g\circ \br_i, \quad i = 1,\hdots,M \\
\sum_{j=1}^M (\mathcal{B}_{\bn,\bx}DL_{\Gamma_j} \mu_j )\circ \br_i = f\circ \br_i, \quad i = 1,\hdots,M,
\end{align*}
 thus $u = \sum_{j=1}^M SL_{\Gamma_j} \lambda_j $ is a solution for the Dirichlet Problem, and  $u = \sum_{j=1}^M DL_{\Gamma_j} \mu_j $ solves the Neumann Problem. We can rewrite the boundary integral equations in matrix form as 
\begin{align}
\label{eq:bios}
\mathbf{V}_{\Gamma_1,\hdots,\Gamma_M} \bla = \bg_{\Gamma_1,\hdots,\Gamma_M}, \quad \mathbf{W}_{\Gamma_1,\hdots,\Gamma_M} \bmu = \mathbf{f}_{\Gamma_1,\hdots,\Gamma_M},
\end{align}
where $(V_{\Gamma_1,\hdots,\Gamma_M})_{i,j} = (SL_{\Gamma_j}  )\circ \br_i$, $(W_{\Gamma_1,\hdots,\Gamma_M})_{i,j} = (\mathcal{B}_{\bn,\bx}DL_{\Gamma_j} )\circ \br_i$, and $(g_{\Gamma_1,\hdots,\Gamma_M})_i = g \circ \br_i$, $(f_{\Gamma_1,\hdots,\Gamma_M})_i = f \circ \br_i$. Notice that even is not direct from the notation, $\bla$ and $\bmu$ depends of the geometry $\Gamma_1,\hdots,\Gamma_M$. The weakly-singular operators can be represented as a Lebesgue integral defined as 
\begin{align*}
(V_{\Gamma_1,\hdots,\Gamma_M})_{i,j}u = \int_{-1}^1G(\br_i(t),\br_j(s)) u(s) ds,
\end{align*}
for a suitable $u$, see Section \todo{ref} for more details. On the other hand the hyper-singular operator could only be expressed as the finite-part of an integral, however in what follows we will assume a Maue's representation of the form 
\begin{align}
\label{eq:mauesrep}
(W_{\Gamma_1,\hdots,\Gamma_M})_{i,j}u =  \frac{d}{dt}\int_{-1}^1G(\br_i(t),\br_j(s)) \frac{d}{ds}u(s) ds + 
\int_{-1}^1\widetilde{G}(\br_i(t),\br_j(s))u(s) ds
\end{align} 
where $\widetilde{G}$ is a function with the same structure of the fundamental solution, i.e. as in \eqref{eq:funsolgen}, and $u$ can be exteneded by 0 to a closed curve containing $\Gamma_j$ in a suitable manner, we again refer to Section \todo{ref} for details. Such expressions for the hyper-singular operators are well known for particular cases of $\mathcal{P}$ on closed boundaries, \todo{ref nedelec maues, hisiao}, there is also a more general result for scalar operators in \todo{ref sauter}, however, to the best of our knowledge there is not known result for the general case. The extension of from closed curves to open arcs is carried by using the $0$ extension of the density. 

To finalize this sections, let us comment on the uniqueness of the solution of the boundary integral formulation. As we pointed out in \todo{ref arcs}, the uniqueness of the boundary integral formulation is equivalent to the solution of \todo{ref strong form}. For some examples of the uniqness of the strong formulation we refer to \todo{ref sthepane} for the Helmholtz problem, and \todo{ref kres} for the Elastic-wave problem.

\subsection{Shape Holomorphy}
\subsection{Outline}

\section{Main Results}
\begin{enumerate}
\item 
Holomorphic extension of V, and W
\item 
Holomorphic extension of $\lambda, \mu$
\item 
remark on linear functionals
\item 
Parametric holomrphism of $\lambda, \mu$
\remark 
linear functionals
\end{enumerate}

\section{Mathematical tools}
\begin{enumerate}
\item 
Abtract holomorphy theorems, maybe something with parametric holomorphism ???
\item 
Functional spaces
\item 
Holomorphy extension of standard functions 
\item 
Integral kernels 
\item 
General integral operators
\end{enumerate}

\section{Shape Holomirphism for Dirichlet and Neumann Problems}
\begin{enumerate}
\item 
Dirichlet problem proof
\item 
Neumann problem proof 
\item 
remarks on linear functionals.
\end{enumerate}

\section{Parametric Holomorphism for Ditichlet and Neumann Problems}
\begin{enumerate}
\item 
result for Dirichlet and Neumann problem 
\item 
remark on linear functionals
\item 
discretizations
\end{enumerate}

\section{Examples}

\section{Future Work}
\begin{enumerate}
\item 
Stokes equation, thermolastic???
\end{enumerate}

\section{Shape Holomorphy of Boundary Integral Operators}

Let us begin this section by recalling a pair of abstract theorems that constitute the main tool to establish the holomorphic dependence of a class of integral operators. These results are not new but slightly generalization of  Theorems 4.13 and 5.21 presented in \todo{ref}. Their proofs also follows verbatim as in the reference, and as so are omitted.

Consider a real Banach space $\mathcal{B}$, and its complexification denoted $\mathcal{B}^{\IC}$. Furthermore, consider a compact set $K \subset \mathcal{B}$ and its complex open extension defined as 
\begin{align}
\label{eq:openext}
K_\delta :=  \left\lbrace k \in \mathcal{B}^{\IC} : \text{ dist}(k, K) < \delta \right\rbrace,
\end{align}
for any $\delta>0$. We then consider a family of integral operators of the form
\begin{align*}
(P_k u)(t) = \int_{-1}^{1} f(t-s) p_k(t,s) u(s) ds,
\end{align*}
with any $k \in K$. We assume that $P_k$ is a linear bounded operator between two Hilbert spaces $H_1,H_2$\footnote{As is classical, we are abusing the notation by representing $P_k$ as an integral. Is it word notice that if $u$ is a continuous function the integral representation is well defined.} and that the continuous functions are dense in $H_1$. Now we present the result that ensure that the map $k \in K_\delta \mapsto P_k \in \mathcal{L}(H_1,H_2)$ has an holomorphic extension to $K_\delta$ for some $\delta>0$. 

\begin{theorem} \label{thrm:abstractholm}
Assume that 
\begin{enumerate}
\item 
The function $f$ is continuous everywhere with the sole possible  exception of 0. In a neighborhood of $0$ assume that $f$ is controlled as 
$$|f(t)| \lesssim| t|^{-\alpha}$$
for some $\alpha \in [0,1)$. 
\item 
There exist a $\delta >0$, such that the map $k \in K \mapsto p_k \in \cmspace{0}{(-1,1)\times(-1,1)}{\IC}$ can be extended to a map $k \in K_\delta \mapsto p_{k,\IC}$ and that extension is such that 
the map $k \in K_\delta \mapsto p_{k,\IC } \in \cmspace{0}{(-1,1)\times(-1,1)}{\IC}$ is holomorphic. 
\item 
For $\delta$ as before, the corresponding extension of $P_k$ defined as $(P_{k,\IC}u)(t) = \int_{-1}^{1} f(t,s) p_{k,\IC}(t,s) u(s) ds$ is uniformly bounded, i.e. 
$$ \sup_{k \in K_\delta} \| P_{k,\IC} \|_{\mathcal{L}(H_1,H_2)}< \infty.$$
If this three condition are satisfied, then the map $k \in K_\xi \mapsto P_{k, \IC} \in \mathcal{L}(H_1,H_2)$ is holomorphic for every $0< \xi<\delta$.  
\end{enumerate}
\end{theorem}

\begin{theorem}
\label{thrm:abtractinverse}
Consider $A_k \in \mathcal{L}(H_1,H_2)$ such that: 
\begin{enumerate}
\item 
For every $k \in K$, $A_k$ is invertible and $(A_k)^{-1} \in \mathcal{L}(H_2,H1)$. 
\item 
There exist a $\delta >0$ and a extension of $A_k$, denoted $A_{k,\IC}$, to $K_\delta$ such that 
$$k \in K_\delta \mapsto A_{k,\IC} \in \mathcal{L}(H_1,H_2),$$
 is holomorphic and uniformly bounded in $k$. 
\end{enumerate}
Then there exist $0<\eta<\delta$ depending of $K$, such that 
$A_{k,\IC}$ is invertible for every $k \in K_\eta$, and the map 
$$k \in K_\eta \mapsto (A_{k,\IC})^{-1} \in \mathcal{L}(H_2,H_1),$$ is holomorphic and uniformly bounded. 
\end{theorem} 
The proof of the latter is based on the holomphism of the inverse function, we refer to \todo{ref theorem 5.21} for the details.

The rest of this section provide the actual framework to invoke the previous two theorems. 
\subsection{Functional Spaces}

According to the previous section, we need two family of functional spaces. First the underlying Banach real space $\mathcal{B}$ that serves as the parameters for the holomorphism. Secondly, the Hilbert spaces $H_1,H_2$ that are the natural spaces for the associated integral operators. 

Let us begin with the parameter space. Given a non-empty open connected set $A$ of a finite dimensional euclidean space (real or complex). We consider the spaces $\cmspaceh{m}{h}{A}{B}$, where $B$ is any euclidean finite dimensional space, as the set of functions from $A$ to $B$ with derivatives up to order $m$ in $A$, each one with continuous extension to $\overline{A}$, for any $m \in \IN$ and all the $m$ derivatives are $h$ holder continuous for $h \in [0,1]$, also in $[-1,1]$. This space has Banach structure when is endowed with the classical norm 
\begin{align*}
\| f \|_{\cmspaceh{m}{h}{A}{B}} := \sum_{\bk: |\bk| \leq m } \sup_{\bx \in A}  \left\vert\left\vert\partial_{\bx}^{\bk} f(\bx) \right\vert\right\vert+ \sum_{k: |k| =m} \sup_{\bx \in A}  \frac{|\partial_{\bx}^{\bk}f(x)-\partial_{\bx}^{\bk}f(y)|}{\| \bx - \by\|^h},
\end{align*}
where we are using the classical multi-index notation \todo{ref}.  The case $h=0$ correspond to functions with $m$ continuous derivatives and the norm is reduced to 
\begin{align*}
\| f \|_{\cmspaceh{m}{0}{A}{B}} := \sum_{\bk: |\bk| \leq m } \sup_{\bx \in A}  \left\vert\left\vert\partial_{\bx}^{\bk} f(\bx) \right\vert\right\vert,
\end{align*}
On the other hand the case $h=1$ correspond to the case where the $m$-derivatives are Lipchitz continuous, \todo{thus the derivatives of order $m+1$ exist and are bounded almost everywhere}. Notice that for $m_1, m_2 \in \IN$ and $h_1, h_2 \in [0,1]$, such that $m_1 + h_1 < m_2 + h_2$, one has that $\cmspaceh{m_2}{h_2}{A}{B} \subset \cmspaceh{m_1}{h_1}{A}{B}$.

In particular we will make use of $\cmspaceh{m}{h}{(-1,1)}{\IR^2}$, and identify its elements with open arcs, Its complexification is denoted by $\cmspaceh{m}{h}{(-1,1)}{\IC^2}$. Furthermore we are going to only use those elements who define non self crossing arcs (Jordan arc) with non null tangent vector at every point. The subset of such functions will be denoted by $\rgeoh{m}{h}$, with the convention $\rgeoh{0}{h}  = \emptyset$.

Next we consider the Hilbert spaces. Throughout we will denote by $w(t) = \sqrt{1-t^2}$,and $T_n(t)$ the $n$th first kind Chebishev polynomial normalized according to $$\int_{-1}^1 T_n(t) T_l(t) w^{-1}(t) dt = \delta_{n,l},$$ and $U_n$ the $n$th seccond kind Chebishev polynomials normalized as 
$$\int_{-1}^1 U_n(t) U_l(t) w(t) dt = \delta_{n,l}.$$
 We will also denote by $e_n(\theta)$ the n-Fourier basis normalized according to the $L^2(-\pi,\pi)$ norm. 
Given a smooth periodic function $u :[-\pi,\pi] \rightarrow \IC$, its Fourier coefficients are denoted as
$$
\widetilde{u}_n = \int_{-\pi}^\pi u(\theta) e_{-n}(\theta) d\theta, 
$$
Similarly we define two families of first kind Chebishev coefficients:
\begin{align*}
u_n = \int_{-1}^{1} u(t) T_n(t) dt, \quad \text{and,} \quad  \widehat{u}_n = \int_{-1}^1 u(t) T_n w^{-1}(t)dt, 
\end{align*}
and two families of second kind Chebishev coefficients:
\begin{align*}
\ddot{u}_n = \int_{-1}^1 u(t) U_n (t) dt,\quad \text{and,}\quad \widecheck{u} _n = \int_{-1}^1 u(t) U_n w(t) dt
\end{align*}
This definitions are extended to bi-variate functions as 
$$\widetilde{u}_{n,l} = \int_{-\pi}^{\pi}\int_{-\pi}^\pi u(\theta,\phi) e_{-n}(\theta)e_{-l}(\phi) d\theta d\phi,$$
for a bi-periodic function, and similarly  for Chebyshev coefficients of  bi-variate functions on $[-1,1]$. Furthermore, the definition of the coefficients are extended to distribution by duality respect to the basis. 

We will make use of the traditional periodic Sobolev spaces defined as 
$$
H^s := \left\lbrace u : \| u\|_{H^s}^2 = \sum_{n=-\infty}^\infty (1+n^2)^s |\widetilde{u}_n|^2 < \infty \right\rbrace,
$$
for $s\in \IR$. We refer to \todo{ref} for a more rigorous definition. We also define the first kind spaces following \todo{ref} as : 
\begin{align*}
T^s := \left\lbrace u : \| u\|_{T^s}^2 = \sum_{n=0}^\infty (1+n^2)^s |{u}_n|^2 < \infty \right\rbrace, \\
W^s := \left\lbrace u : \| u\|_{W^s}^2 = \sum_{n=0}^\infty (1+n^2)^s |\widehat{u}_n|^2 < \infty \right\rbrace,
\end{align*} 
for $s \in \IR$. These two can be defined rigurously from $H^s$ by considering two periodic lifting operators defined as 
\begin{align}
\label{eq:liffings}
(Ju) (\theta) = u(\cos(\theta)) | \sin \theta|, \quad \text{and,} \quad
(\widehat{J}u)(\theta) = u (\cos(\theta)),
\end{align}
which again are extended to distribution by duality, and considering the equivalences 
\begin{align*}
u \in T^s \Leftrightarrow Ju \in H^s, \quad \text{and,} \quad u \in W^s \Leftrightarrow \widehat{J}u \in H^s.
\end{align*}
We also define the second kind spaces as in \todo{ref}, 
\begin{align*}
U^s := \left\lbrace u : \| u\|_{U^s}^2 = \sum_{n=0}^\infty (1+n^2)^s |\ddot{u}_n|^2 < \infty \right\rbrace, \\
M^s := \left\lbrace u : \| {u}\|_{M^s}^2 = \sum_{n=0}^\infty (1+n^2)^s |\widecheck{u}_n|^2 < \infty \right\rbrace,
\end{align*} 
These spaces could also referenced back to the classical Sobolev spaces, to this end we define the odd periodic lifftings, 
$$
Ku(\theta) = Ju(\theta) \text{sign}(\sin\theta), \quad \text{and,}\quad \widehat{K}u = \widehat{J}u(\theta) \text{sign}(\sin\theta),
$$
where the sign function is defined with the convention $\text{sign}(0)=0$, and we have the equivalences: 
\begin{align*}
u \in U^s \Leftrightarrow \widehat{K}u \in H^s, \quad \text{and,} \quad u \in M^s \Leftrightarrow Ku \in H^s.
\end{align*}
From the density of the Fourier basis in $H^s$ using the inverse of the lifting operators, is posible to see that there is a dense set of continuous  functions in $U^s$, such that they take $0$ value in $\pm 1$. Hence general Maue's representation formulas (as in \eqref{eq:mauesrep}) are valid for functions in $U^s$.  

An important property that will play a mayor role in the analysis of the hyper-singular operator are the mapping properties of the derivative, in particular from \todo{ref nicase? o martin} we have 
\begin{align}
\label{eq:devprop}
\frac{d}{dt} : U^s \rightarrow T^{s-1}, \quad \text{and,} \quad 
\frac{d}{dt} : W^s \rightarrow M^{s-1}.
\end{align}
We will use the spaces $T^s,U^s$ as domain of the integral operators on open arcs mapped back to $[-1,1]$, while the spaces $W^s, M^s$ will be used as range spaces.
\subsection{Holomorphic Extension of Standard Functions}
Before we proceed with the analysis of the integral operators we recall the holomorphic extension of some common functions.

The most basic functions that we will use are the square root and the logarithm function. Both of them are extended to the complex plane taking the main branch and the resulting extensions are holomorphic in $\IC \setminus (-\infty,0]$. We denote both extension with the same notation that the real functions, i.e. $\sqrt{\cdot}$, and $\log{(\cdot)}$.   

We will also make use of the extension of the distance function between two points of two arcs (including the possibility that the two arcs are the same).  For real parametrizations $\br, \bp :[-1,1] \rightarrow \IR^2$ the distance function is defined as 
$$d_{\br,\bp}(t,s) = \| \br(t) - \br(s)\|,$$
while for complex arcs $\br, \bp :[-1,1] \rightarrow \IC^2$, the corresponding extension is 
$$d_{\br,\bp}(t,s) =  \sqrt{(\br(t)-\bp(s))\cdot (\br(t)-\bp(s))},$$
where $\ba \cdot \bb  = a_1 b_1 + a_2 b_2$. Notice that we do not use the conjugate function as it will prevent any possibility of obtaining holomorphic extensions.  Now we present some basic tools that will enable us to show that the distance function is well defined for complex arcs. 
\begin{lemma}
\label{lemma:dwelldef}
Let $m \in \IN$, $m\geq1$, $0\leq h \leq 1$, and $K \subset \rgeoh{m}{h}$ a compact set, then 
\begin{enumerate}
\item 
\begin{align*}
\inf_{\br \in K } \inf_{t \in (-1,1)} \| \br'(t) \|>0  \quad \text{and,} \quad  \sup_{\br \in K} \sup_{t \in (-1,1)} \| \br'(t)\| < \infty.
\end{align*}
\item 
There exist $\delta >0 $ such that  
$$ 
\inf_{\br \in K_\delta} \inf_{t \in (-1,1)}Re (\br'(t) \cdot \br'(t)) > 0 .$$
\end{enumerate}
\end{lemma}
\begin{proof}
First part follows from the continuity of $I(\br) = \inf_{t \in (-1,1)} \| \br'(t)\|$, and $S(\br) = \sup_{t \in (-1,1)} \| \br'(t)\|$ in $\cmspaceh{m}{h}{(-1,1)}{\IR^2}$, in fact if $\| \br -\bp \|_{\cmspaceh{m}{h}{(-1,1)}{\IR^2}}< \epsilon$ we have that 
$I(\bp)  < \epsilon + I(\br)$
and symmetrically 
$
I(\br)  < \epsilon + I(\bp)
$
thus $|I(\br) - I(\bp)| < \epsilon$, and then the infimum in $K$ is achieved since $K$ is compact. The supremum case follows similarly. For the second part define $I = \inf_{\br \in K } I(\br)$, $S =\sup_{\br \in K } S(\br)$, and consider any element $\br \in K_\delta$, then there is $\bp \in K$ such that $\| \br -\bp \|_{\cmspaceh{m}{h}{(-1,1)}{\IR^2}} < \delta$, and we have 
\begin{align*}
\br' \cdot \br' = \|\br'\|^2+ 2(\br' -\bp')\cdot \bp' +(\br'-\bp')\cdot(\br'-\bp') 
\end{align*}
thus we get
\begin{align*}
Re(\br' \cdot \br') \geq I^2 - 2S\delta -\delta^2, \end{align*}
the result then follows by selecting $\delta < \sqrt{I^2+S^2}-S$.
\end{proof} 

Now we establish basic proprieties of the function $d_{\br,\br}$. 

\begin{lemma}
\label{lemma:dself}
Let $m \in \IN$, $m\geq 1$, and $\br \in  \cgeoh{m}$ with $\inf_{t \in (-1,1)}Re(\br'(t) \cdot \br'(t)) >0$, then the extension of the distance function 
\begin{align*}
d_{\br} (t,s) = \begin{cases} d_{\br,\br}(t,s) \quad t\neq s \\ 
0 \quad t=s \end{cases},
\end{align*} 
is a continuous function. Moreover, for a compact set $K \subset \rgeoh{m}{h}$ and $\delta$ as in Lemma \ref{lemma:dwelldef}, the map $\br \in K_\delta \mapsto d_{\br}^2 \in \cmspaceh{m}{h}{(-1,1)\times(-1,1)}{\IC}$ is holomporphic, and uniformly bounded.   
\end{lemma}
\begin{proof}
For $t \neq s$ we have that
\begin{align*}
d_{\br}^2(t,s) = (\br(t)-\br(s))\cdot (\br(t) -\br(s)),
\end{align*}
using the Taylor expansion of $\br$ we have 
\begin{align}
\label{eq:expdr}
d_{\br}^2(t,s) = (t-s)^2 \left(\int_{0}^1 \br'(t+\delta(s-t))d\delta \right) \cdot \left(\int_{0}^1 \br'(t+\delta(s-t))d\delta \right)
\end{align}
Using mean value theorem we obtain 
\begin{align}
\label{eq:ddbound}
Re ( d_{\br}^2(t,s)  )  \geq (t-s)^2 \inf_{t \in (-1,1)} Re(\br'(t) \cdot \br'(t)) >0.
\end{align}
thus $d_{\br}$ is well defined. The continuity is obtained directly from \eqref{eq:expdr}. For the second part let us define 
$$
D d_{\br}^2[\bv](t,s) = 2 (\br(t)-\br(s))\cdot (\bv(t) - \bv(s)),
$$
is clear that this function is linear in the $\bv$ variable, and we also have $D d_{\br}^2[\bv] \in \cmspaceh{m}{h}{\iinterv}{\IC}$ for $\br,\bv \in \cmspaceh{m}{h}{(-1,1)}{\IC^2}$. We also have 
\begin{align*}
d^2_{\br+\bv}(t,s) -d^2_{\br}(t,s) =  D d_{\br}^2[\bv](t,s) + d^2_{\bv}(t,s).
\end{align*}
Consider $\alpha, \beta \in \IN$ such that $0 \leq \alpha +\beta \leq m$, by the product rule for derivatives we have that 
\begin{align*}
\partial_s^\beta \partial_t^\alpha d^2_{\bv}(t,s) = \sum_{k=1}^{\alpha-1} \begin{pmatrix} \alpha \\ k \end{pmatrix} \partial_t^k \bv(t) \cdot \partial^{\alpha-k}_t \bv(t) - 2\partial_s^\beta \bv(s) \cdot \partial^\alpha_t \bv(t),
\end{align*} 
using this rule we can directly show that $d_{\bv}^2(t,s) \lesssim \|\bv\|^2_{\cgeoh{m}}$ thus we conclude that $D d_{\br}^2[\bv]$ is in fact the Frechet derivative of $d^2_{\br}$ (in the direction $\bv$), in $\cmspaceh{m}{h}{\iinterv}{\IC}$, and the map $\br  \mapsto d_{\br}^2 \in \cmspaceh{m}{h}{(-1,1)\times(-1,1)}{\IC}$ is holomorphic. Finally we have that given $\br \in K_\delta$, there is $\widetilde{\br} \in K$ such that $\| \br - \widetilde{\br}\|_{\cmspaceh{m}{h}{(-1,1)}{\IC^2}} < \delta$, then 
\begin{align*}
\| d^2_{\br} \|_{\cmspaceh{m}{h}{\iinterv}{\IC}} \leq \| d^2_{\br}  - d_{\widetilde{\br}}^2 \|_{\cmspaceh{m}{h}{\iinterv}{\IC}}  + \| d_{\widetilde{\br}}^2 \|_{\cmspaceh{m}{h}{\iinterv}{\IC}},
\end{align*}
the seccond term of the right hand side is uniformly bounded as $K$ is compact. For the first term we have
\begin{align*}
\begin{split}
\|d^2_{\br}  - d_{\widetilde{\br}}^2 \|_{\cmspaceh{m}{h}{\iinterv}{\IC}} \leq 
\|D d^2_{\widetilde{\br}}[\br - \widetilde{\br}]\|_{\cmspaceh{m}{h}{\iinterv}{\IC}} +\\ \| d^2_{\br -\widetilde{\br}}\|_{\cmspaceh{m}{h}{\iinterv}{\IC}},
\end{split}
\end{align*} 
 using the explicit expresion of the derivative and the product rule for derivatives we get
\begin{align*}
\|d^2_{\br}  - d_{\widetilde{\br}}^2 \|_{\cmspaceh{m}{h}{\iinterv}{\IC}} \lesssim 
\|\widetilde{\br}\|_{\cmspaceh{m}{h}{\iinterv}{\IC}} \delta  + \delta^2,
\end{align*}
where the unspecified constant do not depends of $\br$. The result then follows by the compactness of $K$.
\end{proof}

Now we focus in the distance for two different disjoint arcs. 
\begin{lemma}
\label{lemma:dcross}
Consider $K^1,K^2$ two disjoint compact subsets of $\rgeoh{m}{h}$, with  $m \in \IN$
then, there exist $\delta_1 >0 , \delta_2 >0$ such that the map $(\br, \bp) \in K^1_{\delta_1} \times K^2_{\delta_2} \mapsto d_{\br, \bp} \in \cmspaceh{m}{h}{(-1,1)\times(-1,1)}{\IC}$ is holomorphic and uniformly bounded. 
\end{lemma}
\begin{proof}
First notice that since we are considering compact sets in a metric space we have that
\begin{align*}
\inf_{(\br,\bp) \in K^1 \times K^2} \inf_{(t,s) \in (-1,1)\times(-1,1)}
 \| \br(t) - \bp(s) \| = I > 0,
\end{align*}
and from this we can see that
$$
Re(d_{\br,\bp}^2(t,s)) \geq I^2 -2 S(\delta_1 + \delta_2) - ( \delta_1 + \delta_2)^2,$$
where $S = \sup_{\br \in K^1_{\delta_1}} \sup_{t \in (-1,1)} \| \br'(t)\| +
\sup_{\br \in K^2_{\delta_2}} \sup_{t \in (-1,1)} \| \br'(t)\|$, which is finite since $K^1_{\delta_1}$, and $K^2_{\delta_2}$ are bounded. Hence by selecting $(\delta_1+\delta_2) < \sqrt{I^2+S^2}-S$ the distance is well defined and then $d_{\br, \bp } \in \cmspace{m}{(-1,1)\times (-1,1)}{\IC}$ since we are in the holomorpy domain of the square root function. We left to show the holomorpy  and unoform bound of the the map from two arcs to the distance, but by the holomorphy of the square root function this is reduced to show the properties for the square of the distance function which is done as in the previous lemma. 
% Consider 
%\begin{align*}
%D d_{\br,\bp}[\bh^1,\bh^2](t,s) = \frac{(\br(t) -\bp(s))\cdot(\bh^1(t)- \bh^2(s))}{d_{\br,\bp}((t,s)},
%\end{align*}
%we will show that this map is in fact the Frechet derivative of $d_{\br,\bp}$ in the direction $\bh^1,\bh^2$. Is clear from the definition that $D d_{\br,\bp}$ is a linear map in the $\bh^1,\bh^2$ direction, moreover since, 
%$$\inf_{(\br,\bp) \in K^1 \times K^2} \inf_{(t,s) \in (-1,1)\times(-1,1)}
%d_{\br,\bp}(t,s) \geq \sqrt{I^2 -2S (\delta_1 +\delta^2)-(\delta_1 +\delta_2)^2}>0,$$
%and using the product rule for derivatives we conclude that $$D d_{\br,\bp}\in \mathcal{L}(\cmspace{m}{(-1,1)}{\IC^2}^2,
%\cmspace{m}{(-1,1) \times (-1,1)}{\IC}).$$ To proof the approximation propriety we first notice that 
%$$d_{\br+ \bh^1, \bp+\bh^2}(t,s)^2 - d_{\br,\bp}(t,s)^2 = 2 (\br(t)-\bp(s))\cdot (\bh^1(t)- \bh^2(s)) + O(\| \bh^1(t)- \bh^2(s)\|^2),$$
%hence by the diferiantiability of the square root function we get 
%$$
%d_{\br+ \bh^1, \bp+\bh^2}(t,s) - d_{\br,\bp}(t,s) = \frac{(\br(t) -\bp(s))\cdot(\bh^1(t)- \bh^2(s))}{d_{\br,\bp}((t,s)}  + O( \| \bh^1(t)- \bh^2(s)\|^2).
%$$
%To finish the proof we need to show that for $\alpha, \beta \in \IN$ such that $0 \leq \alpha + \beta \leq m$ we have,  
%\begin{align*}
%\lim_{\bh^1 \rightarrow 0 \atop \bh^2 \rightarrow 0 }\frac{\partial^\beta_s\partial^\alpha_t \| \bh^1(t)- \bh^2(s)\|^2}{\| \bh^1 \|_{\cmspace{m}{(-1,1)}{\IC^2}} + \|\bh^2\|_{\cmspace{m}{(-1,1)}{\IC^2}}} = 0 ,
%\end{align*}
%the proof is the same as in the differentiability part of the Lemma \ref{lemma:dself}. Similarly the uniform bound follows by bounding $d^2_{\br,\bp}$ as in the previous lemma and the fact that the distance has strictly positive real part. 
\end{proof}

An additional function that is crucial for the analysis of integrals kernels is presented now. 

\begin{lemma}
\label{lemma:Qfun}
For an arc $\br$ consider the function 
$$
Q_{\br}(t,s) = \begin{cases}
\frac{d^2_{\br}(t,s)}{(t-s)^2}, \quad t\neq s \\
\br '(t) \cdot \br '(t), \quad t =s 
\end{cases}.
$$
Then for a compact set $K \subset \rgeoh{m}{h}$, and $\delta$ as in Lemma \ref{lemma:dwelldef} the maps, 
\begin{align*}
\br \in K_\delta \mapsto Q_{\br} \in \cmspaceh{m-1}{h}{(-1,1)\times(-1,1)}{\IC}, \\
\br \in K_\delta \mapsto 1/Q_{\br} \in \cmspaceh{m-1}{h}{(-1,1)\times(-1,1)}{\IC}, 
\end{align*} 
are both holomorphic, uniformly bounded,  and we also have \todo{check the proof this last part->i dont think is needed...}
\begin{align*}
\inf_{\br \in K_\delta} \inf_{(t,s) \in (-1,1)\times (-1,1)} Re(Q_{\br}(t,s)) >0 \\
\inf_{\br \in K_\delta} \inf_{(t,s) \in (-1,1)\times (-1,1)} Re(1/Q_{\br}(t,s)) >0 
\end{align*}
\end{lemma}
\begin{proof}
Using the Taylor expansion of $\br$  is immediate that  $$Q_{\br} \in \cmspaceh{m-1}{h}{(-1,1)\times(-1,1)}{\IC}.$$ For $1/Q_{\br}$ the result also follows from the Taylor expansion and Lemma \ref{lemma:dwelldef}.  Now let us define,
\begin{align*}
DQ_{\br}[\bv](t,s) = \frac{2 (\br(t)-\br(s))\cdot (\bv(t)-\bv(s))}{(t-s)^2}, \\
D1/Q_{\br}[\bv](t,s) = -\frac{DQ_{\br}[\bv](t,s)}{(Q_{\br}(t,s))^2} 
\end{align*}
with the continious extension for $t=s$. These two are linear maps in the $\bv$ variable, and again by Taylor expansion we have that for $\alpha, \beta \in \IN$, 
\begin{align}
\label{eq:ddbound}
\left\vert \left\vert \partial_s^\beta \partial_t^\alpha \frac{\br(t)-\br(s)}{t-s} \right \vert  \right \vert\leq \| \br \|_{\cmspace{\alpha+\beta+1}{(-1,1)}{\IC}}
\end{align}
%\begin{align}
%\label{eq:DQbound}
%DQ_{\br}[\bh](t,s)| \leq 2 \|\br\|_{\cmspace{1}{(-1,1)}{\IC^2}} \| \bh\|_{\cmspace{1}{(-1,1)}{\IC^2}}
%\end{align}
hence, $DQ_{\br}$, $D1/Q_{\br} \in \mathcal{L}(\cmspaceh{m}{h}{(-1,1)}{\IC^2}
  , \cmspaceh{m-1}{h}{(-1,1)\times(-1,1)}{\IC}$.
We also have that 
\begin{align*}
d_{\br +\bv}^2(t,s) = d^2_{\br}(t,s) + 2 (\br(t) -\br(s))\cdot (\bv(t)-\bv(s)) + d_{\bv}^2(t,s),
\end{align*}
thus, 
\begin{align*}
Q_{\br +\bv}(t,s) = Q_{\br}(t,s) + DQ_{\br}[v](t,s) + Q_{\bv}(t,s).
\end{align*}
By \eqref{eq:ddbound} we have that $\| Q_{\bv}\|_{\cmspaceh{m-1}{h}{\iinterv}{\IC^2}} \lesssim \| \bv\|_{\cmspaceh{m}{h}{\iinterv}{\IC^2}}^2$, so we get the differentiability of $Q_{\br}$. On the other hand from the differentiability of the function $1/z$ for $z$ away from $0$ we get, 
\begin{align*}
1/Q_{\br+\bv} = 1/Q_{\br} - (Q_{\br})^{-2} (Q_{\br+\bv}-Q_{\br}) + o(|Q_{\br+\bv}-Q_{\br}|),
\end{align*}
the result then follows from the diffentiability of $Q_{\br}$.



The uniform bound of $Q_{\br}$ follows directly from the one of $d^2_{\br}$ in Lemma \ref{lemma:dself}. On the other hand, for $1/Q_{\br}$ need the last part of this lemma, which is obtained using the Taylor expansion of $\br$ and Lemma \ref{lemma:dwelldef}.

The final part of the lemma is and immediate consequence of \eqref{eq:ddbound}, and the continuity in the $\br$ variable. 
\end{proof}

\subsection{Integral Kernels}
In this section we analyze kennel functions (in the sense of the function $p_k$ of Theorem \ref{thrm:abstractholm}) that are constructed as the composition of a smooth function and one of the functions studied in the previous section. 

The following result is a basic consequence of the composition of holomorphic functions. 

\begin{lemma}
\label{lemma:Fcircq}
Let $F :\IC \rightarrow \IC$ holomorphic in $\IC \setminus (-\infty,0]$. Consider $K^1, K^2$ compact subsets of $\cmspace{m}{(-1,1)}{\IR^2}$, for some $m \in \IN$, with their corresponding complex extensions $K^1_{\delta_1}$, $K^2_{\delta_2}$, for a pair $\delta_1 >0$, $\delta_2>0$. We also consider a function $q_{\br,\bp} :(-1,1)\times (-1,1) \rightarrow \IC$ with $(\br,\bp) \in K^1_{\delta_1} \times K^2_{\delta_2}$, such that  
\begin{enumerate}
\item 
The map $(\br,\bp)  \in K^1_{\delta_1} \times K^2_{\delta_2} \mapsto q_{\br,\bp} \in \cmspaceh{m-j}{h}{(-1,1)\times(-1,1)}{\IC}$ is holomorphic and uniformly bounded for a $j\in \IN$, such that $j\leq m$. 
\item we have the uniform bound: 
$$
\inf_{(\br,\bp) \in K_{\delta_1}^1 \times K_{\delta_2}^2} \inf_{(t,s) \in (-1,1)\times(-1,1)} Re( q_{\br,\bp}(t,s))>0.
$$
\end{enumerate}
Then 
$$(\br,\bp)  \in K^1_{\delta_1} \times K^2_{\delta_2} \mapsto  F \circ q_{\br,\bp} \in \cmspace{m-j}{h}{(-1,1)\times(-1,1)}{\IC}$$
is holomorphic and uniformly bounded.
\end{lemma} 
From the latter result we obtain two important consequences. 

\begin{corollary}
\label{cor:smoothcomp}
Let $F$ as in the previous lemma and $m \in \IN$: 
\begin{enumerate}
\item 
For a compact $K \subset \rgeoh{m}{h}$, with $m \geq 1$, , and $\delta$ as in Lemma \ref{lemma:dwelldef} we have that 
$$\br \in K_\delta \mapsto F \circ Q_{\br} \in \cmspaceh{m-1}{h}{(-1,1)\times(-1,1)}{\IC}$$
is holomorphic and uniformly bounded. 
\item 
For compact disjoint sets $K_1,K_2 \subset  \rgeoh{m}{h}$ and $\delta_1, \delta_2$ as in Lemma \ref{lemma:dcross} we have have that 
$$(\br,\bp) \in K^1_{\delta_1} \times K^2_{\delta_2} \mapsto F \circ d_{\br,\bp} \in \cmspaceh{m}{h}{(-1,1)\times(-1,1)}{\IC}$$
is holomorphic and uniformly bounded.
\end{enumerate}
\end{corollary}
\begin{proof}
The proof is direct from previous Lemma, using Lemmas \ref{lemma:Qfun}, and \ref{lemma:dcross} to verify the hipotesis. The only point that was not not explicitly given is that the real parts of $d_{\br,\bp}$ is strictly positive, but this is a condition for the function to be well defined and was also showed in the proof of Lemma \ref{lemma:dcross}.
\end{proof}
Finally we need to analyze the case of smooth functions acting on the function $d_{\br}$ (defined as in Lemma \ref{lemma:dself}). The result is not direct as the distance in this case is only continuous regardless of the regularity of the arcs. To establish the regularity we need special conditions on the kernel function. 
\begin{lemma}
\label{lemma:selfkernell}
Consider $F :\IC \rightarrow \IC$ holomorphic, and assume there is $f : \IC \rightarrow \IC$ also holomorphic such that
$$F(z) = f(z^2).$$ 
We consider again a compact set $K \subset \rgeoh{m}{h}$, and $\delta$ as in Lemma \ref{lemma:dwelldef}. Then we have 
$$\br \in K_\delta \mapsto F\circ d_{\br} \in \cmspaceh{m}{h}{(-1,1)\times(-1,1)}{\IC}$$
is holomorhic and uniformly bounded. 
\end{lemma}
\begin{proof}
The results is direct from the equivalence: $$F\circ d_{\br} = f( (\br(t)-\br(s)) \cdot (\br(t)-\br(s)))$$ the smoothness of $f$, and Lemma \ref{lemma:dself}. 
\end{proof}

\subsection{Abstract Integral Operators}

In this section we study the mapping properties of three kind of integral operators: 
\begin{align*}
(R_f u)(t) &:= \int_{-1}^1f(t,s) u(s) ds,\\
(L_fu)(t) &:= \int_{-1}^1 \log|t-s| f(t,s) u(s) ds\\
(S_f u)(t) &:= \int_{-1}^1 \log|t-s| (t-s)^2 f(t,s)u(s)ds,
\end{align*}
where $f \in \cmspaceh{m}{h}{(-1,1)\times(-1,1)}{\IC}$, for some $m \in \IN$. The results of this section are aiming to prove the hypotesis of the third point of Theorem \ref{thrm:abstractholm}, and are also needed for Theorem \ref{thrm:abtractinverse}.

The analysis is done following \todo{saranen chapters 6 and 11}. In particular we consider periodizations of the three types of integral operators, and then apply \todo{saranen Theorem 6.1.1} to obtain the mapping properties of the operators. 

The main differences with \todo{chapter 6 saranen} is that we consider kernels of limmited regularity, thus in the following lemma we consider describe how the limited regularity affect the range in which the integral operators could be considered.  

\begin{lemma}
\label{lemma:cmdecay2}
Let $m \in \IN$, $h \in [0,1]$, and consider a bi-periodic function $g \in \cmspaceh{m}{h}{[-\pi,\pi]\times[-\pi,\pi]}{\IC}$, then given $s_1, s_2$, non negative real numbers, such that $s_1+s_2 < m+h$, then we have 
\begin{align*}
\sum_{n=-\infty}^{\infty}\sum_{l=-\infty}^{\infty}
(1+n^2)^{s_1} (1+l^2)^{s_2} |\widetilde{g}_{n,l}|^2 \lesssim \|g\|^2_{\cmspaceh{m}{h}{[-\pi,\pi]\times[-\pi,\pi]}{\IC}},
\end{align*}
with unspecified constants independent of $g$. 
\end{lemma}
\begin{proof}
The idea of the proof is to notice that the left hand side correspond to a norm in a sobolev-type space, thus it is equivalent to the $L^2$ norm of the corresponding weak derivatives, and these can be bounded directly using the definition of the $\cmspaceh{m}{h}{[-\pi,\pi]\times[-\pi,\pi]}{\IC}$ spaces. \todo{For completness the proof is included, also I dint not found the result for bi-variate functions...}

For a bi-periodic function $g$, we define 
$$\|g\|_{\mathbf{m},\mathbf{\gamma}}^2 = \sum_{\alpha_1 \leq m_1} \sum_{\alpha_2 \leq m_2} \| \partial_t^{\alpha_1} \partial_s^{\alpha_2} g(t,s) \|^2_{L^2([-\pi,\pi]\times [-\pi,\pi])}+  |\partial_t^{m_1} \partial_{s}^{m_2} g(t,s)|_{\mathbf{\gamma}}^2  ,$$
where the $\mathbf{\gamma}-$semi-norm is defined as 
\begin{align*}
|g|_{\mathbf{\gamma}}^2 = \int_{-\pi}^{\pi}  \int_{-\pi}^{\pi} \int_{-\pi}^{\pi}\int_{-\pi}^{\pi} \frac{| g(x,y)-g(t,y)+g(t,s)-g(x,s)|^2}
{\left(\sin\left(\frac{|x-t|}{2}\right)\right)^{1+2\gamma_1}\left(\sin\left(\frac{|y-s|}{2}\right)\right)^{1+2\gamma_2}}dx dy dt ds.
\end{align*}
We will denote the Sobolev-type norm as,
\begin{align}
\label{eq:sobnormtype}
\|g\|_{s_1,s_2}^2 = \sum_{n=-\infty}^\infty \sum_{l=-\infty}^\infty (1+n^2)^{s_1}(1+l^2)^{s_2}| \widetilde{g}_{n,l}|^2
\end{align}
Now we show how $\|g\|_{s_1,s_2}^2$ can be bounded by $ \|g\|_{\mathbf{m},\mathbf{\gamma}}^2$. We notice that the Fourier coefficients of $\partial_t^{m_1} \partial_s^{m_2} g$ are in fact $(i n)^{m_1} (i l)^{m_2} \widetilde{g}_{n,l}$, this plus the inequality $(1+n^2)^s \lesssim (n^2)^s+1$ give us 
\begin{align}
\label{eq:intdevs}
\begin{split}
\|g\|_{s_1,s_2}^2 \lesssim \sum_{n = -\infty}^\infty \sum_{l = -\infty}^\infty (1+n)^{\{s_1\}} (1+l^2)^{\{s_2\}} \left\vert\left(\widetilde{\partial_t^{[s_1]} \partial_s^{[s_2]}g(t,s)}\right)_{n,l}\right\vert^2+ \\
 \sum_{n = -\infty}^\infty \sum_{l = -\infty}^\infty (1+n)^{\{s_1\}} (1+l^2)^{\{s_2\}} \left\vert\widetilde{g}_{n,l}\right\vert^2
,
\end{split}
\end{align}
where $[s_1], [s_2]$ denote the integer parts of $s_1, s_2$ respectively and $\{s_1\}, \{s_2\}$ are the corresponding fractional parts. From the last inequality is immediate that for the case $\{s_1\}= \{s_2\} =0$ we get 
$$
\|g\|_{s_1,s_2}^2 \lesssim \|g\|^2_{([s_1],[s_2]),\mathbf{0}},
$$
in any other case we define $\varrho = \partial_t^{[s_1]} \partial_s^{[s_2]}g(t,s)$, and from \eqref{eq:intdevs} we see that we only need to show that 
\begin{align*}
 \sum_{n = -\infty}^\infty \sum_{l = -\infty}^\infty (1+n)^{\{s_1\}} (1+l^2)^{\{s_s\}} \left\vert\widetilde{\varrho}_{n,l}\right\vert^2 \lesssim \| \varrho \|^2_{\mathbf{0},(\{s_1\},\{s_2\})},
\end{align*}
the details of the this part are given in Appendix \ref{appendix:fracbivariate}. Finally  from the last observation we arrive to the bound
\begin{align}
\label{eq:gbound1}
\|g\|^2_{s_1,s_2} \lesssim \|g\|^2_{([s_1],[s_2]),(\{s_1\},\{s_1\})}. 
\end{align}
Now to finish the proof we need to bound the general norm $\|g\|_{\mathbf{m},\mathbf{\gamma}}^2$, in terms of the $\cmspaceh{m}{h}{[-\pi,\pi]\times[-\pi,\pi]}{\IC}$-norms.

First notice that is direct from the definition of the Holder norms that 
\begin{align*}
 \|g\|^2_{\mathbf{m},(0,0)} \lesssim \|g\|_{\cmspaceh{m}{h}{[-\pi,\pi]\times[-\pi,\pi]}{\IC}}. 
\end{align*}
for $m+h \geq m_1 + m_2$. Now let us consider the pure fractional case, here we have that for any $\alpha, \beta >0$ such that $\alpha + \beta = 1$
\begin{align*}
\begin{split}
|g|_{\mathbf{\gamma}}^2 \lesssim \|g\|_{\cmspaceh{0}{h}{[-\pi,\pi]\times [-\pi,\pi]}{\IC}}^2\\ \left( \int_{-\pi}^{\pi}\int_{-\pi}^{\pi} \frac{|x-t|^{2\alpha h}}{\sin\left(\frac{|x-t|}{2} \right)^{1+2\gamma_1}} dt dx\right)
\left( \int_{-\pi}^{\pi}\int_{-\pi}^{\pi} \frac{|y-s|^{2\beta h}}{\sin\left(\frac{|x-t|}{2} \right)^{1+2\gamma_2}} ds dy\right),
\end{split}
\end{align*}
thus the right hand side is finite only if $h > \gamma_1 + \gamma_2$. However this condition could not used whenever $\gamma_1 +\gamma_2 \geq 1$. For the latter we assume that $m \geq 1$ and by mean value theorem we see that 
\begin{align*}
\begin{split}
|g(x,y)-g(t,y)+g(t,s)-g(x,s)|
=  \left\vert \int_{t}^x \partial_1 g(\lambda,y) - \partial_1 g(\lambda,s)d \lambda \right\vert = \\ |x-t|
 \left\vert\partial_1 g(\xi,y)-\partial_t g(\xi,s)\right\vert \leq |x-t||y-s|^h \|g \|_{\cmspaceh{1}{h}{[-\pi,\pi]\times[-\pi,\pi]}{\IC}}
\end{split}
\end{align*}
and also by considering the variations in the second variables we obtain
$$
|g(x,y)-g(t,y)+g(t,s)-g(x,s)|\leq |y-s||x-t|^h \|g \|_{\cmspaceh{1}{h}{[-\pi,\pi]\times[-\pi,\pi]}{\IC}},
$$
thus for every $\alpha\geq 0, \beta \geq 0$ such that $\alpha + \beta = 1$ we have 
\begin{align*}
\begin{split}
|g|_{\mathbf{\gamma}}^2 \lesssim \|g \|^2_{\cmspaceh{1}{h}{[-\pi,\pi]\times[-\pi,\pi]}{\IC}} \\
\int_{-\pi}^{\pi}  \int_{-\pi}^{\pi} \int_{-\pi}^{\pi}\int_{-\pi}^{\pi} \frac{ (|x-t|^{\alpha + h \beta}  
|y-s|^{\beta + h \alpha}
)^2}
{\left(\sin\left(\frac{|x-t|}{2}\right)\right)^{1+2\gamma_1}\left(\sin\left(\frac{|y-s|}{2}\right)\right)^{1+2\gamma_2}}dx dy dt ds.
\end{split},
\end{align*}
and one can easily see that integrals in the right hand side are finite if $h > \gamma_1 + \gamma_2 -1$. 
Finally we conclude that 
\begin{align*}
\|g\|_{\mathbf{m},\mathbf{\gamma}}^2 \lesssim \|g\|^2_{\cmspaceh{m}{h}{[-\pi,\pi]\times[-\pi,\pi]}{\IC}}\begin{cases}
 \text{  for $m_1+m_2+\gamma_1 + \gamma_2 < m+h$ if $\gamma_1 + \gamma_2 < 1$},\\
 \text{ for $(m_1+m_2+1)+(\gamma_1+\gamma_2 -1)< m+h$, if $\gamma_1 + \gamma_2 \geq 1$}.
\end{cases}
\end{align*}
which is equivalent to 
$$
\|g\|_{\mathbf{m},\mathbf{\gamma}}^2 \lesssim \|g\|^2_{\cmspaceh{m}{h}{[-\pi,\pi]\times[-\pi,\pi]}{\IC}},
$$
if $m_1+m_2+\gamma_1+\gamma_2 < m+h$. 
The statment of the lemma then follows from this last bound and \eqref{eq:gbound1}.
\end{proof}
 
\begin{remark}
A similar result to Lemma \ref{lemma:cmdecay2} can be established by noticing that the smoothness of $g$ implies a degree of decay in the Fourier coefficients. In fact we can obtain that for a pair of non negative real values $s_1,s_2$ and $g \in \cmspace{m}{[-\pi,\pi]\times[-\pi,\pi]}{\IC}$ one has the same bound of the previous Lemma, under the more restrictive condition $s_1+s_2+1 < m$.
This last condition could be relaxed to the better result $s_1+s_2 < m$ if we consider functions of $\cmspace{m}{[-\pi,\pi]\times[-\pi,\pi]}{\IC}$ with derivatives of order $m+1$ integrables (bounded variation).

We also notice that for $s_1, s_2 \in \IN$ we require the slightly less restrictive condition $s_1+s_2 \leq m+h$. \todo{This fact can be combined with bounds on the ad-joint and interpolation to obtain the more classical bounds for integral operators mapping from a pair Sobolev spaces of orders $s_1,s_2$ of different sign.}
\end{remark}

\begin{remark}
\label{rem:sigmaphi}
In the analysis of the integral operators, given a kernel function $f \in \cmspaceh{m}{h}{\iinterv}{\IC}$ we will consider two periodic functions, namely 
$$
\sigma(\theta,\phi ) = f(\cos \theta, \cos \phi), \quad \text{and,} \quad \varphi(\theta,\phi) = f(\cos \theta, \cos \phi)\sin \theta \sin \phi.
$$
It is clear that both functions bi-periodic and they are in $\cmspaceh{m}{h}{[-\pi,\pi]\times[-\pi,\pi]}{\IC}$. Moreover since the trigonometric functions and they derivatives are trivially bounded we have
$$
\| \sigma \|_{\cmspaceh{m}{h}{[-\pi,\pi]\times[-\pi,\pi]}{\IC}} \cong \| \varphi \|_{\cmspaceh{m}{h}{[-\pi,\pi]\times[-\pi,\pi]}{\IC}} \cong \|f\|_{\cmspaceh{m}{h}{\iinterv}{\IC}}
$$ 
with unspecified constants independent of $f$. 
\end{remark}

\subsubsection{Operator $R_f$}
The two periodic lifting of the operator $R_f$ are 
\begin{align*}
(\widehat{J}R_f u)(\theta) = \frac{1}{2}\int_{-\pi}^{\pi} f(\cos \theta, \cos \phi) Ju(\phi) d \phi,\\
(K R_f u)(\theta) = \frac{1}{2}\int_{-\pi}^{\pi} f(\cos \theta, \cos \phi) \sin \theta \sin \phi  \widehat{K}u(\phi) d \phi.
\end{align*}
We define the periodic operators:
\begin{align*}
R_f^J u(\theta)  &= \frac{1}{2}\int_{-\pi}^{\pi} f(\cos \theta, \cos \phi) u(\phi) d \phi, \\
R_f^K u (\theta) &= \frac{1}{2}\int_{-\pi}^{\pi} f(\cos \theta, \cos \phi) \sin \theta \sin \phi  u(\phi) d \phi.
\end{align*}

The following results follows directly from \todo{Saranen theorem 6.1.1}, and Lemma \ref{lemma:cmdecay2}
\begin{lemma}
\label{lemma:rfper}
Let $m \in \IN$, $h \in [0,1]$ and $s_1, s_2 \in \IR$ such that $s_1 \leq s_2$, thus if one of the following conditions is satisfied: 
\begin{enumerate}
\item $|s_1| \leq \frac{1}{2}$, $|s_2| \leq \frac{1}{2}$ and $1 < m+h$.
\item 
$|s_1| > \frac{1}{2}$, $|s_2| \leq \frac{1}{2}$, and  $\frac{1}{2}+|s_1| < m+h$.
%\begin{align*}
%\frac{1}{2}+ |s_1| < m+h, \text{ for } \{|s_1|\} < \frac{1}{2}, \\ 
%1+ |s_1| < m+h, \text{ for } \{|s_1|\} \geq \frac{1}{2}
%\end{align*}
\item 
$|s_1| \leq \frac{1}{2}$, $|s_2|>\frac{1}{2}$, and $\frac{1}{2}+|s_2| < m+h$.
%\begin{align*}
%\frac{1}{2}+ |s_2| < m+h, \text{ for } \{|s_2|\} < \frac{1}{2}, \\ 
%1+ |s_2| < m+h, \text{ for } \{|s_2|\} \geq \frac{1}{2}
%\end{align*} 
\item 
$|s_1| > \frac{1}{2}$, $|s_2|>\frac{1}{2}$, and $|s_1|+|s_2| < m+h$.
%\begin{align*}
%|s_1|+ |s_2| < m+h, \text{ for } \{|s_2|\}+\{|s_1|\} < 1, \\ 
%[|s_1|]+[|s_2|]+1+ \min\{\{|s_2|\},\{|s_1|\}\} < m+h, \text{ for } \{|s_2|\}+\{|s_1|\} \geq 1
%\end{align*} 
\end{enumerate}
we have that $ R_f^J  \in \mathcal{L}(H^{s_1},H^{s_2})$ and  $R^K_f  \in \mathcal{L}(H^{s_1},H^{s_2})$. Futhermore they are compact operators in the indicated spaces, and we also have the bounds: 
\begin{align*}
\| R_f^J\|_{ \mathcal{L}(H^{s_1},H^{s_2})} \lesssim \|\sigma\|_{\cmspaceh{m}{h}{[-\pi,\pi]\times[-\pi,\pi]}{\IC}},\\
\| R_f^K\|_{ \mathcal{L}(H^{s_1},H^{s_2})} \lesssim \|\varphi\|_{\cmspaceh{m}{h}{[-\pi,\pi]\times[-\pi,\pi]}{\IC}},
\end{align*}
where $\sigma, \varphi$ are defined as in Remark \ref{rem:sigmaphi}.
\end{lemma}

\begin{proof}
From \todo{Saranen theorem 6.1.1} we get
\begin{align*}
\| R^J_f\|_{ \mathcal{L}(H^{s_1},H^{s_2})}^2 \lesssim \sum_{n=-\infty}^\infty \sum_{l=-\infty}^\infty  (1+n^2)^{a}(1+l^2)^{b}|\widetilde{\sigma}_{n,l}|^2,\\
\| R^K_f\|_{ \mathcal{L}(H^{s_1},H^{s_2})}^2 \lesssim \sum_{n=-\infty}^\infty \sum_{l=-\infty}^\infty  (1+n^2)^{a}(1+l^2)^{b}|\widetilde{\varphi}_{n,l}|^2,
\end{align*}
where $a = \max\{|s_2|, \nu\}$, $b = \max\{|s_1|, \nu\} $, and $\nu$ any parameter such that $\nu > \frac{1}{2}$. The continuity follows by considering the cases in which is possible to select $\nu < |s_1|$ ($\nu < |s_2|$ resp.) and using Lemma \ref{lemma:cmdecay2} to bound the sums. 
The compactness follows from the fact that in the fourth cases we can slightly increase the value of $s_2$, to $s_2 + \delta$  such that the conditions still holds thus the maps are in $\mathcal{L}(H^{s_1},H^{s_2+\delta})$, and then the result follows from the compact embedding of the spaces. 
\end{proof}

Now using the properties of the lifting operators we get the properties of the operator $R_f$. 

\begin{corollary}
\label{corollary:rfop}
Let $m \in \IN$, $h \in [0,1]$ and $s_1, s_2  \in \IR$ such that the hypothesis of Lemma \ref{lemma:rfper} are fulfilled. Then we have that 
$R_f  \in \mathcal{L}(T^{s_1},W^{s_2})$, and $R_f \in \mathcal{L}(U^{s_1},M^{s_2})$. Furthermore they are compact operators and we have the bounds:
\begin{align*}
\| R_f\|_{ \mathcal{L}(T^{s_1},W^{s_2})} \lesssim \|f\|_{\cmspaceh{m}{h}{\iinterv}{\IC}},\\
\| R_f\|_{ \mathcal{L}(U^{s_1},M^{s_2})} \lesssim \|f\|_{\cmspaceh{m}{h}{\iinterv}{\IC}}.
\end{align*}
\end{corollary}
\begin{proof}
From the lifting properties, and the periodization of $R_f$ we have 
\begin{align*}
\| R_f u\|_{T^{s_2}} = \|  J R_f u \|_{H^{s_2}} = \|R_f^J Ju \|_{H^{s_2}} \leq \| R_f^J\|_{\mathcal{L}(H^{s_1},H^{s_2})} \|J u \|_{H^{s_1}}. 
\end{align*}
We also recall that $\|J u \|_{H^{s_1}} = \|u\|_{T^{s_1}}$. By previous Lemma and Remark \ref{rem:sigmaphi} we get 
\begin{align*}
\| R_f^J\|_{\mathcal{L}(H^{s_1},H^{s_2})} \cong \|f\|_{\cmspaceh{m}{h}{\iinterv}{\IC}},
\end{align*}
so we obtain the result for the pair $T^{s_1}, W^{s_2}$, and the proof is analogous for $U^{s_1},M^{s_2}$.
\end{proof}

\subsubsection{Operator $L_f$}

As for the operator $R_f$ we consider the two lifting versions of $L_f$, 
\begin{align}
\label{eq:Lsplit}
\begin{split}
(\widehat{J}L_fu)(\theta) &= \frac{log{2}}{2} \int_{-\pi}^{\pi} f(\cos \theta , \cos \phi) Ju(\phi) d\phi \\&+ \int_{-\pi}^{\pi} f(\cos \theta, \cos \phi) \log \left\vert \sin \frac{\theta-\phi}{2} \right\vert Ju(\phi) d\phi.
\end{split}
\end{align}
and, 
\begin{align}
\label{eq:Lsplitodd}
\begin{split}
(KL_fu)(\theta) &= \frac{log{2}}{2} \int_{-\pi}^{\pi} f(\cos \theta , \cos \phi)\sin \theta \sin \phi \widehat{K}u(\phi) d\phi \\&+ \int_{-\pi}^{\pi} f(\cos \theta, \cos \phi)\sin \theta \sin \phi \log \left\vert \sin \frac{\theta-\phi}{2} \right\vert \widehat{K}u(\phi) d\phi.
\end{split}
\end{align}
We see that these two operators can be characterized as the sum of a regular operator plus a logarithmic operator. The logarithmic part give rise to an operator of order $-1$ \todo{add ref}, thus by the same arguments used for the analysis of $R_f$ we arrive to the following result  

\begin{corollary}
\label{corollary:lfop}
Let $m \in \IN$, $h \in [0,1]$ and $s \in \IR$, thus if one of the following conditions is satisfied: 
\begin{enumerate}
\item $s< \frac{-3}{2}$, and $-2s-1 < m+h$
\item 
$\frac{-3}{2} \leq s < \frac{-1}{2}$,and  $ \frac{1}{2}-s < m+h$
\item 
$\frac{-1}{2} \leq s \leq \frac{1}{2}$, and $\frac{3}{2}+s < m+h$
\item 
$s > \frac{1}{2}$, and $2s + 1 < m+h$.
\end{enumerate}
we have that $ L_f  \in \mathcal{L}(T^{s},W^{s+1})$ and  $L_f  \in \mathcal{L}(U^{s},M^{s+1})$. Furthermore we also have the bounds: 
\begin{align*}
\| L_f\|_{ \mathcal{L}(T^{s},W^{s+1})} \lesssim \|f\|_{\cmspaceh{m}{h}{\iinterv}{\IC}},\\
\| L_f\|_{ \mathcal{L}(U^{s},M^{s+1})} \lesssim \|f\|_{\cmspaceh{m}{h}{\iinterv}{\IC}}.
\end{align*}
\end{corollary}
\begin{proof}
The proof is almost the same that for $R_f$, the only difference is that the continuity of the logarithmic part is restricted to $s_2 = s_1+1$, thus we parametrize the sobolev indices as $s_1 = 1$, and $s_2 = s+1$ and express the conditions of Lemma \ref{lemma:rfper} in terms of $s$. 
\end{proof}

\subsubsection{Operator $S_f$}

Finally we consider the $S_f$ operator, whose periodic liftings are 
\begin{align*}
\begin{split}
(\widehat{J}S_fu)(\theta) = \frac{\log 2}{2}\int_{-\pi}^\pi
(cos \theta - \cos \phi)^2 f(\cos \theta, \cos \phi)Ju(\phi) d\phi +\\
4 \int_{-\pi}^\pi \log \left\vert \sin \frac{\theta-\phi}{2} \right\vert \sin\left( \frac{\theta-\phi}{2} \right)^2
\sin\left( \frac{\theta+\phi}{2} \right)^2 f(\cos \theta, \cos \phi) Ju (\phi) d \phi, 
\end{split} \\
\begin{split}
(K S_fu)(\theta) = \frac{\log 2}{2}\int_{-\pi}^\pi
(cos \theta - \cos \phi)^2 f(\cos \theta, \cos \phi)\sin \theta \sin \phi \widehat{K}u(\phi) d\phi +\\
4 \int_{-\pi}^\pi \log \left\vert \sin \frac{\theta-\phi}{2} \right\vert \sin\left( \frac{\theta-\phi}{2} \right)^2
\sin\left( \frac{\theta+\phi}{2} \right)^2 f(\cos \theta, \cos \phi) )\sin \theta \sin \phi \widehat{K}u (\phi) d \phi, 
\end{split} 
\end{align*}
while this operator is of order $-3$, we will only consider it as a compact operator of order $-1$, these can be done by analyzing the mapping proprieties from $T^s$ (resp. $U^s$) to $W^{s+1+\delta}$ (resp. $M^{s+1+\delta}$). In particular we can select $\delta$ small enough such that the same conditions of Corollary \ref{corollary:lfop} apply, hence we obtain the following result. 

\begin{corollary}
\label{corollary:sfop}
Let $m \in \IN$, $h \in [0,1]$ and $s \in \IR$, such that the same conditions of Corollary \ref{corollary:lfop} are satisfied. Then
 $ S_f  \in \mathcal{L}(T^{s},W^{s+1})$ and  $S_f  \in \mathcal{L}(U^{s},M^{s+1})$, and is a compact operator in both cases. Furthermore we have the bounds: 
\begin{align*}
\| S_f\|_{ \mathcal{L}(T^{s},W^{s+1})} \lesssim \|f\|_{\cmspaceh{m}{h}{\iinterv}{\IC}},\\
\| S_f\|_{ \mathcal{L}(U^{s},M^{s+1})} \lesssim \|f\|_{\cmspaceh{m}{h}{\iinterv}{\IC}}.
\end{align*}
\end{corollary}



%\begin{lemma}
%\label{lemma:cmdecay}
%Let $m \in \IN$, with $m\geq 1$, and $h \in[0,1)$. Consider $f \in \cmspaceh{m}{h}{\iinterv}{\IC}$, then 
%\begin{align*}
%|\widehat{f}_{n,l}| \lesssim \|f\|_{\cmspaceh{m}{h}{\iinterv}{\IC}}  \min ( n^{-m}, l^{-m}),\\
%|\widecheck{f}_{n,l}| \lesssim \|f\|_{\cmspaceh{m}{h}{\iinterv}{\IC}}  \min ( n^{-m}, l^{-m}),
%\end{align*}
%for $n>m$ and $l>m$, and the unspecified constant depending of $m$, but not of $f$ For $h=1$ we have one more order in the decay (i.e.the right most term is $ \min ( n^{-m-1}, l^{-m-1})$). Also for any $n \geq 0$, $l \geq 0$ we have the elementary bound 
%\begin{align*} |\widehat{f}_{n,l}| \leq \pi^2 \|f\|_{\cmspace{0}{\iinterv}{\IC}}, \\ 
% |\widecheck{f}_{n,l}| \leq \pi^2 \|f\|_{\cmspace{0}{\iinterv}{\IC}}.
%\end{align*}
%\end{lemma}
%\begin{proof}
%For $\widehat{f}_{n,l}$ the result is a simple extension from the uni-variate results presented in \todo{ref treffeten chapter 7}. The case of $\widecheck{f}_{n,l}$, is reduced to the previous one using integration by parts since using the derivatives of the Chebishev polynomials we have
%\begin{align*}
%\int_{-1}^1 g(t) U_n w(t) dt  = - (n+1) \int_{-1}^1 \left( \int_{-1}^t g(x) dx\right) T_{n+1} w^{-1}(t) dt,
%\end{align*}
%for any continuous function $g$, and the result can be extended to more general cases without difficulties.  
%\end{proof}

%\begin{lemma}
%\label{lemma:cmdecay2}
%Let $m \in \IN$, $h \in [0,1]$, and consider $f \in \cmspaceh{m}{h}{\iinterv}{\IC}$, then given $s_1, s_2$, non negative real numbers with integer parts denoted by $[s_1], [s_2]$ and fractionals parts $\{s_1\},\{s_2\}$ respectively, such that at least one of the following conditions are met 
%\begin{enumerate}
%\item 
% $\{s_1\}+\{s_2\}<1$, and $s_1+s_2 < m+h$.
%\item 
% $\{s_1\} +\{s_2\} \geq 1$ and $[s_1]+[s_2]+1 +\min\{\{s_1\},\{s_2\}\} < m+h$.
%\end{enumerate}
%Then we have 
%\begin{align*}
%\sum_{n=0}^{\infty}\sum_{l=0}^{\infty}
%(1+n^2)^{s_1} (1+l^2)^{s_2} |\widehat{f}_{n,l}|^2 \lesssim \|f\|^2_{\cmspaceh{m}{h}{\iinterv}{\IC}},\\
%\sum_{n=0}^{\infty}\sum_{l=0}^{\infty}
%(1+n^2)^{s_1} (1+l^2)^{s_2} |\widecheck{f}_{n,l}|^2 \lesssim \|f\|^2_{\cmspaceh{m}{h}{\iinterv}{\IC}},
%\end{align*}
%with unspecified constants independent of $f$. 
%\end{lemma}
%\begin{proof}
%The idea of the proof is to notice that the left hand side correspond to a norm in a sobolev-type space, thus it is equivalent to the $L^2$ norm of the corresponding weak derivatives, and these can be bounded directly using the definition of the $\cmspaceh{m}{h}{\iinterv}{\IC}$ spaces. \todo{For completness the proof is included, also I dint not found the result for bi-variate functions...}
%
%First let us consider  the periodic case. For a bi-periodic function $g$, we define 
%$$\|g\|_{\mathbf{m},\mathbf{\gamma}}^2 = \sum_{\alpha_1 \leq m_1} \sum_{\alpha_2 \leq m_2} \| \partial_t^{\alpha_1} \partial_s^{\alpha_2} g(t,s) \|^2_{L^2([-\pi,\pi]\times [-\pi,\pi])}+  |\partial_t^{m_1} \partial_{s}^{m_2} g(t,s)|_{\mathbf{\gamma}}^2  ,$$
%where the $\mathbf{\gamma}-$semi-norm is defined as 
%\begin{align*}
%|g|_{\mathbf{\gamma}}^2 = \int_{-\pi}^{\pi}  \int_{-\pi}^{\pi} \int_{-\pi}^{\pi}\int_{-\pi}^{\pi} \frac{| g(x,s)-g(t,s)+g(t,s)-g(x,s)|^2}
%{\left(\sin\left(\frac{|x-t|}{2}\right)\right)^{1+2\gamma_1}\left(\sin\left(\frac{|y-s|}{2}\right)\right)^{1+2\gamma_2}}dx dy dt ds.
%\end{align*}
%We will denote the Sobolev-type norm as,
%\begin{align}
%\label{eq:sobnormtype}
%\|g\|_{s_1,s_2}^2 = \sum_{n=-\infty}^\infty \sum_{l=-\infty}^\infty (1+n^2)^{s_1}(1+l^2)^{s_2}| \widetilde{g}_{n,l}|^2
%\end{align}
%Now we show how $\|g\|_{s_1,s_2}^2$ can be bounded by $ \|g\|_{\mathbf{m},\mathbf{\gamma}}^2$. Given $s_1, s_2$ positive, and a smooth bi-periodic function $g$ we notice that the Fourier coefficients of $\partial_t^{m_1} \partial_s^{m_2} g$ are in fact $(i n)^{m_1} (i l)^{m_2} \widetilde{g}_{n,l}$, this plus the inequality $(1+n^2)^s \lesssim (n^2)^s+1$ give us 
%\begin{align}
%\label{eq:intdevs}
%\begin{split}
%\|g\|_{s_1,s_2}^2 \lesssim \sum_{n = -\infty}^\infty \sum_{l = -\infty}^\infty (1+n)^{\{s_1\}} (1+l^2)^{\{s_2\}} \left\vert\left(\widetilde{\partial_t^{[s_1]} \partial_s^{[s_2]}g(t,s)}\right)_{n,l}\right\vert^2+ \\
% \sum_{n = -\infty}^\infty \sum_{l = -\infty}^\infty (1+n)^{\{s_1\}} (1+l^2)^{\{s_2\}} \left\vert\left(\widetilde{g}\right)_{n,l}\right\vert^2
%.
%\end{split}
%\end{align}
%From the last inequality is immediate that for the case $\{s_1\}= \{s_2\} =0$ we get 
%$$
%\|g\|_{s_1,s_2}^2 \lesssim \|g\|^2_{([s_1],[s_2]),\mathbf{0}},
%$$
%in any other case we define $\varrho = \partial_t^{[s_1]} \partial_s^{[s_2]}g(t,s)$, and from \eqref{eq:intdevs} we see that we only need to show that 
%\begin{align*}
% \sum_{n = -\infty}^\infty \sum_{l = -\infty}^\infty (1+n)^{\{s_1\}} (1+l^2)^{\{s_s\}} \left\vert\widetilde{\varrho}_{n,l}\right\vert^2 \lesssim \| \varrho \|^2_{\mathbf{0},(\{s_1\},\{s_2\})},
%\end{align*}
%the details of the this part are given in Appendix \ref{appendix:fracbivariate}. Finally  from the last observation we arrive to the bound
%\begin{align}
%\label{eq:gbound1}
%\|g\|^2_{s_1,s_2} \lesssim \|g\|^2_{([s_1],[s_2]),(\{s_1\},\{s_1\})}. 
%\end{align}
%Now to finish the proof for periodic functions we need to bound the general norm $\|g\|_{\mathbf{m},\mathbf{\gamma}}^2$, in terms of the $\cmspaceh{m}{h}{[-\pi,\pi]\times[-\pi,\pi]}{\IC}$-norms.
%
%First notice that is direct from the definition of the Holder norms that 
%\begin{align*}
% \|g\|^2_{\mathbf{m},(0,0)} \lesssim \|g\|_{\cmspaceh{m}{h}{[-\pi,\pi]\times[-\pi,\pi]}{\IC}}. 
%\end{align*}
%for every $m+h \geq m_1 + m_2$. Now let us consider the pure fractional case, here we have that for any $\alpha, \beta >0$ such that $\alpha + \beta = 1$
%\begin{align*}
%\begin{split}
%|g|_{\mathbf{\gamma}}^2 \lesssim \|g\|_{\cmspaceh{0}{h}{[-\pi,\pi]\times [-\pi,\pi]}{\IC}}^2\\ \left( \int_{-\pi}^{\pi}\int_{-\pi}^{\pi} \frac{|x-t|^{2\alpha h}}{\sin\left(\frac{|x-t|}{2} \right)^{1+2\gamma_1}} dt dx\right)
%\left( \int_{-\pi}^{\pi}\int_{-\pi}^{\pi} \frac{|y-s|^{2\beta h}}{\sin\left(\frac{|x-t|}{2} \right)^{1+2\gamma_2}} ds dy\right),
%\end{split}
%\end{align*}
%thus the right hand side is finite only if $h > \gamma_1 + \gamma_2$. However this condition could not used whenever $\gamma_1 +\gamma_2 \geq 1$. For this last case we need to use the result \eqref{eq:smnormeq} of the Appendix \ref{appendix:fracbivariate}. In particular if we assume that $\gamma_1 = \max\{\gamma_1,\gamma_2\}$ we have 
%$$
%|g|_{\mathbf{\gamma}}^2 \leq \sum_{n=-\infty}^\infty \sum_{l=-\infty}^\infty |\widetilde{g}_{n,l}|^2 n^2 l^{2\gamma_2},
%$$
%and we recall that $ |n \widetilde{g}_{n,l}| = \left\vert \left(\widetilde{\partial_t g(t,s)}\right)_{n,l}\right\vert$, thus 
%$$
%|g|_{\mathbf{\gamma}}^2 \leq \|\partial_t g(t,s)\|_{L^2([-\pi,\pi]\times[-\pi,\pi])}^2+ |\partial_t g(t,s)|^2_{(0,\gamma_2)} \lesssim \|g\|_{\cmspaceh{1}{h}{([-\pi,\pi]\times[-\pi,\pi]}{\IC}}
%$$
%for every $h > \gamma_2$ or in general $h > \min \{\gamma_1,\gamma_2\}$. Hence we conclude that 
%\begin{align*}
%\|g\|_{\mathbf{m},\mathbf{\gamma}}^2 \lesssim \|g\|^2_{\cmspaceh{m}{h}{[-\pi,\pi]\times[-\pi,\pi]}{\IC}}\begin{cases}
% \text{  for $m_1+m_2+\gamma_1 + \gamma_2 < m+h$ if $\gamma_1 + \gamma_2 < 1$},\\
% \text{ for $m_1+m_2+1+\min\{\gamma_1, \gamma_2 \}< m+h$, if $\gamma_1 + \gamma_2 \geq 1$}.
%\end{cases}
%\end{align*}
%Notice that if $\mathbf{\gamma} = (0,0)$ only the first case apply and we have the tighter bound $m_1 + m_2 \leq m+ h$. Thus from \eqref{eq:gbound1}, and the last bound we obtain
%\begin{align}
%\label{eq:gbound2}
%\|g\|_{s_1,s_2}^2 \lesssim \|g\|^2_{\cmspaceh{m}{h}{[-\pi,\pi]\times[-\pi,\pi]}{\IC}}\begin{cases}
% \text{  for $s_1+s_2 < m+h$, if $\{s_1\} + \{s_2\} < 1$},\\
% \text{ for $[s_1]+[s_2]+1+\min\{\{s_1\}, \{s_2\} \}< m+h$, if $\{s_1\} + \{s_2\} \geq 1$},
%\end{cases}
%\end{align}
%with the exception that for $\{s_1\} =\{s_2\} = 0$ the bound is reduced to $s_1 +s_2 \leq m+h$.
%
%Now we consider the case of a non periodic function $f \in \cmspaceh{m}{h}{\iinterv}{\IC}$. We define two periodic liftings of $f$ as 
%$$
%\sigma(\theta,\phi ) = f(\cos \theta, \cos \phi), \quad \text{and,} \quad \varphi(\theta,\phi) = f(\cos \theta, \cos \phi)\sin \theta \sin \phi.
%$$
%we can see that the fourier coefficients of these functions are related to the chebyshev coefficients of f, in fact we have
% \begin{align*}
%\widetilde{\sigma}_{n,l} &= \int_{-\pi}^\pi \int_{-\pi} ^\pi f(\cos \theta, \cos \phi)  e_{-l}(\phi) e_{-n}(\theta) d\phi d\theta\\
%&= \int_{-\pi}^\pi \int_{-\pi} ^\pi f(\cos \theta, \cos \phi) \cos(|l|\phi) \cos(|n| \theta) d\phi d\theta \\
%&= 4 \int_{0}^\pi \int_{0} ^\pi  f(\cos \theta, \cos \phi) \cos(|l|\phi) \cos(|n| \theta) d\phi d\theta \\
%& = 4 \int_{-1}^1 \int_{-1}^1 f(t,s) T_{|n|} w^{-1}(t) T_{|l|} w^{-1}(s) dt ds \\
%&= 4 \widehat{f}_{|n|,|l|},  
%\end{align*}
%and also 
%\begin{align*}
%\widetilde{\varphi}_{n,l} &= \int_{-\pi}^\pi \int_{-\pi} ^\pi f(\cos \theta, \cos \phi) \sin \theta \sin \phi e_{-l}(\phi) e_{-n}(\theta) d\phi d\theta\\
%&= -\int_{-\pi}^\pi \int_{-\pi} ^\pi f(\cos \theta, \cos \phi)\sin \theta \sin \phi \sin(l\phi) \sin(n \theta) d\phi d\theta \\
%&= -4 \text{sign}(n) \text{sign}(l) \int_{0}^\pi \int_{0} ^\pi  f(\cos \theta, \cos \phi) \sin \theta \sin \phi \sin(|l|\phi) \sin(|n| \theta) d\phi d\theta \\
%& = -4\text{sign}(n) \text{sign}(l) \int_{-1}^1 \int_{-1}^1 f(t,s) U_{|n|-1} w(t) U_{|l|-1} w(s) dt ds \\
%&= -4 \text{sign}(n) \text{sign}(l)\widecheck{f}_{|n|-1,|l|-1},
%\end{align*}
%with $\widetilde{\varphi}_{n,l}=0$ if $n=0$ or $l=0$. Moreover, since the cosine function and all its derivatives are trivially bounded we have 
%\begin{align*}
%\|f\|_{ \cmspaceh{m}{h}{\iinterv}{\IC}} \cong \|\sigma \|_{\cmspaceh{m}{h}{[-\pi,\pi]\times[-\pi,\pi]}{\IC}}\cong 
%\|\varphi \|_{\cmspaceh{m}{h}{[-\pi,\pi]\times[-\pi,\pi]}{\IC}},
%\end{align*} 
%then the result follows directly by applying the bound \eqref{eq:gbound2} to the functions $\sigma$ and $\varphi$.
%\end{proof}

%\begin{remark}
%A similar result to Lemma \ref{lemma:cmdecay2} can be established by noticing that the smoothness of $f$ implies a degree of decay in the Chebishev coefficients. In fact following \todo{treffeten chapter 7}, we can obtain that for a pair of non negative real values $s_1,s_2$ and $f \in \cmspace{m}{\iinterv}{\IC}$ one has the same bound of the previous Lemma, under the more restrictive condition $s_1+s_2+1 < m$.
%This last condition could be relaxed to the better result $s_1+s_2 < m$ if we consider functions of $\cmspace{m}{\iinterv}{\IC}$ with derivatives of order $m+1$ integrables, which is comparable with the previous lemma with $h=1$, more restrictive condition but the subspaces of $\cmspace{m}{\iinterv}{\IC}$  with $m+1$ integrable derivatives is larger than $\cmspaceh{m}{1}{\iinterv}{\IC}$. 
%
%We also notice that for $s_1, s_2 \in \IN$ we require the slightly less restrictive condition $s_1+s_2 \leq m+h$.
%\end{remark}

%We recall the classical definition of the positive part of a real number defined as $s^+ = \max(0,s)$, for any $s \in \IR$.
%
%Now we can establish the mapping properties of $R_f$ directly. 
%\begin{lemma}
%\label{lemma:Rfoperator}
%Let $m \in \IN$, $h \in [0,1]$, $r_1,r_2 \in \IR$, such that $m,h$ satisfy the conditions of Lemma \ref{lemma:cmdecay2} with $s_1 = (-r_1)^+$, and $s_2 = r_2^+$, then the maps
%\begin{align*}
%f \in \cmspaceh{m}{h}{\iinterv}{\IC} \mapsto R_f \in \mathcal{L}(T^{r_1}, W^{r_2}),\\
%f \in \cmspaceh{m}{h}{\iinterv}{\IC} \mapsto R_f \in \mathcal{L}(U^{r_1}, M^{r_2}),
%\end{align*}  are \todo{compact}, and we have the bounds 
%\begin{align*}
%\| R_f \|_{\mathcal{L}(T^{r_1}, W^{r_2})} \lesssim \| f\|_{\cmspaceh{m}{h}{\iinterv}{\IC}},\\
%\| R_f \|_{\mathcal{L}(U^{r_1}, M^{r_2})} \lesssim \| f\|_{\cmspaceh{m}{h}{\iinterv}{\IC}}.
%\end{align*}
%\end{lemma}
%\begin{proof}
%We will only proof for the pair $(T^{r_1},W^{r_2})$, for $(U^{r_1},M^{r_2})$ the exact same proofs works by changing the first kind Chebishev polynomials for the second kinds and the factor $w^{-1}$ for $w$. We can directly compute,
%\begin{align*}
%\|R_fu\|_{W^{r_2}}^2  &= \sum_{n=0}^\infty (1+n^2)^{r_2} \left\vert 
%\int_{-1}^1 \left( \int_{-1}^1 f(t,s) u(s) ds\right) T_n w^{-1}(t) dt\right\vert^2\\
%& = 
%\sum_{n=0}^\infty (1+n^2)^{r_2} \left\vert  \sum_{p=0}^\infty \sum_{q=0}^\infty \widehat{f}_{p,q} u_q
%\int_{-1}^1 T_p   T_n w^{-1}(t) dt \right\vert^2
%\end{align*} 
%by the orthogonality of the Chebishev polynomials,
% \begin{align*}
%\|R_fu\|_{W^{r_2}}^2  &=
%\sum_{n=0}^\infty (1+n^2)^{r_2} \left\vert   \sum_{q=0}^\infty \widehat{f}_{n,q} u_q
% \right\vert^2 
% \\
% &= 
%\sum_{n=0}^\infty (1+n^2)^{r_2} \left\vert   \sum_{q=0}^\infty (1+q^2)^{-r_1/2}\widehat{f}_{n,q} (1 +q^2)^{r_1/2}u_q
% \right\vert^2  
%\end{align*} 
%by the Cauchy-Schawrz inequality 
%\begin{align}
%\label{eq:rfbound}
%\|R_fu\|_{W^{s_2}}^2  \leq 
%\sum_{n=0}^\infty \sum_{q=0}^\infty (1+n^2)^{r_2}     (1+q^2)^{-r_1}|\widehat{f}_{n,q}|^2  \| u\|^2_{T^{r_1}},
%\end{align}
%the mapping properties of $R_f$ then follows directly from Lemma \ref{lemma:cmdecay2}, and noticing that 
%$(1+n^2)^{r_2}     (1+q^2)^{-r_1} \leq (1+n^2)^{s_2}     (1+q^2)^{s_1}$. 
%The compactness is direct as the conditions in Lemma \ref{lemma:cmdecay2} are given in terms of strict inequalities, thus we could sightly increase $r_2$ and and keep the mapping property, then the result follows from the compact embedding of the spaces. 
%\end{proof}
%
%Now we consider the operator $L_f$. This is done by reparametrizing the operator into a periodic integral operator and using standard results, see \todo{saranen capitulo 6} for the details. We consider two periodizations of $L_f$ given by the corresponding lifftings $\widehat{J}$, and $K$, the excat expresion for these are, 
%\begin{align}
%\label{eq:Lsplit}
%\begin{split}
%(\widehat{J}L_fu)(\theta) &= \frac{log{2}}{2} \int_{-\pi}^{\pi} f(\cos \theta , \cos \phi) Ju(\phi) d\phi \\&+ \int_{-\pi}^{\pi} f(\cos \theta, \cos \phi) \log \left\vert \sin \frac{\theta-\phi}{2} \right\vert Ju(\phi) d\phi.
%\end{split}
%\end{align}
%and, 
%\begin{align}
%\label{eq:Lsplitodd}
%\begin{split}
%(KL_fu)(\theta) &= \frac{log{2}}{2} \int_{-\pi}^{\pi} f(\cos \theta , \cos \phi)\sin \theta \sin \phi \widehat{K}u(\phi) d\phi \\&+ \int_{-\pi}^{\pi} f(\cos \theta, \cos \phi)\sin \theta \sin \phi \log \left\vert \sin \frac{\theta-\phi}{2} \right\vert \widehat{K}u(\phi) d\phi.
%\end{split}
%\end{align}
%\begin{lemma}
%\label{lemma:Lfoperator}
%Let $m \in \IN$, with $m\geq 2$ and $f \in \cmspace{m}{\iinterv}{\IC}$, then  for $s \in \IR$, such that $|s+1| +|s| <m$ we have
%\begin{align*}
%\|L_f\|_{\mathcal{L}(T^s,W^{s+1})} \lesssim \|f\|_{\cmspace{m}{\iinterv}{\IC}},\\
%\|L_f\|_{\mathcal{L}(U^s,M^{s+1})} \lesssim \|f\|_{\cmspace{m}{\iinterv}{\IC}}.
%\end{align*}
%
%
%\end{lemma}

%\begin{proof}
%We recall that $\log |t-s|  = \sum_{n \geq 0} d_n T_n(t) T_n(s)$, where $d_n \sim n^{-1}$, \todo{add ref}, thus we have  \todo{este cambio de suma con integral no es directo}
%\begin{align*}
%\begin{split}
%L_f u &= \sum_{n=0}^\infty d_n \int_{-1}^1 f(t,s) T_n(t) T_n(s) u(s) ds \\ &=  \sum_{n=0}^\infty \sum_{p=0}^\infty
%\sum_{q=0}^\infty d_n \widehat{f}_{p,q} \int_{-1}^1 T_nT_p(t) T_nT_q(s) u(s) ds
%\end{split}
%\end{align*}
%Since \todo{esto no se cumple par alos normalizados}$T_n T_m (t)= \frac{1}{2} (T_{n+m}(t) + T_{|n-m|}(t))$, for every pair of integers $n,m$, we get 
%\begin{align*}
%L_f u = \sum_{n=0}^\infty \sum_{p=0}^\infty
%\sum_{q=0}^\infty \frac{d_n}{2} T_nT_p(t) \widehat{f}_{p,q} ( u_{n+q}+u_{|n-q|})
%\end{align*} 
%we do the changes of variables $n+q = r$, $n-q= r$ for $n>q$ and $q-n = r$ for $n \leq q$, thus the above expression is reduced to  \todo{faltan pequeños ajustes...}
%\begin{align}
%\label{eq:Lfexp}
%L_f u =\sum_{r=0}^\infty u_r  \sum_{n=0}^\infty \sum_{p=0}^\infty
%\frac{d_n}{2} T_nT_p(t)  ( \widehat{f}_{p,|r-n|}+\widehat{f}_{p,r+n}),
%\end{align}
%to obtain the norm of $L_fu$ in $W^{s+1}$ we need to compute its Chebishev coefficients which are given by 
%\begin{align*}
%(L_f u)_l = \int_{-1}^1 L_fu T_l w^{-1}(t) dt 
%\end{align*} 
%using \eqref{eq:Lfexp}, and the expresion for the product of Chebyshev polynomials we obtain, 
%\begin{align*}
%(L_f u)_l = \sum_{r=0}^\infty u_r  \sum_{n=0}^\infty \sum_{p=0}^\infty
%\frac{d_n}{4} ( \widehat{f}_{p,|r-n|}+\widehat{f}_{p,r+n}) \int_{-1}^1 (T_{n+p}+ T_{|n-p|})T_lw^{-1}(t) dt
%\end{align*}
%we can now use the orthogonality of the Chebishev polynomials and after some variable changes in the $p$ variable we get: 
%\begin{align*}
%(L_f u)_l = \sum_{r=0}^\infty u_r  \sum_{n=0}^\infty  \frac{d_n}{4} \left( 
%\widehat{f}_{|l-n|,|r-n|}+\widehat{f}_{|l-n|,r+n}+ \widehat{f}_{l+n,|r-n|}+\widehat{f}_{l+n,r+n}
%\right) ,
%\end{align*}
%using the last expression we get 
%\begin{align*}
%&\| L_f u\|_{W^{s+1}}^2 = \sum_{l= 0}^\infty (1+l^2)^{s+1} | (L_f u)_l| ^2 = \\
%&\sum_{l= 0}^\infty (1+l^2)^{s+1} \left\vert 
%\sum_{r=0}^\infty u_r  \sum_{n=0}^\infty  \frac{d_n}{4} \left( 
%\widehat{f}_{|l-n|,|r-n|}+\widehat{f}_{|l-n|,r+n}+ \widehat{f}_{l+n,|r-n|}+\widehat{f}_{l+n,r+n}
%\right)
%\right\vert^2
%\end{align*} 
%we use the cauchy-schawrz inequality so we obtain 
%\begin{align}
%\label{eq:fourdsum}
%\begin{split}
%\| L_f u\|_{W^{s+1}}^2 \leq \| u\|_{T^s}^2  \sum_{l= 0}^\infty \sum_{r=0}^\infty (1+l^2)^{s+1} (1+r^2)^{-s} \\ \left\vert \sum_{n=0}^\infty
%\frac{d_n}{4} \left( 
%\widehat{f}_{|l-n|,|r-n|}+\widehat{f}_{|l-n|,r+n}+ \widehat{f}_{l+n,|r-n|}+\widehat{f}_{l+n,r+n}
%\right)
%\right\vert^2 
%\end{split}
%\end{align}
%We now estimate the sums on the right hand side of the last equation, since $f \in \cmspace{m}{\iinterv}{\IC}$ we can use Lemma \ref{lemma:cmdecay}, thus for the first sum we get  
%\begin{align*}
%\sum_{n=0}^\infty d_n \widehat{f}_{|l-n|,|r-n|} \lesssim \|f\|_{\cmspace{m}{\iinterv}{\IC}}
%\sum_{n=1}^\infty  n^{-1} (|l-n|+1)^{-(m+1)},
%\end{align*}
%we then split the sum in three parts \footnote{we assume that $l$ is even, the odd case is similar}
%\begin{align*}
%\sum_{n=0}^\infty d_n \widehat{f}_{|l-n|,|r-n|} \lesssim\|f\|_{\cmspace{m}{\iinterv}{\IC}} \\
%\sum_{n=1}^{l/2}  n^{-1} (l-n)^{-(m+1)}+ \sum_{n> l/2}^{3l/2} n^{-1} (|l-n|+1)^{-(m+1)} + \sum_{n >3l/2 }n^{-1} (n-l)^{-(m+1)} ,
%\end{align*}
%each of them is estimated using integrals interpolants so we obtain:  \todo{la segunda esta mal!, esta suma no decae asi...decae con orden 1 solamente creo.....}
%\begin{align*}
%\sum_{n=1}^{l/2}  n^{-1} (l-n)^{-(m+1)} \lesssim \int_{1}^{l/2} x^{-1} (l-x)^{-m-1} dx  \leq \left(\frac{l}{2}\right)^{-m-1}\log{\frac{l}{2}} \lesssim l^{-m-1}\log l \\
%\sum_{n> l/2}^{3l/2} n^{-1} (|l-n|+1)^{-(m+1)}  \lesssim \int_{l/2}^{3l/2} x^{-1}(|l-x|+1)^{-m-1} dx\leq \left(\frac{l}{2}\right)^{-m-1} \log 3  \lesssim l^{-m-1}\\
%\sum_{n >3l/2 }n^{-1} (n-l)^{-(m+1)} \lesssim \int_{3l/2}^\infty x^{-1}(x-l)^{-m-1} \leq \frac{2}{3l} \left( \frac{2}{3l} \right)^{m} \lesssim l^{-m-1},
%\end{align*}
%thus we obtain
%$$
%\sum_{n=0}^\infty d_n \widehat{f}_{|l-n|,|r-n|} \lesssim\|f\|_{\cmspace{m}{\iinterv}{\IC}} l^{-m-1} \log l
%$$ 
%similarly using Lemma \ref{lemma:cmdecay} to bound the coefficients of $f$ in terms of the $r$ variable we can find the following estimation 
%$$
%\sum_{n=0}^\infty d_n \widehat{f}_{|l-n|,|r-n|} \lesssim\|f\|_{\cmspace{m}{\iinterv}{\IC}} r^{-m-1} \log r,
%$$ 
%and combining these two bounds we have that for $\alpha, \beta \in \IR$ such that $\alpha + \beta =1 $, 
%$$
%\sum_{n=0}^\infty d_n \widehat{f}_{|l-n|,|r-n|} \lesssim\|f\|_{\cmspace{m}{\iinterv}{\IC}} r^{-\beta(m+1)} l^{-\alpha(m+1)} (\log r)^\beta (\log l)^\alpha,
%$$ 
%for other three terms in \eqref{eq:fourdsum} we can apply the same ideas so we get, 
%\begin{align*}
%\| L_f u\|_{W^{s+1}}^2  \lesssim \| u\|_{T^s}^2 \|f\|_{\cmspace{m}{\iinterv}{\IC}}^2 \sum_{l= 1}^\infty \sum_{r=1}^\infty \\(1+l^2)^{s+1} (1+r^2)^{-s} r^{-2\beta(m+1)} l^{-2\alpha(m+1)} (\log r)^{2\beta} (\log l)^{2\alpha},
%\end{align*}
%notice that the terms $r=l=0$ are omitted as they don't play any role in the convergence. Now the proof is finished by noticing that a equivalent condition for the last sum to converge is 
%$$2(s+1) -2(m+1)\alpha < -1, \quad \text{and,} \quad -2s -2(m+1)\beta < -1. $$
%we replace $\beta = 1 - \alpha$ sum both inequalities and obtain the condition $1<m$. 
%\end{proof}
%\begin{remark}
%One can conclude that in fact the conditions for the continuity of $L_f$ and $R_f$ are the same, so the inclusion of the extra weakly singular factor makes no difference in this context. The reason for this is based in that $m$ could only take discrete values so the extra logaritmic factor plays no role in the convergence of the series. 
%\end{remark}
%\todo{si dejo la demostracion anterior no necesito nada de fourier.}
%\begin{proof}
%By the definitions of the functional spaces we have that 
%\begin{align*}
%\| \widehat{J} L_f u \|_{H^{s+1}} = \|L_f u  \|_{W^{s+1}},\quad \text{and,} \quad \| KL_f u \|_{H^{s+1}} = \|L_f u \|_{M^{s+1}},
%\end{align*}
%The regular parts of $\widehat{J} L_f u$, and $K L_f u$, defined as the first term of the right hand side of \eqref{eq:Lsplit}, and \eqref{eq:Lsplitodd} respectively can be mapped back to $[-1,1]$ by taking the inverse of the corresponding lifting operator and a cosine change of variable, hence we obtain the terms 
%$$
%\left\vert \left \vert
% \int_{-1}^{1} f(t , s) u(s) ds 
% \right\vert\right\vert_{W^{s+1}}, \quad \text{and,} \quad \left\vert \left \vert
% \int_{-1}^{1} f(t , s) u(s) ds 
% \right\vert\right\vert_{M^{s+1}},
%$$
%which are operator with regular kernels. Since $m>1$ from Lemma \ref{lemma:Rfoperator} these to are bounded by the norm of $u$ in the corresponding space\footnote{$T^s$ for $W^{s+1}$, and $U^s$ for $M^{s+1}$} for $s \in \IR$, in fact these regular part are compact as proven in the previous Lemma. For the singular part, we first notice that this correspond to a traditional weakly-singular operator (order $-1$), and we  can use Theorem \todo{saranen theorem 6.1.1} so for any $\nu >\frac{1}{2}$,
%\begin{align}
%\label{eq:condS}
%\begin{split}
%\left\vert \left\vert \int_{-\pi}^{\pi} f(\cos \theta, \cos \phi) \log \left\vert \sin \frac{\theta-\phi}{2} \right\vert Ju(\phi) d\phi  \right\vert \right\vert_{H^{s+1}}^2\lesssim  \|J u\|_{H^s}^2\\
%\sum_{p=-\infty}^{\infty} \sum_{q = -\infty}^\infty (1+p^2)^{\max(|s+1|,\nu)}(1+q^2)^{\max(|s|,\nu)} | \widetilde{f}_{p,q}|^2,
%\end{split}
%\end{align}
%and also,
%\begin{align}
%\label{eq:condSodd}
%\begin{split}
%\left\vert \left\vert \int_{-\pi}^{\pi} f(\cos \theta, \cos \phi) \sin \theta \sin \phi \log \left\vert \sin \frac{\theta-\phi}{2} \right\vert \widehat{K}u(\phi) d\phi  \right\vert \right\vert_{H^{s+1}}^2\lesssim  \|\widehat{K} u\|_{H^s}^2\\
%\sum_{p=-\infty}^{\infty} \sum_{q = -\infty}^\infty (1+p^2)^{\max(|s+1|,\nu)}(1+q^2)^{\max(|s|,\nu)} | \widetilde{\widetilde{f}}_{p,q}|^2,
%\end{split}
%\end{align}
%and since $\|Ju\|_{H^s} = \|u \|_{T^s}$, and also $\|\widehat{K} u \|_{H^s} = \|u\|_{U^s}$ we only need to verify that the sums converge. The coefficients $\widetilde{f}_{p,q}$ are the Fourier coefficients of the bi-variate function $f(\cos \theta, \cos \phi)$, which are related to the Chebishev coefficients of $f$ as, 
%\begin{align*}
%\widetilde{f}_{p,q} &= \int_{-\pi}^\pi \int_{-\pi} ^\pi f(\cos \theta, \cos \phi)  e_{-q}(\phi) e_{-p}(\theta) d\phi d\theta\\
%&= \int_{-\pi}^\pi \int_{-\pi} ^\pi f(\cos \theta, \cos \phi) \cos(|q|\phi) \cos(|p| \theta) d\phi d\theta \\
%&= 4 \int_{0}^\pi \int_{0} ^\pi  f(\cos \theta, \cos \phi) \cos(|q|\phi) \cos(|p| \theta) d\phi d\theta \\
%& = 4 \int_{-1}^1 \int_{-1}^1 f(t,s) T_{|p|} w^{-1}(t) T_{|q|} w^{-1}(s) dt ds \\
%&= 4 \widehat{f}_{|p|,|q|}. 
%\end{align*}
%On the other hand, the terms $\widetilde{\widetilde{f}}_{p,q}$ correspond to the Fourier coefficients of $f(\cos \theta, \cos \phi) \sin \theta \sin \phi$, and as in the previous case we have
%\begin{align*}
%\widetilde{\widetilde{f}}_{p,q} &= \int_{-\pi}^\pi \int_{-\pi} ^\pi f(\cos \theta, \cos \phi) \sin \theta \sin \phi e_{-q}(\phi) e_{-p}(\theta) d\phi d\theta\\
%&= -\int_{-\pi}^\pi \int_{-\pi} ^\pi f(\cos \theta, \cos \phi)\sin \theta \sin \phi \sin(q\phi) \sin(p \theta) d\phi d\theta \\
%&= -4 \text{sign}(p) \text{sign}(q) \int_{0}^\pi \int_{0} ^\pi  f(\cos \theta, \cos \phi) \sin \theta \sin \phi \sin(|q|\phi) \sin(|p| \theta) d\phi d\theta \\
%& = -4\text{sign}(p) \text{sign}(q) \int_{-1}^1 \int_{-1}^1 f(t,s) U_{|p|-1} w(t) U_{|q|-1} w(s) dt ds \\
%&= -4 \text{sign}(p) \text{sign}(q)\widecheck{f}_{|p|-1,|q|-1}. 
%\end{align*}
%notice that this last expression is not valid for $p=0$ or $q=0$, however one can see from the definition of $\widetilde{\widetilde{f}}_{p,q}$ that this term is null whenever $p=0$ or $q=0$ so we can skip those terms. 
%
%By Lemma \ref{lemma:cmdecay} an equivalent condition to the convergences of the sums in \eqref{eq:condS}, and \eqref{eq:condSodd} are
%\begin{align*}
%2 \max(|s+1|, \nu ) - 2(m+1) \alpha < -1, \\ 
%2 \max(|s|, \nu ) - 2(m+1) \beta < -1 ,
%\end{align*}
%for any $\alpha, \beta$ such that $\alpha + \beta =1$. We again replace $\beta$ and sum both inequalities and obtain the condition 
%\begin{align*}
%\max(|s+1|, \nu ) +\max(|s|, \nu )  < m ,
%\end{align*}
%since we can select $\nu$ as close to $\frac{1}{2}$ as we want, and we already have that $m > 1$, we only need that 
%$$|s+1| + |s| < m. $$
%The continuity in terms of $f$ follows also from \eqref{eq:condS}, and \eqref{eq:condSodd}; and Lemma   \ref{lemma:cmdecay}. 
%\end{proof}
%
%The final part of this section is dedicated to the analysis of $S_f$. The analysis is similar to the one of $L_f$, in particular we have that the lifting versions of this operator are
%\begin{align*}
%\begin{split}
%(\widehat{J}S_fu)(\theta) = \frac{\log 2}{2}\int_{-\pi}^\pi
%(cos \theta - \cos \phi)^2 f(\cos \theta, \cos \phi)Ju(\phi) d\phi +\\
%4 \int_{-\pi}^\pi \log \left\vert \sin \frac{\theta-\phi}{2} \right\vert \sin\left( \frac{\theta-\phi}{2} \right)^2
%\sin\left( \frac{\theta+\phi}{2} \right)^2 f(\cos \theta, \cos \phi) Ju (\phi) d \phi, 
%\end{split} \\
%\begin{split}
%(K S_fu)(\theta) = \frac{\log 2}{2}\int_{-\pi}^\pi
%(cos \theta - \cos \phi)^2 f(\cos \theta, \cos \phi)\sin \theta \sin \phi \widehat{K}u(\phi) d\phi +\\
%4 \int_{-\pi}^\pi \log \left\vert \sin \frac{\theta-\phi}{2} \right\vert \sin\left( \frac{\theta-\phi}{2} \right)^2
%\sin\left( \frac{\theta+\phi}{2} \right)^2 f(\cos \theta, \cos \phi) )\sin \theta \sin \phi \widehat{K}u (\phi) d \phi, 
%\end{split} \\
%\end{align*}
%
%One can see that these are classical periodic operators of order $-3$ \todo{ref saranen capitulo 5..}, so in particular they can be seen as operator of order $-1-\epsilon$ for small $\epsilon >0$. Based on this observation and the compact embeding of the functional spaces, as it was noticed in the proof of Lemma \ref{lemma:Rfoperator}, we obtain the following result. 
%
%\begin{corollary}
%\label{corollary:Soperator}
%Let $m \in \IN$, with $m\geq 2 $, and $f \in \cmspace{m}{\iinterv}{\IC}$, then for $s \in \IR$ such that $|s+1|+ |s| < m$, we have that $S_f \in \mathcal{L}(T^s,W^{s+1})$ and  also, $\S_f \in \mathcal{L}(U^s,M^{s+1})$. Futhermore in both cases $S_f$ is a compact operator. 
%\end{corollary}

\subsection{Weakly-Singular and Hyper-Singular Operators Holomorphic Extensions}
In this section we will show that the operators $\mathbf{V}_{\Gamma_1,\hdots,\Gamma_M}, \mathbf{W}_{\Gamma_1,\hdots,\Gamma_M}$ defined as in \eqref{eq:bios} have holomorphic extension in the parametrizations $\br_1,\hdots,\br_M$. We will first analyze the weakly-singular operator and then use the Maue's representation for obtain the result for the hyper-singular operator. 

Throughout the rest of this section we consider $m\in \IN$, $h \in [0,1]$, and, $M$ compact disjoint sets $K^1,\hdots,K^M \subset \rgeoh{m}{h}$. and the corresponding positive real numbers $\delta_1, ..,\delta_M$ chosen according to Lemmas \ref{lemma:dwelldef}, and \ref{lemma:dcross}, such that for any pair $j,k \in 1,\hdots,M$ (with $j\neq k$) the conclusion of both lemmas holds for the set $K^j_{\delta_j}$ and $ K^j_{\delta_j}\times K^k_{\delta_k}$, respectively \footnote{this selection of values for the $\delta$ variables is possible by taking the minimum value, such that both lemmas holds for all the combination of $j,k$. }. 

We will also consider the spaces $\mathbb{T}^s, \mathbb{W}^s$ which could be either equal to $T^s,W^s$ if the associated operator $\cP$ is scalar, or $T^s\times T^s,W^s\times W^s$ in the vector case. Similarly we define $\mathbb{U}^s, \mathbb{M}^s$ as $U^s,M^s$ for scalar $\cP$ and $U^s\times U^s,M^s\times M^s$ for the vector case.

Finally notice that since the weakly-singular and hyper-singular operators are defined acting on a cartesian product of densities we will have to make use of the product spaces $\prod_{j=1}^M \IT^s,$ $\prod_{j=1}^M \IU^s,$ $\prod_{j=1}^M \IW^s,$ and $\prod_{j=1}^M \IM^s$, and we take the standard euclidean-type norm, i.e.
$$
\| \bu \|_{\prod_{j=1}^M \IT^s}^2 = \sum_{j=1}^M \| u_j \|_{\IT^s}^2,
$$ 
and analogously for the other 3 spaces. 																																																																			   
 
\subsubsection{Weakly-singular Operator}

Let us star with the analysis of the different components of the operator 
$\mathbf{V}_{\Gamma_1,\hdots,\Gamma_M}$. In a more general setting let us consider 
$$V_{\br,\bp}u  = \int_{-1}^1G(\br(t),\bp(s)) u(s) ds,$$
thus $V_{\br_i, \br_j} = (V_{\Gamma_1,\hdots,\Gamma_M})_{i,j}$. As consequence of the structure of the fundamental solution and the results of holomorphic extension of integral operators we arrive to the following result. 

\begin{lemma}
\label{lemma:vcomp}For $s\in \IR$ such that:
\begin{enumerate}
\item[i] 
If $s < \frac{-3}{2}$, then $-2s-1<m-1+h$,
\item[ii] 
If $\frac{-3}{2} \leq s < \frac{-1}{2}$, then $\frac{1}{2} -s < m-1+h$.
\item[iii] 
If $\frac{-1}{2} \leq s \leq \frac{1}{2}$, then $\frac{3}{2}+s< m-1 +h$
\item[iv] 
If $s> \frac{1}{2}$, then $2s +1 < m -1 +h $.
\end{enumerate}
We have:
\begin{enumerate}
\item  Let $\xi_j< \delta_j$, for $j=1,\hdots,M$, then the maps
\begin{align*}
\br \in K^j_{\xi_j} \mapsto V_{\br,\br} \in 
\mathcal{L}(\IT^s, \IW^{s+1}), \\
\br \in K^j_{\xi_j} \mapsto V_{\br,\br} \in 
\mathcal{L}(\IU^s, \IM^{s+1}),
\end{align*} 
are holomorphic, and uniformly bounded. Moreover, when we select the pair $\IT^s,\IW^{s+1}$ the operator is Freedholm of order 0. 
\item 
For $j,k \in \{1,\hdots,M\}$, $j \neq k$,  and let $\xi_j < \delta_j$, and $\xi_k  < \delta_k$, then the maps 
\begin{align*}
 (\br,\bp) \in K^j_{\delta_j} \times K^k_{\delta_k} \mapsto V_{\br,\bp} \in 
\mathcal{L}(\IT^s, \IW^{s+1}),\\
 (\br,\bp) \in K^j_{\delta_j} \times K^k_{\delta_k} \mapsto V_{\br,\bp} \in 
\mathcal{L}(\IU^s, \IM^{s+1}),
\end{align*}
are holomorphic, uniformly bounded and compact. 
\end{enumerate}
\end{lemma}
\begin{proof} 
We are going to focus in the case of scalar operator $\cP$ with the spaces $T^s,W^{s+1}$, as the vector case and spaces $U^s,M^{s+1}$ follows verbatim. We recall the assumptions on the structure of the fundamental solution \eqref{eq:funsolgen}, thus the self interaction can be expressed as 
\begin{align*}
G(\br(t),\br(s)) = \log|t-s| F_1(d_{\br}(t,s)^2)+ \log Q_{\br}(t,s)F_1(d_{\br}(t,s)^2) + F_2(d_{\br}(t,s)^2),
\end{align*}
we define the singular and regular parts as: 
\begin{align*}
G^R(\br(t),\br(s)) &:= \log Q_{\br}(t,s)F_1(d_{\br}(t,s)^2) + F_2(d_{\br}(t,s)^2)\\
G^S(\br(t),\br(s)) &:= \log|t-s| F_1(d_{\br}(t,s)^2).
\end{align*}
Let us first consider the integral kernel $G^R(\br(t),\br(s))$ and the associated integral operator. From the smoothness of $F_1$, $F_2$ combined with Lemma \ref{lemma:selfkernell}, and also Lemma \ref{lemma:Qfun}, we obtain that the map $\br \in K^j_{\delta_j} \mapsto G^R(\br(t),\br(s)) \in \cmspaceh{m-1}{h}{\iinterv}{\IC}$ is holomorphic and uniformly bounded. Furthermore by Corollary \ref{corollary:rfop} and the conditions on $s,m,h$, the associated integral operator is in $\mathcal{L}(T^s,W^{s+1})$, and its norm is controlled by the norm of $G^R$ in $\cmspaceh{m-1}{h}{\iinterv}{\IC}$, thus the operator is uniformly bounded as a map from $K^j_{\delta_j}$ to $\mathcal{L}(T^s,W^{s+1})$, and by the same Lemma is in fact a compact operator.

Now let us consider the singular part. In this case we define $f =F_1(d_{\br}(t,s)^2)$, and from the smoothness of $F_1$ and Lemma \ref{lemma:selfkernell} we have that the map $\br \in K^j_{\delta_j} \mapsto f \in \cmspaceh{m}{h}{\iinterv}{\IC}$ is holomorphic and uniformly bounded. We then can consider the integral operator associated with $G^S$ as an operator of the form $L_f$ of the previous section. Hence, from Corollary \ref{corollary:lfop}, and the conditions on $s,m,h$, we have that the associated integral operator is in $\mathcal{L}(T^s,W^{s+1})$, and its norm is controlled by the norm of $f$ in $\cmspaceh{m}{h}{\iinterv}{\IC}$, thus the integral operator is uniformly bounded as before. Furthermore we see that the integral operator can be written as
\begin{align}
\label{eq:fredSing}
F_1(d_{\br}(t,t)^2)\int_{-1}^1 \log|t-s| u(s) ds + \int_{-1}^1 \log|t-s| (F_1(d_{\br}(t,s)^2)-F_1(d_{\br}(t,t)^2))u(s) ds
\end{align}
By the assumption of the invertibility of $F_1(d_{\br}(t,t)^2)= F_1(0)$  the first part is the clasical logarithmic operator which is an invertible operator in $T^s$ \todo{ref nedelec}\footnote{In fact the logarithmic operator is diagonal, see \todo{ref}.} \todo{check invertibility in $U^s$, is an infinite matrix... using the periodic operators is hard to argue ...}. On the other hand the difference in the second part can be expanded by Taylor and we get 
$$
(F_1(d_{\br}(t,s)^2)-F_1(d_{\br}(t,t)^2)) = d_{\br}(t,s)^2\int_{0}^1 F_1(\delta d_{\br}(t,s))d\delta,
$$
we cam then use the Taylor expansion of parametrization $\br$ (see \eqref{eq:expdr}) so we get,
\begin{align*}
\begin{split}
(F_1(d_{\br}(t,s)^2)-F_1(d_{\br}(t,t)^2)) &= (t-s)^2 \\ \left(\int_{0}^1 \br'(t+\delta(s-t))d\delta \right) &\cdot \left(\int_{0}^1 \br'(t+\delta(s-t))d\delta \right)\int_{0}^1 F_1(\delta d_{\br}(t,s))d\delta,
\end{split}
\end{align*}
from the last expression is clear that the second term in the right hand side of \eqref{eq:fredSing} is an operator of the form of $S_f$ descrived in the previous section, with $f \in \cmspaceh{m-1}{h}{\iinterv}{\IC}$, thus by Corollary \ref{corollary:sfop} and the conditions on $s,m,h$ we can conclude that this part is a compact operator, finally we obtain that the operator with kernel $G^s$ is in fact a Freedholm operator of order $0$.

If we combine the results for the two parts of the kernel function we see that first, the map $\br \in K^j_{\delta_j} \mapsto V_{\br,\br} \in \mathcal{L}(T^s,W^{s+1})$ is uniformly bounded. Secondly, using the last point and also the properties of the kernel functions $G^R$ and $G^S$, by Theorem \ref{thrm:abstractholm} we obtain the holomorphic extension of the self interaction operator, and  finally since is the sum of a Freedholm operator of order $0$ and a compact operator, we also have that $V_{\br,\br}$ is Freedholm of order $0$. 

For the second part of the Lemma, we again use the structure of the fundamental solution so we have,
\begin{align*}
\mathbf{G}(\br(t),\bp(s)) = F_1(d_{\br,\bp}(t,s)^2) \log(d_{\br,\bp}(t,s)) + F_2(d_{\br,\bp}(t,s)^2),
\end{align*}

 The conclusion now follows as in the analysis of the regular part of the self-interaction case since by Lemma \ref{lemma:dcross} the map $(\br,\bp) \in K^j_{\delta_j} \times K^k_{\delta_k}\mapsto d_{\br,\bp} \in \cmspaceh{m}{h}{\iinterv}{\IC}$ is holomorphic, uniformly bounded. 
\end{proof} 

\begin{remark}
\label{remark:mm1extension}
In the above proof we see that the more restrictive conditions on $s,m,h$ are given so that the operator with kernel $\log Q_{\br}(t,s) F_1(d_{\br}(t,s)^2) \in \cmspaceh{m-1}{h}{\iinterv}{\IC}$, give rise to a bounded operator from $\IT^s$ to $\IW^{s+1}$. 
Hence the results from the previous lemma also holds if we consider that the fundamental solution is of the form 
$$
f(t,s) \mathbf{G}(\br(t),\br(s)),
$$
where $\mathbf{G}$ has the same hypotesis as before (see \eqref{eq:funsolgen}), and $f$ is such that $f(t,t) \neq 0$, and also $f \in \cmspaceh{m-1}{h}{\iinterv}{\IC}$. 
\end{remark}

Now let us consider the full weakly singular operator. By the previous lemma and the structure of the product spaces we obtain the following result. 

\begin{theorem}
\label{thrm:Vop}
Let $\xi_j$, $j=1,..,M$ defined as in Lemma \ref{lemma:vcomp}, $s \in \IR$, such that the same conditions on $s,m,h$ of Lemma \ref{lemma:vcomp} holds. Then the maps 
\begin{align*}
(\br_1,..,\br_M) \in K^1_{\xi_1} \times ... K^M_{\xi_M} \mapsto 
\mathbf{V}_{\Gamma_1,\hdots,\Gamma_m} \in \mathcal{L}(\prod_{j=1}^M \IT^s, \prod_{j=1}^M \IW^{s+1}) ,
\end{align*}
is holomorphic and uniformly bounded. Moreover, there is a positive number $\eta$, with $0< \eta < \xi_j$, for every $j \in \{1,..,M\}$ such that 
\begin{align*}
(\br_1,..,\br_M) \in K^1_{\eta} \times ... K^M_{\eta} \mapsto 
\mathbf{V}_{\Gamma_1,\hdots,\Gamma_m}^{-1} \in \mathcal{L}(\prod_{j=1}^M \IW^{s+1}, \prod_{j=1}^M \IT^{s}) 
\end{align*}
is also holomorphic and uniformly bounded.
\end{theorem}
\begin{proof}
As in the previous Lemma we will only consider the case of a scalar operator $\cP$ and the pair of spaces $T^s,W^{s+1}$. Furthermore we will consider only the case $M=2$, more general cases are proven using the same arguments. 

Notice that 
\begin{align*}
\begin{split}
\| \mathbf{V_{\Gamma_1,\Gamma_2}}\bla\|^2_{\prod_{j=1}^M T^s} =& \| V_{\br_1,\br_1} \lambda_1 + V_{\br_1,\br_2} \lambda_2 \|^2_{W^{s+1}}+
\| V_{\br_2,\br_1} \lambda_1 + V_{\br_2,\br_2} \lambda_2 \|^2_{W^{s+1}} \\
\leq& \max(\|V_{\br_1,\br_1} \|^2 ,\|V_{\br_1,\br_2} \|^2,\|V_{\br_2,\br_1} \|^2,\|V_{\br_2,\br_2} \|^2 ) (\|\lambda_1\|^2_{T^s}+ \|\lambda_2\|_{T^s}^2)\\
\leq& \max(\|V_{\br_1,\br_1} \|^2 ,\|V_{\br_1,\br_2} \|^2,\|V_{\br_2,\br_1} \|^2,\|V_{\br_2,\br_2} \|^2 ) (\| \bla\|^2_{\prod_{j=1}^M T^s}),
\end{split}
\end{align*}
where we omitted the sub-indices $\mathcal{L}(T^s,W^{s+1})$ in the norms of the different components of $\mathbf{V}_{\Gamma_1,\Gamma_2}$. From the last inequality, using the uniform bounds for the different components of $\bf{V}$ we obtain the uniform bound of the weakly-singular operator. The holomorphic property is proven in a simmilar manner using the complex Frechet derivative of each of the components. 

For the second part of the Theorem. Let us first notice that $\mathbf{V}_{\Gamma_1,\Gamma_2}$ is a zero order Freedhom operator. To see this one need to write the full operator as the sum of the diagonal term (zero order by Lemma \ref{lemma:vcomp}) and the cross-interaction terms (compact maps by Lemma \ref{lemma:vcomp}). Moreover the operator is invertible in the compact set of real geometries $K^1 \times K^2$ as it was pointed in Section \todo{ref sec boundary integrals..}. 

Now the result follows from Theorem \ref{thrm:abtractinverse}, as a particular conclusion of what we have already proved in the first part is that for $\delta = \min(\delta_1,\delta_2)$,  the map 
\begin{align*}
(\br_1,\br_2) \in K^1_\delta \times K^2_\delta \mapsto \mathbf{V}_{\Gamma_1,\Gamma_2} \in \mathcal{L}(T^s\times T^s, W^{s+1} \times W^{s+1}),
\end{align*}
is holomorphic and uniformly bounded. 
\end{proof}

\subsubsection{Hyper-Singular Integral Operator}

Now we consider the hyper-singular operator. As in the previous case we consider the general interacion $W_{\br,\bp}$, which can be written using Maue's representation as, 
\begin{align*}
W_{\br,\bp} u (t)= \frac{d}{dt}
 \int_{-1}^1 G(\br(t),\bp(s))\frac{d}{ds}u(s) ds + \int_{-1}^1\widetilde{G}(\br(t),\bp(s))u(s)ds
 \end{align*}
and hence we have $W_{\br_i,\br_j} = (W_{\Gamma_1,\hdots,\Gamma_M})_{i,j}$. The main properties of this operator are given in the following lemma. 

\begin{lemma}
\label{lemma:wcomp}
For and $s\in \IR$ such that the same conditions of Lemma \ref{lemma:vcomp} for $s,m,h,$ holds, we have:
\begin{enumerate}
\item 
Let $\xi_j< \delta_j$, for $j =1,\hdots,M$, then the map
\begin{align*}
\br \in K^j_{\xi_j} \mapsto W_{\br,\br} \in 
\mathcal{L}(\IU^s, \IM^{s-1}), 
\end{align*} 
is holomorphic, uniformly bounded, and Freedholm of order 0. 
\item 
For $j,k \in \{1,\hdots,M\}$, $j \neq k$, $s \in \IR$, and let $\xi_j < \delta_j$, and $\xi_k  < \delta_k$, then the map 
\begin{align*}
  (\br,\bp) \in K^j_{\delta_j} \times K^k_{\delta_k} \mapsto V_{\br,\bp} \in 
\mathcal{L}(\IU^s, \IM^{s-1}),
\end{align*}
is holomorphic, uniformly bounded and compact. 
\end{enumerate}
\end{lemma}

\begin{proof}
Form the Maue's representation formula we define 
\begin{align*}
W^1_{\br,\bp} = \frac{d}{dt}
 \int_{-1}^1 G(\br(t),\bp(s))\frac{d}{ds}u(s) ds,\\
 W^2_{\br,\bp} = \int_{-1}^1\widetilde{G}(\br(t),\bp(s))u(s)ds.
\end{align*}
Since $\widetilde{G}$ has the same structure of $G$ the properties of $ W^2_{\br,\bp}$ are given by Lemma \ref{lemma:vcomp}. In fact this lemma give us the result in the more restrictive space $\mathcal{L}(\mathbb{U}^s,\mathbb{M}^{s+1})$. The sole exception is the freedom order, but since the operator is in $\mathcal{L}(\mathbb{U}^s,\mathbb{M}^{s+1})$, is compact in $\mathcal{L}(\mathbb{U}^s,\mathbb{M}^{s-1})$.

Now we focus on $W^1_{\br,\bp}$, again we will only consider the scalar case. From Lemma \ref{lemma:vcomp}, and the mapping properties of the derivative operator (see \eqref{eq:devprop}), we have that $W^1_{\br,\bp} \in \mathcal{L}(U^{s},M^{s-1})$, and is uniformly bounded in $K^j_{\xi_j}$.

Moreover, the derivatives are linear operators and do not depends of the parametrizations $\br,bp$, thus by Theorem \ref{thrm:abstractholm} $W^1_{\br,\bp}$ has holomorphic extension to $K^j_{\xi_j}$, hence we conclude that the full operato $W_{\br,\bp}$ has an uniformly bounded holomorphic extension. 

Finally we need to show that $W^1_{\br,\br}$ is a Fredholm operator of order 0. We proceed as in the weakly-singular case, thus we have 
\begin{align*}
\begin{split}
W^1_{\br,\br} = \frac{d}{dt}\left( F_1 (d_{\br}(t,t)^2)
\int_{-1}^1 \log|t-s| \frac{d}{ds}u(s)ds\right) +\\
\frac{d}{dt}\int_{-1}^1(F_1 (d_{\br}(t,s)^2)-F_1 (d_{\br}(t,t)^2)) \log|t-s| \frac{d}{ds}u(s)ds+\\
\frac{d}{dt} \int_{-1}^1 \left(\log Q_{\br}(t,s)F_1(d_{\br}(t,s)^2)+F_2(d_{\br}(t,s)^2) \right)\frac{d}{ds}u(s)ds,
\end{split}
\end{align*}
using the same arguments than in the proof of Lemma \ref{lemma:vcomp}, and the properties of the derivative operator we conclude that the second and third term of the right hand side of the last equation are compact operators as maps from $U^s$ to $M^{s-1}$. For the first term we use that $d_{\br}(t,t) = 0$ for every $t \in [-1,1]$, thus it can be written as 
$$
 F_1 (d_{\br}(t,t)^2)\frac{d}{dt}
\int_{-1}^1 \log|t-s| \frac{d}{ds}u(s)ds,
$$
the result then follows by the assumption that $F_1(0)$ has inverse, and the the standard result for hyper-singular operator on a slit \todo{ref nedelec}. 
\end{proof}


\begin{remark}
\label{remark:gtildereg}
Typically $\widetilde{\mathbf{G}}$ includes explicit operations on the normal vector, thus the corresponding kernel lost one order of regularity, however this do not affect the results as it was pointed out in Remark \ref{remark:mm1extension}.

Moreover, since we only need to proof that the associated integral operator is bounded from $\IU^s$ to $\IM^{s-1}$ (instead of $M^{s+1}$), we could also consider less regular terms. In concrete is possible to consider that the fundamental solution is of the form 
$$ f(t,s) \widetilde{\mathbf{G}}(\br(t),\br(s)),$$ where $f \in \cmspaceh{m-2}{h}{\iinterv}{\IC}$, for $m \geq 2$. 
\end{remark}

The result for the full hyper-singular operator follows directly from the last lemma. The proof is similar to Theorem \ref{thrm:Vop} so is omitted. 

\begin{theorem}
\label{thrm:Wop}
Let $\xi_j$, $j=1,..,M$ defined as in Lemma \ref{lemma:wcomp}, $s \in \IR$, such that the same conditions on $s,m,h$ of Lemma \ref{lemma:vcomp} holds. Then the map
\begin{align*}
(\br_1,..,\br_M) \in K^1_{\xi_1} \times ... K^M_{\xi_M} \mapsto 
\mathbf{W}_{\Gamma_1,\hdots,\Gamma_m} \in \mathcal{L}\left(\prod_{j=1}^M \IU^s, \prod_{j=1}^M \IM^{s-1}\right) ,
\end{align*}
is holomorphic and uniformly bounded. Moreover, there is a positive number $\eta$, with $0< \eta < \xi_j$, for every $j \in \{1,..,M\}$ such that 
\begin{align*}
(\br_1,..,\br_M) \in K^1_{\eta} \times ... K^M_{\eta} \mapsto 
\mathbf{W}_{\Gamma_1,\hdots,\Gamma_m}^{-1} \in \mathcal{L}(\prod_{j=1}^M \IM^{s-1}, \prod_{j=1}^M \IU^{s}) 
\end{align*}
is also holomorphic and uniformly bounded.
\end{theorem}

An immediate conclusion of the holomorphic extension of the integral operators and its inverses, is the holomorphic extension of the solution map. 

\begin{corollary}
\label{cor:solholom}
Let $\eta,s \in \IR $ as in Theorem \ref{thrm:Vop}. Under the assumption of $g$, the right hand side of Problem \todo{ref} is an entire function we have that the solution map for the Dirichlet problem 
\begin{align*}
(\br_1,..,\br_M) \in K^1_{\eta} \times ... K^M_{\eta} \mapsto 
\bla \in \IT^{s},
\end{align*}
where $\bla$ is defined as in \todo{ref}, is holomorphic and uniforlmly bounded. Analogously, if $\eta$ is as in Theorem \ref{thrm:Wop}, and $f$ ,the right hand side of the Neumann problem \todo{ref} is also entire we have, 
\begin{align*}
(\br_1,..,\br_M) \in K^1_{\eta} \times ... K^M_{\eta} \mapsto 
\bmu \in \IU^{s},
\end{align*}
where $\bmu$ is defined as in \todo{ref}, is holomorphic and uniformly bounded.
\end{corollary}

\begin{remark}
\label{rem:functionalholorphism}
For practical applications one is typically concerned with linear functionals of $\bla$ or $\bmu$. Since $\bla$, and $\bmu$ are no the boundary data, these linear functional would typically depend of the parametrizations $\br_1, \hdots,\br_M$. For example, the far-field for the Helmholtz equation ($\cP = -\delta - k^2$)  with Dirichlet condition, on a given angle $\alpha$ is 
$$
\sum_{-1}^1 \int_{-1}^1 \lambda_j(s) e^{-ik(\cos \alpha , \sin \alpha)\cdot \br_j(s)} ds ,
$$
which depends on the parametrizations $\br_1, \hdots, \br_j$ as they appear explicitly in the exponential factor. Hence, the holomorphy extension of $\bla$ or $\bmu$ is not enough to argument that the linear functionals have holomporpic extensions. However, typical functionals could also be considered with the framework presented in this work. For example for the Helmholtz far-field one could easily see from Theorem \ref{thrm:abstractholm}, and Corollary \ref{corollary:rfop} that integral operators acting on an arbitrary $u$ of the forms
$$
\int_{-1}^1 u(s) e^{-ik(\cos \alpha , \sin \alpha)\cdot \br_j(s)} ds
$$
have holomorphic extension, thus the far-field could be seen as a composition of operators of this kind, with $\bla$, and obtain the holomorpic extension of the linear functional in this case. 
\end{remark}

\section{Parametric Holomorphism}

For practical applications one is concerned if the solution map is smooth when considering very particular families of geometries. The class of families that are typically considered are given as a parametric perturbation of a fixed configuration, i.e given a initial configuration given by the parametrizations $\br_1^0,\hdots,\br_M^0$,  the new configuration is parametrized by 
\begin{align}
\label{eq:geoY}
\br_j = \br_j^0 +\sum_{n=0}^\infty y^j_n \bb_n, \quad j =1,\hdots,M,
\end{align}
where $y^j_n \in [-1,1]$ are the parameters, and $\bb_n$ are a set of fixed functions that describe the possible perturbations to the initial configuration. The elements $\br_j$ would also be denoted by $\br_j(\by)$ to make the dependence of the parameters $\by$ explicitly.

Under this framework we are interested in the smoothness of the solution map respect to the real parameters $y^j_n$.

In order to keep well-posed problems for every possibility of the parameters $y^j_n$ we will need to make some assumptions on the initial configuration and the perturbation basis $\{\bb_n\}_{n=1}^\infty$. 

\begin{assumption}
\label{assump:geoparam}
For $m \in \IN$, with $m\geq 1$, and $h \in [0,1]$ such that $\br^0_j \in \rgeoh{m}{h}$, for $j =1,\hdots,M$ we assume that
\begin{enumerate}
\item 
The sequence $\|\bb_n\|_{\cmspaceh{m}{h}{(-1,1)}{\IR^2}}$, $n\in \IN$ is $p-$sumable for some $p \in (0,1)$.
\item 
$$\sum_{n=0}^\infty \sup_{t \in (-1,1)} \|  \bb_n'(t)\| < \inf_{t \in (-1,1)}\|(\br^0_j)'(t)\|,\quad \forall
j \in \{1,\hdots,M\}.
$$
\item
There exist a sequence of points $(\bc_j)_{j=1}^M \subset \IR^2$, and a sequence of strictly positive real numbers $(\tau_j)_{j=1}^M $, both depending of $(\br^0)_{j=1}^M$ such that 
$$\sup_{t\in(-1,1)} \| \br_j(t) - \bc_j \| < 2\tau_j,\ \forall j \in \{1,\hdots,M\},$$
and also assume that 
$$
\sum_{n=0}^\infty \sup_{t \in (-1,1)} \| \bb_n(t)\| <  \tau_j,\ \forall j \in \{1,\hdots,M\}.
$$
\end{enumerate}
\end{assumption}
\begin{remark}
The first two points in the previous assumption ensure that 
$$
\inf_{t \in (-1,1)} \| \br_j(\by)'(t) \| > 0 \ \forall j \in \{1,\hdots,M\}.
$$
and also that not self crossings occurs. The final point ensure that not crosses between different arcs occurs.
\end{remark}
Obviously to be able to apply the results presented in Section \todo{ref}, we would need that the families $\{\br_j(\by), \by \in [-1,1]^{\IN} \} \subset \rgeoh{m}{h}$, are a compact subset for every $j \in \{1,\hdots,M\}$. This is consequence of standard results of general topology as is show in what follows. 

\begin{lemma}
\label{lemma:parcompact}
For every $j \in  \{1,\hdots,M\}$, the sets $\{\br_j(\by), \by \in [-1,1]^{\IN} \}$  are compact subsets of $\rgeoh{m}{h}$. 
\end{lemma}
\begin{proof}
Since $[-1,1]^{\IN}$ is compact under the product topology \footnote{this is the classical Tychonoff's equivalence to the axiom of choice.} the result is equivalent to show that the map 
\begin{align}
\label{eq:mapys}
\by \in [-1,1]^{\IN} \mapsto \br_j(\by) \in \rgeoh{m}{h},
\end{align}
is continuous. We proof this using neighborhoods, let us consider a ball of radius $\epsilon$ with center $\br_j(\bx)$, for a fixed parameter $\bx$. By the assumptions on the perturbation basis $\bb$ we can find $N \in \IN$ such that $\sum_{n>N}\|\bb_n\|_{\rgeoh{m}{h}}< \frac{\epsilon}{2}$. Then consider the set 
$$
A_N =\left\lbrace \by \in [-1,1]^{N}:  \sum_{n=0}^N |y_n-x_n| \|\bb_n \|_{\rgeoh{m}{h}} < \frac{\epsilon}{2}\right\rbrace,
$$
Notice that $A_N$ is an open subset of $[-1,1]^N$ since the map $ \by \in [-1,1]^N \mapsto  \br_j^0 + \sum_{n=0}^N y_n \bb_n$, is continuous\footnote{This maps can be interpreted as an afin map between two finite dimensional spaces}. Thus, we denote by $\widetilde{A}_N$, the canonical embedding of $A_N$ into $[-1,1]^\IN$ which by definition of product topology is an open set in $[-1,1]^N$, finally we notice that for $\by \in \widetilde{A}_N$ we have 
\begin{align*}
\begin{split}
\|\br_j(\by) - \br_j(\bx)\|_{\rgeoh{m}{h}} 
\leq 
\sum_{n=0}^N |y_n-x_n|\|\bb_n \|_{\rgeoh{m}{h}} +\\ \sum_{n >N}\|\bb_n \|_{\rgeoh{m}{h}} < \epsilon
\end{split}
\end{align*}
thus since $\epsilon$ is arbitrary we have that for any neighborhood of $\br_j(\bx)$ there is an open set ($\widetilde{A}_N$) such that the image of the latter under the map \ref{eq:mapys} is contained in the given neighborhood of $\br_j(\bx)$, which is the continuity that we needed.
\end{proof}


Now we recall the the notion of $(b,\epsilon,p)$-holomrphic which was introduced in \todo{add refs...}.

 For a sequence $\{\rho_n\}_{n \in \IN}$ of real numbers, such that $\rho_n  >1$ we define a sequence of complex tubes by 
\begin{align*}
\mathcal{T}_n := \left\lbrace z \in \IZ : \text{dist}(z,[-1,1]) \leq \rho_n-1 \right\rbrace.
\end{align*}
Notice that for every $n \in \IN$, $\mathcal{T}_n$ is a closed domain, we will make use of the following family of tubes 
\begin{align*}
\mathcal{T}_{\rho} = \prod_{n \in \IN} \mathcal{T}_{\rho_n} \supseteq [-1,1]^{\IN}, \quad \text{and,} \quad
\mathcal{T}^M_{\rho} = \prod_{k=1}^M \mathcal{T}_{\rho}.
\end{align*}

\begin{definition}
Given $\epsilon >0$, $p \in (0,1)$, a real $p$-sumable sequence $\{b_n\}_{n \in \IN}$ . We say that $f:\prod_{j=1}^M[-1,1]^\IN\rightarrow B$, where $B$ is a Banach space, is a ($b$,p,$\epsilon$)-holomorphy function if
\begin{enumerate}
\item The function $f$ is bounded.
\item 
For any real sequence $\{\rho_n\}_{n \in \IN}$ of real numbers, such that $\rho_n >1$ that satisfy 
\begin{align*}
\sum_{n \in \IN} (\rho_n -1) b_n < \epsilon, 
\end{align*}
The function $f$ admits a extension to a set of the form $\mathcal{O}^M_\rho = \prod_{j=1}^M \prod_{n \in \IN} \mathcal{O}_{\rho_n} \supset \mathcal{T}^M_\rho$, where each set $\mathcal{O}_{\rho_n}$ is open, and the extension is holomophic in each variable. 
\item 
There exist a set $\widetilde{\mathcal{O}}_\rho^M =  \prod_{j=1}^M \prod_{n \in \IN} \widetilde{\mathcal{O}}_{\rho_n}$ such that for every $n \in \IN$, $\widetilde{\mathcal{O}}_{\rho_n} \supset \overline{\mathcal{O}}_{\rho_n}$, and $\widetilde{\mathcal{O}}_{\rho_n}$ is an open subset. Futhermore, $f$ can also be exteded to $\widetilde{\mathcal{O}}_\rho^M$ and the extension is bounded as 
\begin{align*}
\sup_{\mathbf{z} \in \widetilde{\mathcal{O}}_\rho^M } \| f(\mathbf{z})\|_B \leq C(\epsilon).
\end{align*}
\end{enumerate}
\end{definition}

Let us denote by $\mathbf{V}(\by^1,\hdots,\by^M)$, and $\mathbf{W}(\by^1,\hdots,\by^M)$ the weakly-singular and hyper-singular operators acting on a geometric configuration of the canonical form \eqref{eq:geoY}. An immediate consequence of the holomorphy dependence of the parametrizations (theorems \ref{thrm:Vop}, and \ref{thrm:Wop}) is the parametric holomorphism of the operators and its inverses. 

\begin{theorem}
\label{thrm:bepholm}
Let $m \in \IN$, $h \in [0,1]$, $s \in \IR$, such that the conditions of Lemma \ref{lemma:vcomp} holds. Also denote by $\{b_n\}_{n\in \IN}$ the sequence of positive numbers given by $b_n = \| \bb \|_{\rgeoh{m}{h}}$, then under Assumption \ref{assump:geoparam} we have 
\begin{enumerate}
\item 
If $\epsilon < \xi_j$, for every $j \in \{1,\hdots,M\}$, where $\chi_j$ are given in Theorem \ref{thrm:Vop}, we have that the map 
\begin{align*}
(\by^1,\hdots, \by^M ) \mapsto \mathbf{V}(\by^1,\hdots,\by^M) \in \mathcal{L}\left(\prod_{j=1}^M \IT^s, \prod_{j=1}^M \IW^{s+1}\right), 
\end{align*}
is $(b,p,\epsilon)-$holomorphic, and continuous in product topology. 
\item 
If $\epsilon < \eta$, where $\eta$ is also given by Theorem \ref{thrm:Vop} then 
\begin{align*}
(\by^1,\hdots, \by^M ) \mapsto \mathbf{V}^{-1}(\by^1,\hdots,\by^M)^ \in \mathcal{L}\left(\prod_{j=1}^M \IW^{s+1}, \prod_{j=1}^M \IT^{s}\right), 
\end{align*}
is $(b,p,\epsilon)-$holomorphic, and continuous in product topology. 
\item 
If $\epsilon < \xi_j$, for every $j \in \{1,\hdots,M\}$, where $\chi_j$ are given in Theorem \ref{thrm:Wop}, we have that the map 
\begin{align*}
(\by^1,\hdots, \by^M ) \mapsto \mathbf{W}(\by^1,\hdots,\by^M) \in \mathcal{L}\left(\prod_{j=1}^M \IU^s, \prod_{j=1}^M \IM^{s-1}\right), 
\end{align*}
is $(b,p,\epsilon)-$holomorphic, and continuous in product topology. 
\item 
If $\epsilon < \eta$, where $\eta$ is also given by Theorem \ref{thrm:Wop} then 
\begin{align*}
(\by^1,\hdots, \by^M ) \mapsto \mathbf{W}^{-1}(\by^1,\hdots,\by^M) \in \mathcal{L}\left(\prod_{j=1}^M \IM^{s-1}, \prod_{j=1}^M \IU^{s}\right), 
\end{align*}
is $(b,p,\epsilon)-$holomorphic, and continuous in product topology. 
\end{enumerate}
The parameter $p$ is the same than in Assumption \ref{assump:geoparam}. 
\end{theorem} 
\begin{proof}
We will only consider the first case, the other three follows with similar arguments.

First notice that the map 
\begin{align}
\label{eq:yVmap}
(\by^1,\hdots, \by^M) \mapsto \mathbf{V}(\by^1,\hdots, \by^M),
\end{align}
can be written as the following composition
$$
(\by^1,\hdots, \by^M) \mapsto  (\br_1(\by^1),\hdots,\br_M(\by^M))
\mapsto 
\mathbf{V}_{\Gamma_1,\hdots,\Gamma_m}, 
$$
where the geometry configuration $\Gamma_1,\hdots,\Gamma_M$ is given by the parametrization $\br_1(\by^1),\hdots,\br_M(\by^M)$. From this representation and Lemma \ref{lemma:parcompact} the continuity in the product topology follows. Let us denote by $\mathbf{P}$ the map 
$$
\mathbf{P} : (\by^1,\hdots, \by^M) \mapsto  (\br_1(\by^1),\hdots,\br_M(\by^M)),
$$
following the same arguments of Lemma \ref{lemma:parcompact} we have that $\mathbf{P}$ is continuous map with respect to the product topology respect to the following spaces 
\begin{align*}
\mathbf{P} : \prod_{j=1}^M [-1,1]^{\IN} 
\mapsto
\prod_{j=1}^M K^j \subset \prod_{j=1}^M \rgeoh{m}{h},
\end{align*}
where $K^j =\{ \br_j(\by): \by \in [-1,1]^{ \IN}\}$ are compact. Now we will consider extensions of $P$, given a sequence of real numbers $\{\rho_n\}_{n \in \IN}$, such that $\rho_n >1$ for every $n \in \IN$, and $\sum_{n=0}^\infty  b_n (\rho_n -1) < \epsilon$, we can construct another real sequence $\{r_n\}_{n \in \IN}$, such that $r_n > \rho_n$ for every $n\in \IN$, and also  $\sum_{n=0}^\infty  b_n (r_n -1) < \epsilon $. Finally we consider the family of open sets $O_n = \left\lbrace z \in \IC: \ \text{dist}([-1,1],z) < r_n-1\right\rbrace$. We see that $\mathcal{T}_{\rho_n} \subset O_n$, and we define a extension of $\mathbf{P}$ over $\prod_{j=1}^\M \prod_{n \in \IN} O_n$ as, 
$$
\mathbf{P} : (\bz^1,\hdots,\bz^M) \mapsto \left(r^0_1 + \sum_{n=0}^\infty z^1_n \bb_n , \hdots, r^0_M + \sum_{n=0}^\infty z^M_n \bb_n \right).
$$
Now we describe the range of this last operator. Consider $\bz^j \in \prod_{n \in \IN} O_n$, by definition there exist $\by^j \in [-1,1]^{\IN}$ such that $|z^j_n - y^j_n| < r_n-1$ for every $n \in \IN$, hence 
$$
\| \br_j(\bz^j) - \br_j(\by^j)\|_{\cmspaceh{m}{h}{(-1,1)}{\IC^2}} < \sum_{n=0}^\infty b_n (r_n -1) < \epsilon,
$$
where the last inequality follows from the construction of $r_n$. This implies that the range of $\mathbf{P}$ is contained in $\prod_{j=1}^M K^j_\epsilon$\footnote{we recall that $K^j$ are defined as $\{\br_j(\by): \by \in [-1,1]^\IN \}$, and $K^j_\epsilon$ is defined acordding to \eqref{eq:openext}.}. Now we consider the extension of \eqref{eq:yVmap} defined by 
\begin{align*}
\begin{split}
(\bz^1,\hdots,\bz^M) \in \prod_{j=1}^\M \prod_{n \in \IN} O_n \mapsto (\br_1(\bz^1),\hdots,\br_M(\bz^M)) \in \prod_{j=1}^M K^j_\epsilon \\\mapsto
V_{\Gamma_1,\hdots,\Gamma_M} \in \mathcal{L}\left(\prod_{j=1}^M \IT^s, \prod_{j=1}^M \IW^{s+1}\right) ,
\end{split}
\end{align*}
which is well defined by Theorem \ref{thrm:Vop}. Moreover also from the same theorem we also get that this map is uniformly bounded, since 
\begin{align*}
\begin{split}
\sup_{(\bz^1,\hdots,\bz^M) \in \prod_{j=1}^\M \prod_{n \in \IN} O_n} \| V(\bz^1,\hdots,\bz^M ) \|_{\mathcal{L}\left(\prod_{j=1}^M \IT^s, \prod_{j=1}^M \IW^{s+1}\right)}  \leq \\
\sup_{(\br_1,\hdots,\br_M) \in \prod_{j=1}^\M K^j_\epsilon} \| V_{\Gamma_1,\hdots,\Gamma_M}\|_{\mathcal{L}\left(\prod_{j=1}^M \IT^s, \prod_{j=1}^M \IW^{s+1}\right)} < \infty.
\end{split}
\end{align*}
Also since the map $z^j_n \in O_n \mapsto (\br_1(\bz^1),\hdots,\br_M(\bz^M))$ is a one dimensional linear map, is holomorphic, thus the map 
\begin{align*}
z^j_n \in O_n \mapsto (\br_1(\bz^1),\hdots,\br_M(\bz^M)) \mapsto V_{\Gamma_1,\hdots,\Gamma_M},
\end{align*}
 is also holomorphic by theorem \ref{thrm:Vop}. Hence since the map is uniformly bounded and holomorphic with respect to each variable we have that the map is $(b,p,\epsilon)-$holomorphic.
\end{proof}

As in the previous section we can use the properties of the inverse of the associated integral operators to obtain a result on the regularity of the solution map. The following result state exactly that. The proof is omitted as is the same than of the previous theorem. 

\begin{corollary}
\label{corollary:bepsol}
Under the hypotesis of Theorem \ref{thrm:bepholm}. If $g$ the right-hand-side of the Dirichlet problem is an entire function then the map 
\begin{align*}
(\by^1,\hdots,\by^M) \mapsto \bla \in \prod_{j=1}^M \IT^s,
\end{align*}
where $\bla$ denotes the solution of the Dirichlet problem, is $(b,p,\epsilon)-$holomorphic with $\epsilon$ as in the second point of Theorem \ref{thrm:bepholm}, and continuous in product topology. Similarly, if $f$ the right-hand-side of the Neumann problem, is an entire function, then the map 
\begin{align*}
(\by^1,\hdots,\by^M) \mapsto \bmu \in \prod_{j=1}^M \IT^s,
\end{align*}
where $\bmu$ denotes the solution of the Neumann problem, is $(b,p,\epsilon)-$holomorphic, with $\epsilon$ as in the fourth point of Theorem \ref{thrm:bepholm}, and continuous in product topology.
\end{corollary}

\begin{remark}
It is also possible to establish the $(b,p,\epsilon)-$holomorphism for a number of   linear functional acting on $\bla, \bmu$. The results follows by showing first that the linear functional has an holomorphic extension (as in Remark \ref{rem:functionalholorphism}), and then this would implies the $(b,p,\epsilon)-$holomorphism as in Theorem \ref{thrm:bepholm}.
\end{remark}

The final part of this section is devoted to establish the $(b,p,\epsilon)-$holomorphism for approximations of $\bla$,$\bmu$. We consider approximations, $\bla_N \in \prod_{j=1}^M\IT_N  \subset \prod_{j=1}^M\IT^s$, and $\bmu_N \in \prod_{j=1}^M \IU_N \subset \prod_{j=1}^M \IU^s$ for every $s \in \IR$, where the dimension of $\IT_N$ (resp. $\IU_N$) is $N+1$. Furthermore we assume that the families $\{\IT_N\}_{N \in \IN}$, and $\{\IU_N\}_{N \in \IN}$ are dense in $\IT^s$, and $\IU^s$ respectively (for every $s$ in $\IR$). Under this considerations the approximations are seeks as the solutions of the discrete problems: 
\begin{align*}
P_N \mathbf{V}(\by^1,\hdots,\by^M) \bla_N = P_N \bg,\\
Q_N \mathbf{W}(\by^1,\hdots,\by^M) \bmu_N = Q_N \mathbf{f},
\end{align*}
where the operators, $P_N,Q_N$ are linear continuous maps from $\prod_{j=1}^M \IW^{s+1}$ (resp. $\prod_{j=1}^M \IM^{s-1}$) to a subspace of finite dimension $\prod_{j=1}^M \IW_N$ (resp. $\prod_{j=1}^M \IM_N$), and also independent of the geometry.

First let us remark that the operators $ \mathbf{V}_{\Gamma_1,\hdots,\Gamma_M} \in \mathcal{L}\left(  \prod_{j=1}^M\IT_N, \prod_{j=1}^M\IW^{s+1}\right)$, and $ \mathbf{W}_{\Gamma_1,\hdots,\Gamma_M} \in \mathcal{L}\left(  \prod_{j=1}^M\IU_N, \prod_{j=1}^M\IM^{s-1}\right)$, have holomorphic extensions in the parametrizations, in fact the proof is the same than in Theorems \ref{thrm:Vop}, and \ref{thrm:Wop}, with the difference that domain of the operators are $\prod_{j=1}^M\IT_N$ and $\prod_{j=1}^M\IU_N$ respectively, which are Banach spaces since they are finite dimensional subspaces.

On the other hand, since $P_N,Q_N$ are linear and independent of the geometry we obtain that the maps 
$P_N\mathbf{V}_{\Gamma_1,\hdots,\Gamma_M} \in \mathcal{L}\left(  \prod_{j=1}^M\IT_N, \prod_{j=1}^M\IW_N\right)$ and $Q_N\mathbf{V}_{\Gamma_1,\hdots,\Gamma_M} \in \mathcal{L}\left(  \prod_{j=1}^M\IU_N, \prod_{j=1}^M\IM_N\right)$ also have holomorphic extensions in the parametrization variables. Hence arguing as in point one of Theorem \ref{thrm:bepholm} we obtain the $(b,p,\epsilon)-$holomorphism of  $P_N\mathbf{V}(\by^1,\hdots,\by^M)$ and $Q_N\mathbf{W}(\by^1,\hdots,\by^M)$.

If the operators $P_N\mathbf{V}(\by^1,\hdots,\by^M)$, $Q_N\mathbf{W}(\by^1,\hdots,\by^M)$ were invertible we would obtain the $(b,p,\epsilon)$-holomorphism of $\bla_N$, $\bmu_N$ arguing as in Theorem \ref{thrm:bepholm}. However one typically only have the following condition 
\begin{assumption}
\label{assumption:discinverze}
For every $(\by^1,\hdots,\by^M) \in \prod_{j=1}^M [-1,1]^\IN$ there exist an integer number $N_0(\by^1,\hdots,\by^M)$, such that for $N>N_0(\by^1,\hdots,\by^M)$ the operators $P_N \mathbf{V}(\by^1,\hdots,\by^M)$ have inverse and also 
$$\left\vert \left\vert (P_N V(\by^1,\hdots,\by^M))^{-1} \right \vert \right \vert_{\mathcal{L}\left(\prod_{j=1}^M \IW^{s+1}, \prod_{j=1}^M \IT^s\right)} \leq C(s,\by^1,\hdots,\by^M),$$
where $C$ do not depend of $N$. The same holds (with potentially different value of $N_0$, and $C$, and the corresponding changes in the functional spaces) for $Q_N\mathbf{W}(\by^1,\hdots,\by^M)$.
\end{assumption}

The following result ensure that we can select a single value of $N_0$ such that $P_N\mathbf{V}(\by^1,\hdots,\by^M)$ and $Q_N\mathbf{W}(\by^1,\hdots,\by^M)$ have inverse for $N>N_0$, and these inverses are uniformly bounded.
\begin{lemma}
\label{lemma:discreteinverse}
Under Assumption \ref{assumption:discinverze} there exist $N_0 \in \IN$, such that for $N> N_0$ the operator $P_N \mathbf{V}(\by^1,\hdots,\by^M)$ has inverse, and also 
\begin{align*}
\sup_{(\by^1,\hdots,\by^M)\in \prod_{j=1}^M[-1,1]^\IN}\left\vert \left\vert (P_N V(\by^1,\hdots,\by^M))^{-1} \right \vert \right \vert_{\mathcal{L}\left(\prod_{j=1}^M \IW^{s+1}, \prod_{j=1}^M \IT^s\right)} \leq C(s),
\end{align*}
where $C$ do not depend of $N$. And the same holds  (with the adequate modifications) for $Q_N \mathbf{W}(\by^1,\hdots,\by^M)$.
\end{lemma}
\begin{proof}
Fix $s \in \IR$ and consider the sets
\begin{align*}
\begin{split}
&A_{N,C} = \\ &\left\lbrace 
(\by^1,\hdots,\by^M) \in \prod_{j=1}^M[-1,1]^{\IN} \ : \ \left\vert \left\vert (P_N V(\by^1,\hdots,\by^M))^{-1} \right\vert \right\vert_{\mathcal{L}\left(\prod_{j=1}^M \IT^s,\prod_{j=1}^M \IW^{s+1}\right)} < C, \text{ for $N>N_0$}
\right\rbrace.
\end{split}
\end{align*}
Is clear from Assumption \ref{assumption:discinverze} that the family $\{A_{N,C}\}_{N \in \IN, C > 0}$ is a cover of the compact space $\prod_{j=1}^M[-1,1]^{\IN}$, thus if we show that each $A_{N,C}$ is open we can select a finite sub-cover and the results follows directly. Arguing as in the proof of Lemma \ref{lemma:parcompact} the sets $A_{N,C}$ are open if the sets 
\begin{align*}
\begin{split}
&B_{N,C} = \\ &\left\lbrace 
(\br^1,\hdots,\br^M) \in \prod_{j=1}\rgeoh{m}{h} \ : \ \left\vert \left\vert (P_N V_{\Gamma_1,\hdots,\Gamma_M})^{-1} \right\vert \right\vert_{\mathcal{L}\left(\prod_{j=1}^M \IW^{s+1}\right)} < C, \text{ for $N>N_0$}
\right\rbrace.
\end{split}
\end{align*}
are open (since $A_{N,C}$ are the pre-image of $B_{N,C}$ by a continuous map). The result then follows  from the fact that the map $\br^1,\hdots \br^M \mapsto P_N V_{\Gamma_1,\hdots,\Gamma_M} $  is continious (in fact has holomorphic extension as proven in Theorem \ref{thrm:Vop}), and using that the inverse map is also a continuous mapping.  
The result for $Q_N \mathbf{W}(\by^1,\hdots,\by^M)$ follows the same ideas.
\end{proof}

Now following Theorem \ref{thrm:bepholm} we obtain the result for the approximated solutions.

\begin{corollary}
Under the hypothesis of Theorem \ref{thrm:bepholm}, and assumption \ref{assumption:discinverze}. There exist $\epsilon >0$, and $N_0 \in \IN$ given by Lemma \ref{lemma:discreteinverse}, such that if $\bg, \bf$ the right-hand-sides of the Dirichler and Neumann problems are  entire functions, the maps 
\begin{align*}
(\by^1,\hdots,\by^M) \mapsto \bla_N \in  \prod_{j=1}^M \IT_N ,\\
(\by^1,\hdots,\by^M) \mapsto \bmu_N \in  \prod_{j=1}^M \IU_N ,
\end{align*}
are $(b,p,\epsilon)-$holomorphic and continuous in product topology for $N > N_0$. 
\end{corollary} 

\section{Examples for Particular Integral operators.}
Now we consider particular cases of the operator $\cP$. In particular we will consider the scalar Helmholtz case ($\cP  = -\Delta - k^2$) and the elastic wave operator 
($\cP = 
\alpha \Delta + (\alpha +\beta) \nabla \nabla \cdot  + \omega^2
$), which is a vectorial operator.

\subsection{Helmholtz Operators}
The helmhotz operator is $\cP = -\Delta -k^2$, with a fixed parameter $k \in \IR$, $k\neq 0$, typically called wavenumber. For this operator the co-normal trace is reduced to the classical Neumann trace, $\mathbf{\mathcal{B}}_{\bn}u = \bn \cdot \nabla u  = \partial_{\bn} u$, and the radiation condition is the classical Somerfeld condition, 
$$
\lim_{\|\bx\| \rightarrow \infty}
\| \bx \|^{\frac{1}{2}}\left( \frac{\partial u}{\partial \bx}-i k u\right)=0.
$$
The corresponding fundamental solution is
$$
G(\bx,\by) = \frac{i}{4}H^{(1)}_0(k \| \bx-\by\|),
$$
where $H^{(1)}_0$ denotes the zero-order Hankel function of first kind, and is defined as $H^{(1)}_0(z) = J_0(z) + i Y_0(z)$, where $J_0, Y_0$ are the zeroth order Bessel function of first and seccond kind respectively. From \todo{ref stegusson, 9.1.12 and 9.1.13} we have that 
$$
G(\bx,\by) = \log\|\bx-\by\|\left(\frac{J_0(k \|bx -\by\|)}{-2\pi}\right)+ R(k\|bx-by\|^2),
$$
where the first kind Bessel function $J_0$, and $R$ are entire functions.
Furthermore also from \todo{ref steguson 9.1.12} we have that $J_0(0) = 1$, and also $J_0(z) = j_0(z^2)$, with $j_0$ an entire function.  Hence, we have a representation of the form of \eqref{eq:funsolgen}, with $F_1(z) = \frac{j_0(k z)}{-2\pi}$, and $F_2(z) = R(k z)$.

From the last observations the holomorphism results for the Dirichlet problems follows directly. For the Neumann problem we make use of the corresponding Maue's representation formula  formula \todo{ref sauter?},
\begin{align*}
\begin{split}
(W_{\Gamma_1,\hdots,\Gamma_M})_{i,j}u = \frac{d}{dt} \int_{-1}^1 G(\br_i(t),\br_j(s)) \frac{d}{ds} u(s) ds \\- k^2\int_{-1}^1 \frac{\br_i'(t)\cdot \br_j'(s)}{\|\br_i'(t)\|} G(\br_i(t),\br_j(s))u(s) ds. 
\end{split}
\end{align*} 
thus, following the notation of Section \ref{sec:bif} we have that 
$$
\widetilde{G}(\br_i(t),\br_j(s)) = -k^2 \frac{\br_i'(t)\cdot \br_j'(s)}{\|\br_i'(t)\|} G(\br_i(t),\br_j(s)).
$$
Notice that this function is has almost the same structure that $G(\br_i(t),\br_j(s))$ except for the fact that it lost one degree of smoothness because of the factors $\br_i'$, and $\br_j'$, however, as it was noticed in Remark \ref{remark:mm1extension} this do not have any impact, and we obtain the corresponding results for the Neumann problem.

Finally let us remark the special case $k=0$, which correspond to the Laplace equation. While most of the result follows directly with fundamental solution $G(\bx,\by) = \frac{-1}{2\pi} \log \| \bx - \by \|$, special care has to be put in the radiation condition of this problem to be well posed. One particular alternative is to impose that the solutions decay at infinity and change the space $T^s$ for the subspace of functions such that $\langle u, 1 \rangle = 0$, see \todo{ref sthephan, paper arcs} for more details on the Laplace case. 
\subsection{Elastic Wave Operators}

Now let us consider the operator $\cP = \alpha \Delta + (\alpha + \beta) \nabla \nabla \cdot + \omega^2$. The parameters $\alpha,\beta$ are called Lame parameters \footnote{Tipically this are denoted $\lambda, \mu = \alpha, \beta$ but we have changed to avoid confusion with the solutions of the Dirichlet and Neumann problems denoted by $\bla, \bmu$.}, and $\omega$ is the frequency parameter. These parameters are  fixed real numbers and we also impose that  $\beta > 0$, $\alpha > - \beta$, $\omega \neq 0$. We also define two associated wavenumbers $$
k_p^2 = \frac{\omega^2}{\alpha+2\beta}, \quad
k_s^2 = \frac{\omega^2}{\beta}.$$  
The co-normal trace correspond to the tractions, defined as: 
\begin{align*}
\mathcal{B}_{\bn} \mathbf{u} =
\alpha \mathbf{n} (\div(\mathbf{u}))
+ 2 \beta \partial_{\mathbf{n}} \mathbf{u}+
\beta  \mathbf{n}^\perp (\div(\mathbf{u}^\perp))
\end{align*}
where the $\mathbf{v}^\perp  = (v_2,-v_1)$. For a vector function $\bu$ we introduce the Helmholtz decomposition: 
\begin{align*}
\mathbf{u}_p := \frac{-1}{k^2_p}\nabla \nabla \cdot \mathbf{u}, \quad
\mathbf{u}_s := \mathbf{u} - \mathbf{u}_p,
\end{align*} 
and the radiations conditions associated with this operator are the Somerfeld condition applied to $\bu_p, \bu_s$. 
The corresponding fundamental solution given by 
\begin{align}
\label{eq:efunsol}
\mathbf{G} (\mathbf{x}, \mathbf{y}) := 
\frac{i}{4 \beta} H^{(1)}_0(k_s d)\mathbf{I}+ \frac{i}{4\omega^2} \nabla_\mathbf{x} 
\nabla_\mathbf{x} \cdot \left( 
H^{(1)}_0(k_sd ) - H^{(1)}_0(k_p d)
\right),
\end{align}
where $d = \| \bx -\by\|$, and $\mathbf{I}$ denotes the identity matrix. Alternatively following \todo{reff kress} this can be expressed as 
\begin{align*}
\mathbf{G}(\mathbf{x},\mathbf{y})  = G^1(d) \mathbf{I} + 
{G}^2(d)\mathbf{D}(\mathbf{x}-\mathbf{y}),
\end{align*}
where  $D(\mathbf{d}) = \frac{\mathbf{d} \mathbf{d}^t}{\|\mathbf{d}\|^2}$, and 
\begin{align*}
G^1(d) = \frac{i}{4\beta} H_{0}^{(1)}(k_s d) - \frac{i}{4\omega^2d}(k_s H_1^{(1)}(k_s d)- k_p H_1^{(1)}(k_p d)) \\
G^2(d) = \frac{i}{4\omega^2} \left( 
\frac{2k_s H^{(1)}_1(k_s d)-2k_p H^{(1)}_1(k_p d)}{d}+
k_p^2H^{(1)}_0(k_p d)- k_s^2H^{(1)}_0(k_s d)  
\right)
\end{align*}
Using the expansion of the Hankel functions \todo{ref abramowish seguson} see that we can express $G^1,G^2$ could be expressed in the form 
$$
G^j(d) = R^j(d) +\log(d)J^j(d), \quad j=1,2.
$$
where $R^1,R^2$ are entire functions on the variable $d^2$, and 
\begin{align*}
J^1(d) = -\frac{J_0(k_s d)}{2\pi\beta}+ \frac{1}{2\pi\omega^2 d}(k_sJ_1(k_s d) - k_p J_1(k_p d)), \\
J^2(d) = \frac{-1}{2 \pi \omega^2} \left(
\frac{2k_s J_1(k_sd)-2k_p J_1(k_p d)}{d} +k_p^2J_0(k_p d) -k_s^2J_0(k_s d)
\right),
\end{align*}
hence from the serie expansion of the Bessel function we have $J^1,J^2$ are entire functions in the $d^2$ variable, and also 
\begin{align*}
J^1(0) = \frac{-1}{2\pi\beta} + \frac{-1}{4 \pi\omega^2}(k_p^2-k_s^2), \quad J^2(0) = 0.
\end{align*}
Hence we can express the fundamental solution as 
\begin{align*}
\mathbf{G}(\bx,\by) = \log(d)\mathbf{J}+ \mathbf{R},
\end{align*}
where $\mathbf{J} = J^1(d) \mathbf{I} +J^2(d)\mathbf{D}(\bx-\by)$, and 
$\mathbf{R} = R^1(d) \mathbf{I} +R^2(d) \mathbf{D}(\bx-\by)$. The only difference with the canonical expression  \eqref{eq:funsolgen} is the presence of factor $\mathbf{D}(\bx-\by)$. Let us study the properties of this factor. 

\begin{lemma}
\label{lemma:Dmatrix}
Consider two arcs $\br$, and $\bp$ and define the matrix function $\mathbf{D}_{\br,\bp}$, with components
\begin{align*}
(D_{\br,\bp}(t,s))_{j,k} :=  \begin{cases}
\frac{(r_j(t)-p_j(s))\cdot (r_k(t)-p_k(s))}{d_{\br,\bp}^2(t,s)} \quad \br \neq \bp\text{, or, } t \neq s \\
\frac{r_j(t) r_k(s)}{\br'(t)\cdot \br'(s)} \quad \text{i.o.c.} 
\end{cases} \quad j,k=1,2,
\end{align*} 
and also two compact sets $K^1,K^2 \subset \rgeoh{m}{h}$ for some $m \in \IN$. 
\begin{enumerate}
\item
If $K^1 = K^2$, and $m \geq 1$, if we select $\delta$ as in Lemma \ref{lemma:dwelldef} then 
\begin{align*}
\br \in K^1_\delta \mapsto ({D}_{\br,\br})_{j,k} \in \cmspaceh{m-1}{h}{(-1,1)\times(-1,1)}{\IC}, \quad j,k =1,2,
\end{align*} 
is holomorphic and uniformly bounded.
\item 
If $K^1, K^2$ are disjoint sets and select $\delta_1, \delta_2$ as in Lemma \ref{lemma:dcross}, then 
\begin{align*}
(\br,\bp)  \in K^1_{\delta_1} \times K^2_{\delta_2} \mapsto ({D}_{\br,\bp})_{j,k} \in \cmspaceh{m}{h}{(-1,1)\times(-1,1)}{\IC}, \quad j,k =1,2,
\end{align*} 
is holomorphic and uniformly bounded.
\end{enumerate}
\end{lemma} 
\begin{proof}
For the first part, if $t \neq s$ we have, 
\begin{align*}
(D_{\br,\br}(t,s))_{j,k} = 1/Q_{\br}(t,s) \left( \frac{r_j(t)-r_j(s)}{t-s} \right) \cdot \left( \frac{r_k(t)-r_k(s)}{t-s} \right)
\end{align*}
and the results follows as in the proof of Lemma \ref{lemma:Qfun}.
The second part is direct from Lemma \ref{lemma:dcross} and elementary results of complex variable. 
\end{proof}

We conclude that this function has holomorphic extension, but it also lose one order of regularity (since the components are in $\cmspaceh{m-1}{h}{\iinterv}{\IC}$). This implies that the proof of Lemma \ref{lemma:vcomp} the corresponding functions $F_1(d_{\br}(t,s)^2)$, $F_2(d_{\br}(t,s)^2)$ are in $\cmspaceh{m-1}{h}{\iinterv}{\IC}$ instead of $\cmspaceh{m}{h}{\iinterv}{\IC}$.  However this do not moddify the condition on $s,m,h$ as it was noticed in Remark \ref{remark:mm1extension}.

For the Neumann problem we need a Maue's representation formula. While we are not aware of a proper formula for this case, a result with the same properties is given in \todo{kress On the numerical solution of a hypersingular integral equation for elastic scattering from a planar crack}. In this paper the authors use the Maue's representation for the kernel of the static case ($\omega = 0$), that was showed in \todo{ref nedelec Integral equations with non integrable kernels}, and then proceed to explicitly find the kernel given by the co-normal traces, in both variables, of the difference of the fundamental solution \eqref{eq:efunsol}  minus the one coming from the static case. The resulting function only has a logarithmic singularity and hence in principle could be considered as part of $\widetilde{\mathbf{G}}$ in the general Maue's representation formula \eqref{eq:mauesrep}. The only difference is that this function includes part related to the second derivative of the parametrizations, hence is only in $\cmspaceh{m-2}{h}{\iinterv}{\IC}$, for $m\geq 2$, but this do not modify the results as it was noticed in Remark \ref{remark:gtildereg}. 

\bibliography{biblio}
\bibliographystyle{siam}

\appendix
\section{Fractional norms for periodic functions}
\label{appendix:fracbivariate}
This Appendix show how we get bounds for the fractional semi-norm of periodic functions. This follow verbatim of the analysis of the uni-variate case presented in \todo{ref kress singular integral equations}. 

In what follows we denote the non-normalized Fourier basis by $e_n(t) = \exp{(i n t) }$

The fractional semi-norm $|\varrho|_{\mathbf{\gamma}}|$ is generated by the inner product 
\begin{align*}
\langle u , v \rangle{\mathbf{\gamma}} := \int_{-\pi}^{\pi}  \int_{-\pi}^{\pi} \int_{-\pi}^{\pi}\int_{-\pi}^{\pi} \frac{\Delta u (x,t,y,s)\Delta \overline{v} (x,t,y,s) }
{\left(\sin\left(\frac{|x-t|}{2}\right)\right)^{1+2\gamma_1}\left(\sin\left(\frac{|y-s|}{2}\right)\right)^{1+2\gamma_2}}dx dy dt ds, 
\end{align*}
where $,u,v$ are bi-periodic functions and the difference operator is given by.
$$\Delta u(x,t,y,s) = u(x,y)-u(t,y)+u(t,s)-u(x,s).$$
Consider the bi-periodic Fourier basis 
$e_{n,l}(t,s) = e_n(t)e_l(s)$, and let us compute the product between two basis, $e_{n_1,l_1}$ and $e_{n_2,l_2}$.  By permuting the names of the integration variables we have that 
$$\langle e_{n_1,l_1} , e_{n_2,l_2} \rangle{\mathbf{\gamma}}  =
4 \int_{-\pi}^{\pi}  \int_{-\pi}^{\pi} \int_{-\pi}^{\pi}\int_{-\pi}^{\pi} \frac{e_{n_1,l_1}(x,y)\Delta e_{-n_2,-l_2} (x,t,y,s) }
{\left(\sin\left(\frac{|x-t|}{2}\right)\right)^{1+2\gamma_1}\left(\sin\left(\frac{|y-s|}{2}\right)\right)^{1+2\gamma_2}}dx dy dt ds, 
$$
the right most term is decomposed in the sum of forth integrals denoted as
\begin{align*}
I_1 := \int_{-\pi}^{\pi}\int_{-\pi}^{\pi} \int_{-\pi}^{\pi}\int_{-\pi}^{\pi} \frac{e_{n_1,l_1}(x,y)e_{-n_2,-l_2} (x,y) }
{\left(\sin\left(\frac{|x-t|}{2}\right)\right)^{1+2\gamma_1}\left(\sin\left(\frac{|y-s|}{2}\right)\right)^{1+2\gamma_2}}dx dy dt ds,\\
I_2 :=  -\int_{-\pi}^{\pi}\int_{-\pi}^{\pi} \int_{-\pi}^{\pi}\int_{-\pi}^{\pi} \frac{e_{n_1,l_1}(x,y)e_{-n_2,-l_2} (t,y) }
{\left(\sin\left(\frac{|x-t|}{2}\right)\right)^{1+2\gamma_1}\left(\sin\left(\frac{|y-s|}{2}\right)\right)^{1+2\gamma_2}}dx dy dt ds,\\
I_3 := \int_{-\pi}^{\pi}\int_{-\pi}^{\pi} \int_{-\pi}^{\pi}\int_{-\pi}^{\pi} \frac{e_{n_1,l_1}(x,y)e_{-n_2,-l_2} (t,s) }
{\left(\sin\left(\frac{|x-t|}{2}\right)\right)^{1+2\gamma_1}\left(\sin\left(\frac{|y-s|}{2}\right)\right)^{1+2\gamma_2}}dx dy dt ds,\\
I_4 := -\int_{-\pi}^{\pi}\int_{-\pi}^{\pi} \int_{-\pi}^{\pi}\int_{-\pi}^{\pi} \frac{e_{n_1,l_1}(x,y)e_{-n_2,-l_2} (x,s) }
{\left(\sin\left(\frac{|x-t|}{2}\right)\right)^{1+2\gamma_1}\left(\sin\left(\frac{|y-s|}{2}\right)\right)^{1+2\gamma_2}}dx dy dt ds.
\end{align*}

We perform the change of variables $u = t-x$, and $v = s-y$ in the four integrals and by the periodicity of the involving factors we get 
\begin{align*}
I_1 := \delta_{n_1,n_2}\delta_{l_1,l_2}\int_{-\pi}^{\pi}
\left(\sin\left(\frac{|u|}{2}\right)\right)^{-1-2\gamma_1}du
\int_{-\pi}^{\pi}  
\left(\sin\left(\frac{|v|}{2}\right)\right)^{-1-2\gamma_2}dv,\\
I_2 := -\delta_{n_1,n_2}\delta_{l_1,l_2}\int_{-\pi}^{\pi} e_{-n_2}(u)
\left(\sin\left(\frac{|u|}{2}\right)\right)^{-1-2\gamma_1}du
\int_{-\pi}^{\pi}  
\left(\sin\left(\frac{|v|}{2}\right)\right)^{-1-2\gamma_2}dv,\\
I_3 := \delta_{n_1,n_2}\delta_{l_1,l_2}\int_{-\pi}^{\pi} e_{-n_2}(u)
\left(\sin\left(\frac{|u|}{2}\right)\right)^{-1-2\gamma_1}du
\int_{-\pi}^{\pi}  e_{-l_2}(v)
\left(\sin\left(\frac{|v|}{2}\right)\right)^{-1-2\gamma_2}dv,\\
I_4 := -\delta_{n_1,n_2}\delta_{l_1,l_2}\int_{-\pi}^{\pi} 
\left(\sin\left(\frac{|u|}{2}\right)\right)^{-1-2\gamma_1}du
\int_{-\pi}^{\pi}  e_{-l_2}(v)
\left(\sin\left(\frac{|v|}{2}\right)\right)^{-1-2\gamma_2}dv,
\end{align*}
where $\delta_{n,m} = 2 \pi $ if $n=m$ and zero otherwise. Since the Fourier basis are $e_n(t) =  (cos(nt) + i \sin (nt)) $, we can use the symmetries of the cosine and sine function, and we get
$$
\int_{-\pi}^{\pi} (1-e_{-n_2}(u))
\left(\sin\left(\frac{|u|}{2}\right)\right)^{-1-2\gamma_1}du = 
2 \int_{0}^{\pi} (1-\cos(n_2 u))
\left(\sin\left(\frac{|u|}{2}\right)\right)^{-1-2\gamma_1}du,
$$
furthermore we use formula for the cosine of double angle and we get
$$
\int_{-\pi}^{\pi} (1-e_{-n_2}(u))
\left(\sin\left(\frac{|u|}{2}\right)\right)^{-1-2\gamma_1}du = 
4 \int_{0}^{\pi} \sin(\frac{n_2 u}{2}) ^2
\left(\sin\left(\frac{|u|}{2}\right)\right)^{-1-2\gamma_1}du,
$$
using this we get
\begin{align*}
\begin{split}
\langle e_{n_1,n_2}, e_{l_1,l_2} \rangle_{\mathbf{\gamma}}= 16 \delta_{n_1,n_2} \delta_{l_1,l_2}\\ \int_{0}^{\pi} \sin(\frac{n_2 u}{2}) ^2
\left(\sin\left(\frac{|u|}{2}\right)\right)^{-1-2\gamma_1}du 
\int_{0}^{\pi} \sin(\frac{l_2 v}{2}) ^2
\left(\sin\left(\frac{|v|}{2}\right)\right)^{-1-2\gamma_2}du,
\end{split}
\end{align*}
notice that the bi-periodic Fourier functions are orthogonal with this product. Denote by $S_\gamma(n) =\int_{0}^{\pi} \sin(\frac{n}{2} u) ^2
\left(\sin\left(\frac{|u|}{2}\right)\right)^{-1-2\gamma}du $, hence by \todo{kress 8.8?} we have $S_\gamma(n) \cong (n^2)^\gamma$, thus by the previously showed orthogonality  
\begin{align}
\label{eq:smnormeq}
|\varrho |_{\mathbf{\gamma}}^2 = \sum_{n=-\infty}^\infty \sum_{l=-\infty}^{\infty} |\widetilde{\varrho}_{n,l}|^2 S_{\gamma_1}(n) S_{\gamma_2}(l) \cong \sum_{n=-\infty}^\infty \sum_{l=-\infty}^{\infty} |\widetilde{\varrho}_{n,l}|^2 (n^2)^{\gamma_1} (l^2)^{\gamma_2},
\end{align}
 now we can use the traditional inequality  $(1+n^2)^{\gamma_1}  \leq (n^2)^\gamma_1 +1$ (and analogly for $l$ and $\gamma_2$), so we have 
 \begin{align*}
 |\varrho|^2_{\gamma} + \| \varrho\|^2_{L^2([-\pi,\pi]\times [-\pi,\pi])} \geq  \|g\|_{\gamma_1,\gamma_2}^2,
 \end{align*}
 where the left most term is the Sobolev-type norm for orders $\gamma_1,\gamma_2$, defined as in \eqref{eq:sobnormtype}.

\end{document}