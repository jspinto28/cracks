%Typeset with LaTeX
%\documentclass{owrart}
\documentclass[10pt,reqno]{amsart}
% \usepackage[a4paper, total={6in,10in}]{geometry}
\usepackage[letterpaper, margin=1in]{geometry}
%% -------------------------------------------------------------------------------
%% Enter additionally required packages below this comment.

\usepackage{amsmath}
%\usepackage[a4paper]{geometry}
%\usepackage{amsfonts}
\usepackage{amssymb}
\usepackage{amsthm}
\usepackage{mathrsfs}
% \usepackage{amscd}
% \usepackage{latexsym}
% \usepackage{nicefrac} 
% \usepackage{cite}
 %\usepackage{epic}
% \usepackage{eepic}
% \usepackage{fancybox}
\usepackage{graphicx} 
\usepackage{caption}
\usepackage{subcaption}
\usepackage{color}
\usepackage{floatrow}
\floatsetup[table]{capposition=bottom}
% \usepackage{caption} \captionsetup[table]{skip=50pt}

\renewcommand{\thesubfigure}{\roman{subfigure}}
\captionsetup[subfigure]{labelfont=rm}

%Standard operators
\newcommand{\abs}[1]{\lvert #1\rvert}
\newcommand{\supp}{\operatorname{supp}}
\newcommand{\grad}{\operatorname{\mathbf{grad}}}
\newcommand{\curl}{\operatorname{\mathbf{curl}}}
\renewcommand{\div}{\operatorname{div}}
\renewcommand{\ker}{\operatorname{Ker}}
\newcommand{\sgn}{\operatorname{sign}}
\newcommand{\dist}{\operatorname{dist}}
\newcommand{\meas}{\operatorname{\mathsf{meas}}}
\newcommand{\card}{\operatorname{\mathsf{card}}}
\newcommand{\sign}{\operatorname{\mathsf{sig}}}

%fields, mathbb operators
\newcommand{\IR}{\mathbb{R}}
\newcommand{\IC}{\mathbb{C}}
\newcommand{\IZ}{\mathbb{Z}}
%\newcommand{\Q}{\mathbb{Q}}
\newcommand{\IN}{\mathbb{N}}
%\newcommand{\T}{\mathbb{T}}

\newtheorem{theorem}{Theorem}[section]
\newtheorem{corollary}[theorem]{Corollary}
\newtheorem{lemma}[theorem]{Lemma}
\newtheorem{proposition}[theorem]{Proposition}
\newtheorem{definition}[theorem]{Definition}

\theoremstyle{remark}
\newtheorem{remark}[theorem]{Remark}
\newtheorem{assum}[theorem]{Assumption}
\newtheorem{example}[theorem]{Example}
\newtheorem{problem}[theorem]{Problem}
\newtheorem{algorithm}[theorem]{Algorithm}

\newcommand{\norm}[1]{\left\Vert #1 \right\Vert}
\newcommand{\snorm}[1]{\left| #1 \right|}
\newcommand{\dotp}[2]{\left(#1 ,#2 \right)} 
\newcommand{\dual}[2]{\left\langle #1 ,#2 \right\rangle}
\newcommand{\Vx}{\boldsymbol{x}}
\newcommand{\Vy}{\boldsymbol{y}}
\newcommand{\Id}{\mathsf{I}}
\newcommand{\OK}{\mathsf{K}}
\newcommand{\OW}{\mathsf{W}}
\newcommand{\OV}{\mathsf{V}}
\newcommand{\OA}{\mathsf{A}}
\newcommand{\OC}{\mathsf{C}}
\newcommand{\OP}{\mathsf{P}}
\newcommand{\OT}{\mathsf{T}}
\newcommand{\OX}{\mathsf{X}}
\newcommand{\Vm}{\mathcal{V}}
\newcommand{\Ns}{\mathcal{N}}
\newcommand{\ce}{\mathsf{c}}

\newcommand{\OZ}{\mathsf{Z}}

\newcommand{\D}{\text{D}}
\newcommand{\N}{\text{N}}
\newcommand{\I}{\mathfrak{I}}
\newcommand{\E}{\mathfrak{E}}
\newcommand{\SL}[1]{\mathcal{S}_{#1}}
\newcommand{\DL}[1]{\mathcal{D}_{#1}}
\newcommand{\Pt}[1]{\Psi_{#1}}
\newcommand{\Ps}{\mathsf{P}_\sigma}
\newcommand{\Jp}[1]{\left [#1\right]}
\newcommand{\Av}[1]{\left\{#1\right\}}
\newcommand{\spV}{{\bf V}}
\newcommand{\spH}{\mathbb{H}}
\newcommand{\spL}{\mathbb{L}}
\newcommand{\half}{\frac{1}{2}}
\newcommand{\bgamma}{\boldsymbol\gamma}
\newcommand{\blambda}{\boldsymbol\lambda}

\newcommand{\com}[1]{{{\color{red} [#1]}}}
\newcommand{\cj}[1]{{\color{blue} #1}}
\newcommand{\fh}[1]{{\color{magenta} #1}}

\numberwithin{equation}{section}

\begin{document}

\title[]{Bayesian Inversion for Acoustic Scattering by Multiple Open Arcs in 2D}
\author[Fernando Henr\'iquez]{Fernando Henr\'iquez} \address{Seminar for Applied Mathematics, ETH Z$\ddot{\mbox{u}}$rich, CH-8092
  Z$\ddot{\mbox{u}}$rich, Switzerland} \email{fernando.henriquez@sam.math.ethz.ch}
\author[Jos\'e Pinto]{Jos\'e Pinto} \address{School of Engineering, Pontificia
  Universidad Cat\'olica de Chile, Santiago, Chile} \email{jspinto@uc.cl}
\author[Carlos Jerez-Hanckes]{Carlos Jerez-Hanckes} \address{School of Engineering, Pontificia
  Universidad Cat\'olica de Chile, Santiago, Chile} \email{cjerez@ing.puc.cl}
  
\thanks{This work was partially funded by \com{CJ}}

\maketitle

\begin{abstract}
\com{CJ}


\end{abstract}

\newpage

\tableofcontents

\newpage

%%%%%%%%%%%%%%%%%%%%%%%%%%%%%%%%%%%
\section{Introduction}
\label{sec:intro}
%%%%%%%%%%%%%%%%%%%%%%%%%%%%%%%%%%%
\com{CJ}

%%%%%%%%%%%%%%%%%%%%%%%%%%%%%%%%%%%
\section{Boundary Integral Formulation for Multiple Open Arcs in 2D}
\label{sec:BIF_2D}
%%%%%%%%%%%%%%%%%%%%%%%%%%%%%%%%%%%

\com{JP y CJ}

%%%%%%%%%%%%%%%%%%%%%%%%%%%%%%%%%%%
\subsection{Shape Holomorphy of the Boundary Integral Formulation for Multiple Open Arcs in 2D {\com{FH y CJ}}}
\label{ssec:shape_hol}
%%%%%%%%%%%%%%%%%%%%%%%%%%%%%%%%%%%


%%%%%%%%%%%%%%%%%%%%%%%%%%%%%%%%%%%
\section{Bayesian Inversion}
\label{sec:bi}
%%%%%%%%%%%%%%%%%%%%%%%%%%%%%%%%%%%
\com{JP y FH}

%%%%%%%%%%%%%%%%%%%%%%%%%%%%%%%%%%%
\subsection{The Bayesian Framework}
\label{ssec:bayesian_frame}
%%%%%%%%%%%%%%%%%%%%%%%%%%%%%%%%%%%

%%%%%%%%%%%%%%%%%%%%%%%%%%%%%%%%%%%
\subsection{Parametric Regularity of the Posterior}
\label{ssec:parametric_posterior}
%%%%%%%%%%%%%%%%%%%%%%%%%%%%%%%%%%%


%%%%%%%%%%%%%%%%%%%%%%%%%%%%%%%%%%%
\section{Numerical Results}
\label{sec:intro}
%%%%%%%%%%%%%%%%%%%%%%%%%%%%%%%%%%%
\com{JP y FH}


%%%%%%%%%%%%%%%%%%%%%%%%%%%%%%%%%%%
\section{Conclusions and Outlook}
\label{sec:intro}
%%%%%%%%%%%%%%%%%%%%%%%%%%%%%%%%%%%

\com{CJ}





\end{document}
