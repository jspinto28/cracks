%Typeset with LaTeX
%\documentclass{owrart}
\documentclass[10pt,reqno]{amsart}
% \usepackage[a4paper, total={6in,10in}]{geometry}
\usepackage[letterpaper, margin=1in]{geometry}
%% -------------------------------------------------------------------------------
%% Enter additionally required packages below this comment.

\usepackage{amsmath}
%\usepackage[a4paper]{geometry}
%\usepackage{amsfonts}
\usepackage{amssymb}
\usepackage{amsthm}
\usepackage{mathrsfs}
% \usepackage{amscd}
% \usepackage{latexsym}
% \usepackage{nicefrac} 
% \usepackage{cite}
 %\usepackage{epic}
% \usepackage{eepic}
% \usepackage{fancybox}
\usepackage{graphicx} 
\usepackage{caption}
\usepackage{subcaption}
\usepackage{color}
\usepackage{floatrow}
\floatsetup[table]{capposition=bottom}
% \usepackage{caption} \captionsetup[table]{skip=50pt}

\renewcommand{\thesubfigure}{\roman{subfigure}}
\captionsetup[subfigure]{labelfont=rm}

%Standard operators
\newcommand{\abs}[1]{\lvert #1\rvert}
\newcommand{\supp}{\operatorname{supp}}
\newcommand{\grad}{\operatorname{\mathbf{grad}}}
\newcommand{\curl}{\operatorname{\mathbf{curl}}}
\renewcommand{\div}{\operatorname{div}}
\renewcommand{\ker}{\operatorname{Ker}}
\newcommand{\sgn}{\operatorname{sign}}
\newcommand{\dist}{\operatorname{dist}}
\newcommand{\meas}{\operatorname{\mathsf{meas}}}
\newcommand{\card}{\operatorname{\mathsf{card}}}
\newcommand{\sign}{\operatorname{\mathsf{sig}}}

%fields, mathbb operators
\newcommand{\IR}{\mathbb{R}}
\newcommand{\IC}{\mathbb{C}}
\newcommand{\IZ}{\mathbb{Z}}
%\newcommand{\Q}{\mathbb{Q}}
\newcommand{\IN}{\mathbb{N}}
%\newcommand{\T}{\mathbb{T}}

\newcommand{\IH}{\mathbb{H}}
\newcommand{\IT}{\mathbb{T}}
\newcommand{\IQ}{\mathbb{Q}}
\newcommand{\IP}{\mathbb{P}}

\newtheorem{theorem}{Theorem}[section]
\newtheorem{corollary}[theorem]{Corollary}
\newtheorem{lemma}[theorem]{Lemma}
\newtheorem{proposition}[theorem]{Proposition}
\newtheorem{definition}[theorem]{Definition}

\theoremstyle{remark}
\newtheorem{remark}[theorem]{Remark}
\newtheorem{assum}[theorem]{Assumption}
\newtheorem{example}[theorem]{Example}
\newtheorem{problem}[theorem]{Problem}
\newtheorem{algorithm}[theorem]{Algorithm}

\newcommand{\norm}[1]{\left\Vert #1 \right\Vert}
\newcommand{\snorm}[1]{\left| #1 \right|}
\newcommand{\dotp}[2]{\left(#1 ,#2 \right)} 
\newcommand{\dual}[2]{\left\langle #1 ,#2 \right\rangle}
\newcommand{\Vx}{\boldsymbol{x}}
\newcommand{\Vy}{\boldsymbol{y}}
\newcommand{\Id}{\mathsf{I}}
\newcommand{\OK}{\mathsf{K}}
\newcommand{\OW}{\mathsf{W}}
\newcommand{\OV}{\mathsf{V}}
\newcommand{\OA}{\mathsf{A}}
\newcommand{\OC}{\mathsf{C}}
\newcommand{\OP}{\mathsf{P}}
\newcommand{\OT}{\mathsf{T}}
\newcommand{\OX}{\mathsf{X}}
\newcommand{\Vm}{\mathcal{V}}
\newcommand{\Ns}{\mathcal{N}}
\newcommand{\ce}{\mathsf{c}}
\newcommand{\tN}{\text{N}}
\newcommand{\tD}{\text{D}}
\newcommand{\vr}{\mathbf{r}}
\newcommand{\vo}{\mathbf{o}}
\newcommand{\vq}{\mathbf{q}}
\newcommand{\vg}{\mathbf{g}}
\newcommand{\vf}{\mathbf{f}}
\newcommand{\vb}{\mathbf{b}}
\newcommand{\vx}{\mathbf{x}}
\newcommand{\wH}{\widetilde{H}}
\newcommand{\vy}{\mathbf{y}}
\newcommand{\vn}{\mathbf{n}}
\newcommand{\vu}{\mathbf{u}}
\newcommand{\vv}{\mathbf{v}}
\newcommand{\kk}{\kappa}
\newcommand{\OZ}{\mathsf{Z}}

\newcommand{\D}{\text{D}}
\newcommand{\N}{\text{N}}
\newcommand{\I}{\mathfrak{I}}
\newcommand{\E}{\mathfrak{E}}
\newcommand{\SL}[1]{\mathcal{S}_{#1}}
\newcommand{\DL}[1]{\mathcal{D}_{#1}}
\newcommand{\Pt}[1]{\Psi_{#1}}
\newcommand{\Ps}{\mathsf{P}_\sigma}
\newcommand{\Jp}[1]{\left [#1\right]}
\newcommand{\Av}[1]{\left\{#1\right\}}
\newcommand{\spV}{{\bf V}}
\newcommand{\spH}{\mathbb{H}}
\newcommand{\spL}{\mathbb{L}}
\newcommand{\half}{\frac{1}{2}}
\newcommand{\bgamma}{\boldsymbol\gamma}
\newcommand{\blambda}{\boldsymbol\lambda}

\newcommand{\opL}{\mathcal{L}}
\newcommand{\bopL}{\boldsymbol{\mathcal{L}}}
\newcommand{\opK}{\mathcal{K}}
\newcommand{\bopK}{\boldsymbol{\opK}}
\newcommand{\opV}{\mathcal{V}}


\newcommand{\QN}[1]{\IQ_N(\Gamma_{#1})}

\newcommand{\QNZ}[1]{\IQ_{N, \langle 0 \rangle}(\Gamma_{#1})}


\newcommand{\matL}{\mathsf{L}}

\newcommand{\bmatL}{\boldsymbol{\matL}}
\newcommand{\matA}{\mathsf{A}}
\newcommand{\matB}{\mathsf{B}}
\newcommand{\matI}{\mathsf{I}}
\newcommand{\matF}{\mathsf{F}}
\newcommand{\matR}{\mathsf{R}}
\newcommand{\bmatR}{\boldsymbol{\matR}}

\DeclareMathOperator{\Span}{span}


\newcommand{\wGamma}{\widehat{\Gamma}}
\newcommand{\wIH}{\widetilde{\IH}}

\newcommand{\vlambda}{\boldsymbol{\lambda}}
\newcommand{\vvartheta}{\boldsymbol{\vartheta}}
\newcommand{\vphi}{\boldsymbol{\phi}}


\newcommand{\fu}{\frak{u}}
\newcommand{\fg}{\frak{g}}

\newcommand{\vfu}{\boldsymbol{\frak{u}}}
\newcommand{\vfg}{\boldsymbol{\frak{g}}}


\newcommand{\com}[1]{{{\color{red} [#1]}}}
\newcommand{\cj}[1]{{\color{magenta} #1}}
\newcommand{\fh}[1]{{\color{green} #1}}
\newcommand{\jp}[1]{{\color{blue} #1}}

\numberwithin{equation}{section}

\begin{document}

\title[]{Uncertainty Quantification Acoustic Scattering by Multiple Open Arcs in 2D}
\author[Fernando Henr\'iquez]{Fernando Henr\'iquez} \address{Seminar for Applied Mathematics, ETH Z$\ddot{\mbox{u}}$rich, CH-8092
  Z$\ddot{\mbox{u}}$rich, Switzerland} \email{fernando.henriquez@sam.math.ethz.ch}
\author[Jos\'e Pinto]{Jos\'e Pinto} \address{School of Engineering, Pontificia
  Universidad Cat\'olica de Chile, Santiago, Chile} \email{jspinto@uc.cl}
\author[Carlos Jerez-Hanckes]{Carlos Jerez-Hanckes salud!} \address{School of Engineering, Pontificia
  Universidad Cat\'olica de Chile, Santiago, Chile} \email{cjerez@ing.puc.cl}
  
\thanks{This work was partially funded by \com{CJ}}

\maketitle

\begin{abstract}
\com{CJ}


\end{abstract}

\newpage

\tableofcontents

\newpage

%%%%%%%%%%%%%%%%%%%%%%%%%%%%%%%%%%%
\section{Introduction}
\label{sec:intro}
%%%%%%%%%%%%%%%%%%%%%%%%%%%%%%%%%%%

%%%%%%%%%%%%%%%%%%%%%%%%%%%%%%%%%%%
\section{Boundary Integral Formulation for Multiple Open Arcs in 2D}
\label{sec:BIF_2D}
%%%%%%%%%%%%%%%%%%%%%%%%%%%%%%%%%%%

In this section we reformulate the direct scattering problem, into a system of boundary integral equations defined on the arcs. We also provide a numerical method to find approximations to the solutions of the later. 

Following the classical theory of integral operators (REF SAUTER) the solution of Problem CROSREF can be written as 

\begin{equation}
\label{eq:intrep}
U(\vx) =  \sum_{i=1}^M  (\SL{\Gamma_i}[\kappa]  \lambda_i)(\vx),\quad \forall \ \vx\in\Omega,
\end{equation}
where 
$$(\SL{\Gamma_i} [\kappa]  \lambda_i)(\vx):= \int_{\Gamma_i}G_\kappa(\vx,\vy)  \lambda_i(\vy)  d\Gamma_i(\vy)$$
denotes the single layer potential generated by the arc $\Gamma_i$ 
with fundamental solution $G_\kk$ given by \cite[Section 3.1]{sauter2010boundary}:
\begin{equation}
\label{eq:FunSol}
G_\kappa(\vx,\vy) = 
\dfrac{i}{4}H^{1}_0(\kk\|\vx-\vy\|_2).
\end{equation}
The definition of the fundamental solution ensure that $U \in H^1_{loc}(\Omega)$ is a radiatie Helmholtz solution for any vector of densities $\vlambda : \lambda_i \in \widetilde{H}^{-\half}(\Gamma_i)$. The densities are selected by imposing the boundary conditions which read as, 
$$\sum_{j =1}^M \opL_{ij}[\kappa] \lambda_j  = g_i \quad i \in  \{1,2 \hdots M\},$$
where $\opL_{ij}[\kappa]$ is the weakly singular integral operator obtained by taking the trace of $\SL{\Gamma_i} [\kappa]$ over $\Gamma_j$. Hence the following Problem is equivalent to the scattering problem, when  $\vg$ is the Dirichlet data in CROSREF.

\begin{problem}[Boundary Integral Problem]
\label{prob:BIE}
Let $\vg \in \IH^{\half}(\Gamma)$. For $\kk>0$, we seek $\vlambda=(\lambda_1,\ldots,\lambda_M)\in\widetilde{\IH}^{-\half}(\Gamma)$ such that
\begin{equation}
\bopL[\kappa] \vlambda = \vg,
\end{equation}
where $
\bopL[\kappa]:\widetilde{\IH}^{-\half}(\Gamma)\rightarrow{\IH}^{\half}(\Gamma)$ is a matrix operator, with entries $\bopL[\kk]_{ij} = \opL_{ij}[\kk]$, for $i,j \in \{1,\hdots M\}$.
%In the case $\kk=0$, we look for $\vlambda\in\widetilde{\IH}^{-\half}_{\langle0\rangle}(\Gamma)$.
\end{problem}

\begin{theorem}[Theorem 4.13 in \cite{PaperA}] \label{theo:invL} 
For $\kk>0$, Problem \ref{prob:BIE} has a unique solution $\vlambda \in \widetilde{\IH}^{-\half}(\Gamma)$. Also, the continuity estimate holds
\begin{equation}
\| \vlambda \|_{\widetilde{\IH}^{-\half}(\Gamma)} \leq C(\Gamma,\kk) \| \vg \|_{\IH^{\half}(\Gamma)}.
\end{equation}
\end{theorem}

\subsection{Spectral Discretization}

As it was pointed in the introduction, in order to approximate the statistical moments of a quantity that depends of the solution $U$ of Problem REF \jp{esta bien?}, we need to approximate the solution of the direct scattering problem for large number of geometric configurations. This approximations can be constructed by means of a Galerkin discretization REF of Problem \ref{prob:BIE}. 

Since numerous instances of this Galerkin discretization are needed, one has to relay in discretization which  convergence fast. In particular we will use the spectral method proposed in REFPAPERB, that was shown to converge super-algebraically. In what follows we detail which are the basis for the spectral method and review the convergence result. 

 Let $\IT_N(\wGamma) = \Span \{ T_n \}_{n=0}^N$, where $T_n$ denote the n-degree first kind Chebyshev, polynomials orthogonal under the weight $w^{-1}(t) = (1-t^2)^{-1/2}$ over $(-1,1)$. Now, let us construct elements $p^i_n = T_n \circ \vr_i^{-1}$ over each arc $\Gamma_i$ spanning the space $\IT_N(\Gamma_i)$.  We define the normalized space:
\begin{equation}
\overline{\IT}_N(\Gamma_i) := \left\{ \bar{p}^i {\ \in C(\Gamma_i)}  : \ \bar{p}^i_n := \frac{p^i_n}{\norm{\vr_i' \circ \vr^{-1}_i}_{2}}\ , \quad p^i_n \in \IT_N(\Gamma_i)\right\}.
\end{equation}
Since the solutions $\vlambda$ exhibit singular behavior at the end points of each arc, we consider the weighted polynomial space: 
\begin{equation}
\label{eq:QN}
\IQ_N(\Gamma_i) := \left\{ q^i_n:=w_i^{-1} \bar{p}^i_n  : \ \bar{p}^i_n \in \overline{\IT}_N(\Gamma_i) \right\},
\end{equation}
wherein $w_i:=w\circ \vr^{-1}_i$. The corresponding basis for ${\IQ}_N(\Gamma_i)$ will be denoted $\{q^i_n\}_{n=0}^N$. 

Now the discretization of Problem \ref{prob:BIE} is seek for a coefficient vector $\vfu=(\vfu_1,\ldots,\vfu_M)\in\mathbb{C}^{M(N+1)}$, such that 
\begin{equation}\label{eq:linsys}
\bmatL[\kappa] \vfu = \vfg,
\end{equation}
where $\bmatL[\kappa]$ is the matrix corresponding to the discretization of operator $\opL[\kappa]$, given by

\begin{equation}\label{eq:matLij}
(\matL_{ij}[\kappa])_{lm} = \dual{\opL_{ij}[\kappa] q_m^j}{q_l^i}_{\Gamma_i} 
=  \dual{\widehat{\opL}_{ij}[\kappa] w^{-1}T_m}{w^{-1}T_l}_{\wGamma},
\end{equation}
 with $\widehat{\opL}_{ij}[\kappa]$ being the weakly-singular operator whose kernel is parametrized by $\vr_i, \vr_j$ and right-hand $\vfg=(\vfg_1,\ldots,\vfg_M) \in \mathbb{C}^{M(N+1)}$ with components $$(\vfg_i)_l = \dual{g_i}{q_l^i}_{\Gamma_i}= \dual{\widehat{g}_i}{w^{-1}T_l}_{\wGamma},$$ where $\widehat{g}_i = g_i \circ \vr_i$. 
 
The approximation $\vlambda_N$ of $\vlambda$  is given by 
\begin{equation}
\label{eq:Lambapprox}
(\vlambda_N)_i = \sum_{m=0}^N (\vfu_i)_m q^i_m \ \mbox{ in } \Gamma_i , \quad\mbox{for all} \ i \in \{ 1, \ldots ,M\}.
\end{equation} 

\begin{theorem}[Theorem 4.23 \cite{PaperB}]
\label{teo:discerror}
Let $\kk > 0$, $m \in \mathbb{N}$ with $m>2$, $\Gamma \in \mathcal{C}^m$, $\vg \in \mathcal{C}^m(\Gamma)$, and $\vlambda$ be the only solution of Problem \ref{prob:BIE}. Then, there exists $N_0 \in \mathbb{N}$ such that for every $N> N_0 \in \mathbb{N}$ there is a unique $\vlambda_N$, constructed as in \eqref{eq:Lambapprox} that converge as

$$\norm{\vlambda - \vlambda_N }_{\widetilde{\IH}^{-\half}(\Gamma)} \leq C(\Gamma,\kappa) N^{-m+1};$$
 Moreover, if $\Gamma$, and $\vg$ are analytic, there exists $\rho >1$ such that 
$$\norm{\vlambda - \vlambda_N }_{\widetilde{\IH}^{-\half}(\Gamma)} \leq C(\Gamma,\kappa) \rho^{-N+2}\sqrt{N}.$$
with $C(\Gamma,\kappa)$ being a positive constant.
\end{theorem} 

\subsection{Numerical Implementation}
\label{sec:NumImplemntation}

Now we explain how $\bmatL[\kappa] $ and $\vfg$ of Equation \eqref{eq:linsys} are obtained. The vector $\vfg$  is composed of integrals of the form:
$$\int_{-1}^{1} \widehat{g}(t) w^{-1}(t) T_l(t) dt, \quad \forall \ l \in \{0,..,N\}$$ 
this set of integrals can be computed using FFT as is detailed in \cite{trefethen2013approximation}. We remark that cited technique computes the full set of integrals simultaneously. 

The matrix terms $\matL_{ij}[\kappa]$ computations are split into two groups: (a) \emph{cross-interactions}, which consist of non singular integral and correspond to the case $j \neq i$. (b) \emph{self-interactions}
which are weakly singular integrals correspond to the diagonal block of the matrix ($i=j$). 

For the first group we use the same procedure as of the right-hand side, while for the second we can characterize the kernel singularity as 
$$G_k(\vr(t), \vr(s) ) = (2\pi)^{-1} \log |t-s| J_0(k \| \vr(t)- \vr(s) \|_2 ) + G_r(t,s), \quad t\neq s,$$
with $t,s\in (-1,1)$, and where $J_0$ is the zeroth-order first kind Bessel function, and $G_r$ is a regular function. Integration for the regular part is done as in the cross-interaction case, while integrals with the first term are obtained by convolution as integrals for $\log |t-s|$ have analytic expressions (see \cite[Remark 4.2]{Jerez-Hanckes2017}).

\subsection{Matrix Compression}

One extra feature of the discrete space, utilized to construct the Galerkin discretization, is that we can obtain an sparse approximation of the dense matrix $\bmatL[\kappa]$. 
The key ingredient to compress the matrix, is to approximate the cross interaction blocks by just a few entries. This is justified by the fact the kernel of the cross-interaction terms is an smooth function, thus the integrals of the form \eqref{eq:matLij} decay super-algebracly with respect to the indices $m$ and $l$ (see REFPAPERB). 
The detailed algorithm of how the cross-interaction blocks are compressed, as well some results concerning the accuracy that is lost are in REF.

For our context is important to consider that once cross-interaction blocks are compressed we can accelerate the matrix-vector product, and so the solution of the linear system by means of an iterative method.  
  

%%%%%%%%%%%%%%%%%%%%%%%%%%%%%%%%%%%
\section{Shape Holomorphy of the Boundary Integral Formulation for Multiple Open Arcs in 2D}
\label{ssec:shape_hol}
%%%%%%%%%%%%%%%%%%%%%%%%%%%%%%%%%%%


%%%%%%%%%%%%%%%%%%%%%%%%%%%%%%%%%%%
\section{Numerical Results}
\label{sec:intro}
%%%%%%%%%%%%%%%%%%%%%%%%%%%%%%%%%%%


%%%%%%%%%%%%%%%%%%%%%%%%%%%%%%%%%%%
\section{Conclusions and Outlook}
\label{sec:intro}
%%%%%%%%%%%%%%%%%%%%%%%%%%%%%%%%%%%

\com{CJ}





\end{document}
